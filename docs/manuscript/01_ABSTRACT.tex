% ABSTRACT

\justify Mutation of the WASH complex subunit, SWIP, is implicated in human
intellectual disability, but the cellular etiology of this association is
unknown. We identify the neuronal WASH complex proteome, revealing a network of
endosomal proteins. To uncover how dysfunction of endosomal SWIP leads to
disease, we generate a mouse model of the human
WASHC4\textsuperscript{c.3056C>G} mutation.  Quantitative spatial proteomics
analysis of SWIP\textsuperscript{P1019R} mouse brain reveals that this mutation
destabilizes the WASH complex and uncovers significant  perturbations in both
endosomal and lysosomal pathways.  Cellular and histological analyses confirm
that SWIP\textsuperscript{P1019R} results in  endo-lysosomal disruption and
uncover indicators of neurodegeneration. We find that
SWIP\textsuperscript{P1019R} not only impacts cognition, but also causes
significant progressive motor deficits in mice.  Remarkably, a retrospective
analysis of SWIP\textsuperscript{P1019R} patients confirms motor deficits in
humans. Combined, these findings support the model that WASH complex
destabilization, resulting from SWIP\textsuperscript{P1019R}, drives cognitive
and motor impairments via endo-lysosomal dysfunction in the brain.
