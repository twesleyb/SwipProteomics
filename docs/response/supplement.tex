% title: response.tex
% description: Response to eLife Reviewers
% author: twab

% USAGE: 
% to compile this document:
%    R  >>> knitr::knit("response.Rnw")
%    sh >>> pdflatex response.tex


%% R --------------------------------------------------------------------------



%% latex document setup -------------------------------------------------------

\documentclass[11pt]{elife}
\usepackage{amsmath}
\usepackage{amssymb}
\usepackage{amsthm}
\usepackage{ragged2e}
\usepackage{caption}
\usepackage{fancyhdr}
\usepackage{graphicx}
\usepackage{titlesec}
\usepackage{blkarray}
\usepackage{csquotes}

\graphicspath{ {./figs/} }


\title{Supplementary Methods\\
\small{Genetic Disruption of WASHC4 Drives Endo-lysosomal Dysfunction and \\
Cognitive-Movement Impairments in Mice and Humans}}

\author[1\authfn{0}]{Jamie Courtland}
\author[1\authfn{0}]{Tyler W. A. Bradshaw}
\author[2]{Greg Waitt}
\author[2,3]{Erik J. Soderblom}
\author[2]{Tricia Ho}
\author[4]{Anna Rajab}
\author[5]{Ricardo Vancini}
\author[2\authfn{1}]{Il Hwan Kim}
\author[6]{Ting Huang}
\author[6]{Olga Vitek}
\author[3]{Scott H. Soderling}

\affil[1]{Department of Neurobiology, Duke University School of Medicine, 
Durham, NC 27710, USA}
\affil[2]{Proteomics and Metabolomics Shared Resource, 
Duke University School of Medicine, Durham, NC 27710, USA}
\affil[3]{Department of Cell Biology, Duke University School of Medicine, 
Durham, NC 27710, USA}
\affil[4]{Burjeel Hospital, VPS Healthcare, Muscat, Oman}
\affil[5]{Department of Pathology, Duke University School of Medicine, 
Durham, NC 27710, USA}
\affil[6]{Khoury College of Computer Sciences, Northeaster University,
Boston, MA 02115, USA}

\contrib[\authfn{0}]{These authors contributed equally to this work.}
\presentadd[\authfn{1}]{Department of Anatomy and Neurobiology, 
University of Tennessee Health Science Center, Memphis, TN 38163, USA}

\corr{jlc123@duke.edu}{JC}
\corr{tyler.w.bradshaw@duke.edu}{TWAB}
\corr{greg.waitt@duke.edu}{GW}
\corr{erik.soderblom@duke.edu}{EJB}
\corr{tricia.ho@duke.edu}{TH}
\corr{drannarajab@gmail.com}{DR}
\corr{ricardo.vancini@duke.edu}{RV}
\corr{ikim9@uthsc.edu}{IK}
\corr{huang.tin@northeastern.edu}{TH}
\corr{o.vitek@northeastern.edu}{OV}
\corr{scott.soderling@duke.edu}{SHS}


\setlength{\abovedisplayskip}{3pt}
\setlength{\belowdisplayskip}{3pt}


%% main -----------------------------------------------------------------------
\begin{document}

\maketitle

\renewcommand{\abstractname}{Summary}
\begin{abstract}

Here we address concerns about the statistical validity of our previous approach
to assess differential protein abundance in the \textbf{WASH-iBioID} and
\textbf{SWIP-TMT} proteomics datasets. Our previous approach depended
upon the R package \texttt{edgeR}. We used \texttt{edgeR} to perform
both protein- and module-level inference---assessing differential
abundance of individual proteins as well as protein groups in
SWIP\textsuperscript{P1019R} mouse brain. \texttt{edgeR} utilizes a
negative binomial (NB) generalized linear model (GLM) framework
originally developed for analysis of RNA-Seq data.  Previously, we
failed to fully consider the validity of \texttt{edgeR's} NB assumption
for proteomics data. We evaluate the goodness-of-fit of the NB GLM for
our TMT dataset and find evidence of a lack-of-fit.  Thus, we revise our
statistical approach and reanalyze our data, making use of Huang
\textit{et al.}'s recently published R package \texttt{MSstatsTMT} for analysis
of TMT mass spectrometry.  We extend the flexible
linear-mixed model (LMM) framework used by \texttt{MSstatsTMT} to
re-evaluate both protein- and module-level statistical comparisions in
our SWIP-TMT spatial proteomics dataset.\\

\end{abstract}

\newpage


\section{Lack-of-fit of the NB model}

Our previous approach can be summarized as the 'Sum + IRS' method (Huang2020).
Following protein summarization and internal
reference scaling (IRS) normalization (Plubell2017),  we applied
\texttt{edgeR} to assess differential abundance of individual proteins and
protein-groups.  The use of \texttt{edgeR} for protein-level comparisons was
based on work by Plubell \textit{et al.} who describe IRS normalization and the
use of \texttt{edgeR} for statistical testing in TMT MS experiments
(Plubell2017).  We failed however, to consider the overall adequacy of the NB
GLM model for our TMT proteomics data.

Statisitical inference in \texttt{edgeR} is performed for each gene or protein 
using a negative binomial framework. The data are assumed to 
be adequately described by a NB distribution parameterized by a dispersion 
parameter, $\phi$. Practically, the dispersion parameter accounts for the
observed mean-variance relationship in proteomics and transcriptomics data.

As signal intensity in protein MS is fundamentally related to the number of ions generated from
an ionized, fragmented protein, we incorrectly inferred that TMT
mass spectrometry data can be modeled as negative binomial count data. Based on
this assumption, we justified the use of \texttt{edgeR}.  Here, we reconsider
the overall adequacy of the \texttt{edgeR} NB GLM model for TMT MS data.

To evaluate the overall adequacy of the \texttt{edgeR} model, we plot the
residual protein deviance statistics of all proteins against their theoretical,
normal quantiles in a quantile-quantile (QQ) plot (FIG:gof).  The QQ plot
addresses the question of how similar the observed data are to the theoretical
distribution.  A linear relationship between the observed and theoretical
values is an indicator of goodness-of-fit.  Deviation from this linear trend is
evidence of a lack-of-fit.

Following protein summarization and normalization with \texttt{MSstatsTMT}, the
SWIP-TMT data were fit with a NB GLM using \texttt{edgeR::glmFit}. FIG:gof illustrates
the divergence of the observed and theoretical quantiles for our SWIP-TMT
dataset fit with \texttt{edgeR's} NB GLM.

Given our experimental design, \texttt{MSstatsTMT} fits an
appropriate linear-mixed model expressing the major sources of variation in our
experiment.  The quantile-quantile plot in FIG:gof indicates that the data are
well described by \texttt{MSstatsTMT's} LMM, which does not depend
upon the negative binomial assumption.

% NOTE: we also assess the gof of the Khan TMT dataset using the IRS method and
% observe a similar lack of fit for the NB GLM.
%These plots emphasize the overall lack-of-fit for proteomics data fit with the \texttt{edgeR} model.\\ 

\section{Statistical Inference with MSstatsTMT}

The strength of linear mixed-models lies in their flexibility. In LMMs 
the response variable is taken to be a function of both fixed- and random-effects. 
If the set of possible levels of a covariate is fixed and reproducible, then the
factor is modeled as a fixed-effect parameter.  In contrast, if the levels of an
observation reflect a sampling of the set of all possible levels, then the
covariate is modeled as a random-effect.  Random or mixed-effects represent
categorical variables that reflect experimental or observational units within
the dataset (Bates2015).  As such, mixed-effect parameters account for the
variation occurring among the lower levels of an upper level unit in the data
(Bates2015).  Using LMMs we can untangle the variance attributable to the
biological effect we are interested in from the experimental and biological
covariates which mask this response.

Huang \textit{et al.} created \texttt{MSstatsTMT}, an R package for data
normalization and hypothesis testing in multiplex TMT proteomics experiments. 
They outline a common vocabulary for describing the experimental design of 
a general TMT mass spectrometry experiment. An experiment consists of 
\texttt{m = 1} ... \texttt{M}\ concatenations of isobarically labeled samples or
\texttt{Mixtures}.  This mixture is then analyzed by the mass spectrometer in a
mass spectrometry \texttt{Run}.  This mixture is often fractionated into
multiple liquid chromotography \texttt{Fractions} to decrease sample complexity,
and thereby increase the depth of proteome coverage.  Within a mixture, each of
the unique TMT channels is dedicated to the analysis of \texttt{c = 1} ...
\texttt{C}\ individual biological or treatment \texttt{Conditions}.  There may
then be \texttt{b = 1} ... \texttt{B}\ biological replicates or
\texttt{Subjects}. Finally, a single TMT mixture may be repeatedly analyzed in
\texttt{t = 1} ... \texttt{T}\ technical replicate mass spectrometry runs.

Equation \ref{eq:full} is a LMM describing protein abundance measured in a general TMT
experiment composed of  \texttt{M} mixtures, \texttt{T} technical replicates of
mixture, \texttt{C} conditions, and \texttt{B} biological subjects.
\begin{equation} % eq:full
  \label{eq:full} 
	Y_{mcbt} = \mu + Mixture_m + TechRep(Mixture)_{m(t)} + Condition_c + 
	Subject_b + \epsilon_{mcbt}\\
\end{equation}

\begin{equation}
  \begin{gathered}
    \label{eq:constraints}
	\sum_{c=1}^{C} Condition_c = 0 \\
	Subject_{mcb} \stackrel{iid}{\sim} N(0,\sigma^2_S) \\
	Mixture_m \stackrel{iid}{\sim} N(0,\sigma^2_M) \\
	TechRep(Mixture)_{t(m)} \stackrel{iid}{\sim} N(0,\sigma^2_T) \\
	\epsilon{mtcb} \stackrel{iid}{\sim} N(0,\sigma^2) \\
  \end{gathered}
\end{equation}

The model's constraints \ref{eq:constraints} distinguish fixed- and mixed-effect
components of variation in the response, $Y_{mcbt}$. \texttt{Mixture} is a
mixed-effect and represents the variation between TMT mixtures. By definition
mixed-effects are assumed to be independent and normally distributed
(\textit{iid}). \texttt{TechRep(Mixture)} represents random variation between
replicates of a single MS \texttt{Run}.  The term \texttt{Subject} cooresponds
to each unique biological replicate and represents biological variation among
the levels of the fixed-effect term \texttt{Condition}. The term
$\epsilon_{mtcb}$, is a mixed-effect representing both biological and technical
variation, quantifying any remaining error. If a component of the model is not
estimable, then it is removed.  For example, if there is no technical
replication of mixture \texttt{(T=0)}, 
then the model is reduced to equation \ref{eq:reduced}.
\begin{equation} % NOTE: dont put blank lines above equations!
	\label{eq:reduced} % equation -- reduced
	Y_{mcbt} = \mu + Mixture_m + Condition_c + Subject_b + \epsilon_{mcb}
\end{equation}

\texttt{MSstatsTMT} performs protein-wise comparisons between pairs of 
\texttt{Conditions} by comparing the estimates obtained from the fit LMM. 
We are interested in testing the hypothesis:
\begin{equation}
	\label{eq:null} % equation -- null
	H0 : l^T * \beta = 0. 
\end{equation}

Where $l^T$ is a vector of $\sum=1$ specifying the positive and negative
coefficients of a contrast. $\beta$ is the model-based estimates of
\texttt{Condition}.  The null hypothesis (\ref{eq:null}) is that the fold
change, $\l^T * \beta$, is 0.  A test statistic for such a two-way contrasts is
given by Kutzenova \textit{et al.,} (Kutzenova2017):
\begin{equation} 
	\label{eq:tstatistic} % equation -- tstatistic
	t = \frac{l^T \hat{\beta}}{\sqrt{l \sigma^2 \hat{V} l^T}}
\end{equation}

We obtain the models estimates $\hat{\beta}$, error $\sigma^2$, and
variance-covariance matrix $\hat{V}$ from the model fitted by restricted maximum
likelihood (Bates2015). Given a contrast, $l^T$, the numerator of equation
(\ref{eq:tstatistic}) is the fold change of a comparison.  The product of
$\sigma^2$ and $\hat{V}$ is the scaled variance-covariance matrix describing
error estimates of the model's fixed- and mixed-effect parameters.  Together the
denominator represents the standard error of the comparison. The degrees of
freedom for the contrast are derived using the Satterthwaite moment of
approximation method (Satterthwaite1946, Kutzenova2017).  Finally, a p-value is
calculated given the t-statistic and degrees of freedom.  P-values for the
protein-wise tests are adjusted using the Benjamini-Hochberg FDR method
(Benjamini1995,Huang2020).\\


\section{SWIP-TMT Experimental Design}

Each 16-plex TMT mixture contains seven repeated measurements made from each
biological subject (FIG:design).  To account for this repeated measures
design, we should include the random-effect term \texttt{Subject}.
In our experiment however, \texttt{Mixture} is confounded with \texttt{Subject}.
In each \texttt{Mixture} we analyzed all seven \texttt{BioFractions} from a
single Control and Mutant mouse. Thus we can choose to
account for the effect of \texttt{Mixture} or \texttt{Subject}, but not both. We
choose to account for variability of \texttt{Mixture} based on the assumption
that the variance associated with this experimental batch effect is greater than
the intra-Subject error inherent in the repeated measures of each subject.
In our experiment, the fixed-effect term \texttt{Condition} in equation
\ref{eq:reduced} represents the fourteen combinations of \texttt{Genotype} and
\texttt{BioFraction} obtained from subcellular fractionation of Control and
SWIP\textsuperscript{P1019R} Mutant mouse brains. We refer to these as a
\texttt{BioFraction} to distinguish them from an MS \texttt{Fraction}. 
We omit the un-estimable terms \texttt{TechRep(Mixture)} and \texttt{Subject}
from equation (\ref{eq:full}). The reduced model is equation \ref{eq:fx0}.
\begin{equation}
	\label{eq:fx0}
	Y_{mcbt} = \mu + Mixture_m + Condition_c + \epsilon_{mcb}
\end{equation}


\section{Protein-level comparisions}

Following data preprocessing, summarization, and normalization, statistical
inference by \texttt{MSstatsTMT} is performed by (1) fitting each protein in the
dataset with an appropriate LMM and then (2) given the fitted model, assessing a
contrast of interest. Using \texttt{MSstatsTMT} we assesssed two types of
protein comparisons:
\begin{itemize}
	\item \texttt{intra-BioFraction} comparisons
	\item \texttt{Mutant-Control} contrast
\end{itemize}

\texttt{Intra-BioFraction} comparisons are the seven pairwise comparisons of 
Control and Mutant protein abundance for each subcellular
\texttt{BioFraction}. We also assessed differential abundance for the 
overall \texttt{Mutant-Control} comparison. Each of these contrasts is 
represented by a vector, $l^T$, which specifies a comparison between 
coefficients of \texttt{Condition} in the LMM (\ref{eq:fx0}).
FIG:contrasts illustrates a matrix defining all eight unique comparisons.

\texttt{MSstatsTMT} attempts to automatically parse the experimental design and
fit the appropriate LMM to each protein in the dataset. In order to understand 
and extend the function of \texttt{MSstatsTMT}, 
we extracted \texttt{MSstatsTMT's} core model-fitting and statistical 
testing steps and illustrate them here.

At the core of the model fitting-step is the R package \texttt{lme4} which
implements mixed-effects models with its function \texttt{lme4::lmer}(Bates2015). The
package \texttt{lmerTest} extends \texttt{lme4's} functionality and enables the
computation of Sattertwaite degrees of freedom (Kutzenova2017). As an example,
we illustrate the analysis of WASHC4. First, we fit the model (\ref{eq:fx0}) to
a subset of the data, the data for WASHC4. 


%% R ---------------------------------------------------------------------------



%% latex -----------------------------------------------------------------------
The model's estimates ($\beta$) represent our best estimate of the mean protein
abundance in the fourteen conditions of \texttt{Genotype:BioFraction}. 
To illustrate an \texttt{intra-BioFraction} comparison, we 
define a contrast comparing the \texttt{Mutant:F7} and \texttt{Control:F7}
conditions. The function \texttt{lmerTestContrast} performs the statstical comparison given
a fitted model and a contrast vector defining a comparison between the models
coefficients. While the work done by this function 
is the same as the work done internally by \texttt{MSstatsTMT's}
\texttt{groupComparisonsTMT} function, \texttt{lmerTestContrast} is more
flexible. Provided the correct contrast, we also easily assess the overall
\texttt{Mutant-Control} comparison.\\


%% R --------------------------------------------------------------------------
\begin{knitrout}
\definecolor{shadecolor}{rgb}{0.969, 0.969, 0.969}\color{fgcolor}\begin{kframe}
\begin{alltt}
\hlcom{# create a contrast}
\hlstd{coeff} \hlkwb{<-} \hlstd{lme4}\hlopt{::}\hlkwd{fixef}\hlstd{(fm0)}
\hlstd{contrast7} \hlkwb{<-} \hlkwd{setNames}\hlstd{(}\hlkwd{rep}\hlstd{(}\hlnum{0}\hlstd{,}\hlkwd{length}\hlstd{(coeff)),} \hlkwc{nm} \hlstd{=} \hlkwd{names}\hlstd{(coeff))}
\hlstd{contrast7[}\hlstr{"GenotypeMutant:BioFractionF7"}\hlstd{]} \hlkwb{<-} \hlopt{+}\hlnum{1} \hlcom{# positive coeff}
\hlstd{contrast7[}\hlstr{"GenotypeControl:BioFractionF7"}\hlstd{]} \hlkwb{<-} \hlopt{-}\hlnum{1} \hlcom{# negative coeff}

\hlcom{# evaluate contrast}
\hlkwd{lmerTestContrast}\hlstd{(fm0, contrast7)}
\end{alltt}
\end{kframe}
\end{knitrout}

\begin{knitrout}
\definecolor{shadecolor}{rgb}{0.969, 0.969, 0.969}\color{fgcolor}\begin{kframe}
\begin{alltt}
\hlcom{# use convenience function to contruct a contrast}
\hlstd{contrast8} \hlkwb{<-} \hlkwd{getContrast}\hlstd{(fm0,} \hlstr{"Mutant"}\hlstd{,}\hlstr{"Control"}\hlstd{)}

\hlcom{# assess the comparison}
\hlkwd{lmerTestContrast}\hlstd{(fm0, contrast8)}
\end{alltt}
\end{kframe}
\end{knitrout}


%% latex -----------------------------------------------------------------------
\section{LMM Goodness-of-fit}

It is useful to consider the goodness-of-fit of our LMM. A straight forward
measure of a LMM's quality is the Nakagawa coefficient of 
determination (Nakagawa2013,Nakagawa2017). Nakagawa's conditional $R^2$ is 
interpreted as the total variance explained by a LMM ($R^2_{total}$).
The marginal $R^2$ is interpreted as the variance explained by the LMM's 
fixed-effects ($R^2_{fixed}$). We implment Nakagawa's coeffficient of 
determination using the \texttt{r.squaredGLMM} function taken from the 
\texttt{MuMin} package (WangMerkel2018).\\


%% R --------------------------------------------------------------------------
\begin{knitrout}
\definecolor{shadecolor}{rgb}{0.969, 0.969, 0.969}\color{fgcolor}\begin{kframe}
\begin{alltt}
\hlcom{# assess gof with Nakagawa coefficient of determination}
\hlkwd{r.squaredGLMM.merMod}\hlstd{(fm0)}
\end{alltt}
\end{kframe}
\end{knitrout}


%% latex -----------------------------------------------------------------------
The total variation explained, $R^2_{c}$, for the LMM fit to WASHC4 is 
\texttt{0.949}. The variance explained by fixed-effects, represents a large
fraction of this total ($R^2{m}$=0.935). It follows that 1.5\% of the remaing
variance is attributable to residuals and the mixed-effect \texttt{Mixture}.\\


\section{Module-level analysis}

We wish to extend the LMM framework developed by \texttt{MSstatsTMT} to perform 
inference at the level of protein groups or modules.
That is, for module-level comparisons, we are interested in the overall affect 
of Genotype on a group of proteins. Where modules are groups of covarying 
proteins which represent biological niches defined by proteins that 
localized together in subcellular space.\\

Here we hypothesize that the proteins within a module, which are a subset of the
overall proteome,  are a part of a common group, a module, with a common mean
effect. Proteins within a module are correlated observations which we model as a
mixed-effect as we are primarily interested in making inference about the
overall distribution of the responses for a module rather than among its
sublevels. The following LMM includes the additional mixed effect term
\texttt{Protein}, capturing variance among a module's constintuent proteins.\\

\begin{equation} 
  \begin{gathered}\label{eq:fx1} % equation -- fx1
	Y_{mcbt} = \mu + Mixture_m + Condition_c + Protein_p + \epsilon_{mcb}\\
	Protein_p \stackrel{iid}{\sim} N(0,\sigma^2_P) \\
  \end{gathered}
\end{equation}

The term \texttt{Protein} quantifies the variance $\sigma_P$ attributable to all
proteins in a module.  As a means of example, we demonstrate an ideal module, by
fitting LMM (\ref{eq:fx1}) to the five WASH complex proteins.  As before, we
calculate the coefficient of determination for LMM's with the
\texttt{r.squaredGLMM} function (WangMerkel2018).\\


%% R --------------------------------------------------------------------------
\begin{knitrout}
\definecolor{shadecolor}{rgb}{0.969, 0.969, 0.969}\color{fgcolor}\begin{kframe}
\begin{alltt}
\hlcom{# the module-level formula to be fit:}
\hlstd{fx1} \hlkwb{<-} \hlstr{'Abundance ~ 0 + Condition + (1|Mixture) + (1|Protein)'}

\hlcom{# load WASH Complex proteins}
\hlkwd{data}\hlstd{(washc_prots)}

\hlstd{fm1} \hlkwb{<-} \hlkwd{lmer}\hlstd{(fx1, msstats_prot} \hlopt \hlkwd{subset}\hlstd{(Protein} \hlopt \hlstd{washc_prots))}

\hlkwd{r.squaredGLMM.merMod}\hlstd{(fm1)}
\end{alltt}
\end{kframe}
\end{knitrout}


%% latex -----------------------------------------------------------------------
Again, we consider the total variance explained as a measure of the model's
overall quality. Our model explains 89.2\% of the total variance among these
five proteins. The fixed-effect term \texttt{Genotype:BioFraction} explains the
majority of variance ($R^2_m=0.762$). The remaining 13.0\% variance is
attributable to a combination of mixed-effects \texttt{Mixture} and
\texttt{Protein} as well as the residual variance. We assess the overall
\texttt{Mutant-Control} difference between responses of of 'Mutant' and Control
groups as before. The R package \texttt{variancePartition} enables us to
calculate the percent variance explained by a LMM's parameters. To do so, it
expects all terms to be mixed-effects. FIG:variance.


%% R --------------------------------------------------------------------------




