% title: response.tex
% description: Response to eLife Reviewers
% author: twab

% USAGE: 
% to compile this document:
%    R  >>> knitr::knit("response.Rnw")
%    sh >>> pdflatex response.tex


%% R --------------------------------------------------------------------------



%% latex document setup -------------------------------------------------------

\documentclass[12pt]{elife}\usepackage[]{graphicx}\usepackage[]{color}
% maxwidth is the original width if it is less than linewidth
% otherwise use linewidth (to make sure the graphics do not exceed the margin)
\makeatletter
\def\maxwidth{ %
  \ifdim\Gin@nat@width>\linewidth
    \linewidth
  \else
    \Gin@nat@width
  \fi
}
\makeatother

\definecolor{fgcolor}{rgb}{0.345, 0.345, 0.345}
\newcommand{\hlnum}[1]{\textcolor[rgb]{0.686,0.059,0.569}{#1}}%
\newcommand{\hlstr}[1]{\textcolor[rgb]{0.192,0.494,0.8}{#1}}%
\newcommand{\hlcom}[1]{\textcolor[rgb]{0.678,0.584,0.686}{\textit{#1}}}%
\newcommand{\hlopt}[1]{\textcolor[rgb]{0,0,0}{#1}}%
\newcommand{\hlstd}[1]{\textcolor[rgb]{0.345,0.345,0.345}{#1}}%
\newcommand{\hlkwa}[1]{\textcolor[rgb]{0.161,0.373,0.58}{\textbf{#1}}}%
\newcommand{\hlkwb}[1]{\textcolor[rgb]{0.69,0.353,0.396}{#1}}%
\newcommand{\hlkwc}[1]{\textcolor[rgb]{0.333,0.667,0.333}{#1}}%
\newcommand{\hlkwd}[1]{\textcolor[rgb]{0.737,0.353,0.396}{\textbf{#1}}}%
\let\hlipl\hlkwb

\usepackage{framed}
\makeatletter
\newenvironment{kframe}{%
 \def\at@end@of@kframe{}%
 \ifinner\ifhmode%
  \def\at@end@of@kframe{\end{minipage}}%
  \begin{minipage}{\columnwidth}%
 \fi\fi%
 \def\FrameCommand##1{\hskip\@totalleftmargin \hskip-\fboxsep
 \colorbox{shadecolor}{##1}\hskip-\fboxsep
     % There is no \\@totalrightmargin, so:
     \hskip-\linewidth \hskip-\@totalleftmargin \hskip\columnwidth}%
 \MakeFramed {\advance\hsize-\width
   \@totalleftmargin\z@ \linewidth\hsize
   \@setminipage}}%
 {\par\unskip\endMakeFramed%
 \at@end@of@kframe}
\makeatother

\definecolor{shadecolor}{rgb}{.97, .97, .97}
\definecolor{messagecolor}{rgb}{0, 0, 0}
\definecolor{warningcolor}{rgb}{1, 0, 1}
\definecolor{errorcolor}{rgb}{1, 0, 0}
\newenvironment{knitrout}{}{} % an empty environment to be redefined in TeX

\usepackage{alltt}
\usepackage{amsmath}
\usepackage{amssymb}
\usepackage{amsthm}
\usepackage{ragged2e}
\usepackage{caption}
\usepackage{fancyhdr}
\usepackage{graphicx}
\usepackage{titlesec}
\usepackage{blkarray}
\usepackage{csquotes}

\graphicspath{ {./figs/} }


\title{Supplementary Methods\\
\small{Genetic Disruption of WASHC4 Drives Endo-lysosomal Dysfunction and \\
Cognitive-Movement Impairments in Mice and Humans}}

\author[1\authfn{0}]{Jamie Courtland}
\author[1\authfn{0}]{Tyler W. A. Bradshaw}
\author[2]{Greg Waitt}
\author[2,3]{Erik J. Soderblom}
\author[2]{Tricia Ho}
\author[4]{Anna Rajab}
\author[5]{Ricardo Vancini}
\author[2\authfn{1}]{Il Hwan Kim}
\author[6]{Ting Huang}
\author[6]{Olga Vitek}
\author[3]{Scott H. Soderling}

\affil[1]{Department of Neurobiology, Duke University School of Medicine, 
Durham, NC 27710, USA}
\affil[2]{Proteomics and Metabolomics Shared Resource, 
Duke University School of Medicine, Durham, NC 27710, USA}
\affil[3]{Department of Cell Biology, Duke University School of Medicine, 
Durham, NC 27710, USA}
\affil[4]{Burjeel Hospital, VPS Healthcare, Muscat, Oman}
\affil[5]{Department of Pathology, Duke University School of Medicine, 
Durham, NC 27710, USA}
\affil[6]{Khoury College of Computer Sciences, Northeaster University,
Boston, MA 02115, USA}

\contrib[\authfn{0}]{These authors contributed equally to this work.}
\presentadd[\authfn{1}]{Department of Anatomy and Neurobiology, 
University of Tennessee Health Science Center, Memphis, TN 38163, USA}

\corr{jlc123@duke.edu}{JC}
\corr{tyler.w.bradshaw@duke.edu}{TWAB}
\corr{greg.waitt@duke.edu}{GW}
\corr{erik.soderblom@duke.edu}{EJB}
\corr{tricia.ho@duke.edu}{TH}
\corr{drannarajab@gmail.com}{DR}
\corr{ricardo.vancini@duke.edu}{RV}
\corr{ikim9@uthsc.edu}{IK}
\corr{huang.tin@northeastern.edu}{TH}
\corr{o.vitek@northeastern.edu}{OV}
\corr{scott.soderling@duke.edu}{SHS}


\setlength{\abovedisplayskip}{1pt}
\setlength{\belowdisplayskip}{1pt}


%% main -----------------------------------------------------------------------
\IfFileExists{upquote.sty}{\usepackage{upquote}}{}
\begin{document}

\maketitle

\renewcommand{\abstractname}{Summary}
\begin{abstract}

Here we address concerns about the statistical validity of our previous approach
to assess differential protein abundance in the \textbf{WASH-iBioID} and
\textbf{SWIP-TMT} proteomics datasets. Our previous approach depended
upon the R package \texttt{edgeR}. We used \texttt{edgeR} to perform
both protein- and module-level inference---assessing differential
abundance of individual proteins as well as protein groups in
SWIP\textsuperscript{P1019R} mouse brain. \texttt{edgeR} utilizes a
negative binomial (NB) generalized linear model (GLM) framework
originally developed for analysis of RNA-Seq data.  Previously, we
failed to fully consider the validity of \texttt{edgeR's} NB assumption
for proteomics data. Here, we evaluate the goodness-of-fit of the NB GLM
and find evidence of a lack-of-fit.  Thus, we
revise our statistical approach and reanalyze our data, making use of
Huang \textit{et al.}'s recently published R package \texttt{MSstatsTMT}.
\texttt{MSstatsTMT} models the complex sources of variation in TMT mass
spectrometry (MS) experiments using linear mixed-models (LMM).  We
extend the LMM framework used by \texttt{MSstatsTMT} to re-evaluate both
protein- and module-level statistical comparisions in our
\textbf{SWIP-TMT} proteomics dataset.\\

\end{abstract}


\section{edgeR Lack-of-fit for TMT Proteomics}

Our previous approach can be summarized as the 'Sum + IRS' method (Huang2020).
Following protein summarization by summing its features, we performed Internal
Reference Scaling (IRS) normalization (Plubell2017).  We then applied
\texttt{edgeR} to assess differential abundance of individual proteins and
protein-groups.  The use of \texttt{edgeR} for protein-level comparisons was
based on work by Plubell \textit{et al.} who describe IRS normalization and the
use of \texttt{edgeR} for statistical testing in TMT MS experiments
(Plubell2017).  We failed however, to consider the overall adequacy of the NB
GLM model for our TMT proteomics data.\\

Statisitical inference in \texttt{edgeR} is performed for each gene or protein 
using a negative binomial framework. The data are assumed to 
be adequately described by a NB distribution parameterized by a dispersion 
parameter, $\phi$. Practically, the dispersion parameter accounts for the
observed mean-variance relationship in proteomics and transcriptomics data.\\

As signal intensity in protein MS is fundamentally related to the
number of ions generated from an ionized, fragmented protein, we incorrectly
inferred that TMT mass spectrometry data can be modeled as negative binomial
count data. Based on this assumption, we justified the use of \texttt{edgeR}.
Here, we reconsider the overall adequacy of the \texttt{edgeR} NB GLM model for
TMT mass spectrometry data.\\

To evaluate the overall adequacy of the \texttt{edgeR} model, we plot the
residual protein deviance statistics of all proteins against their theoretical,
normal quantiles in a quantile-quantile (QQ) plot (\FIG{gof}).  The QQ plot
addresses the question of how similar the observed data are to the theoretical
NB distribution.  A linear relationship between the observed and theoretical
values is an indicator of goodness-of-fit.  Deviation from this linear trend is
evidence of a lack-of-fit.\\

Following protein summarization and normalization with \texttt{MSstatsTMT}, the
data were fit with a NB GLM using \texttt{edgeR::glmFit}. \FIG{gof} illustrates
the divergence of the observed deviance statistics from the theoretical NB
distribution for \textbf{SWIP-TMT} data fit with \texttt{edgeR's} NB GLM.\\

\begin{figure}[ht!] % figure -- gof
	\begin{fullwidth}
		\begin{center}
		\includegraphics[width=0.9\paperwidth,keepaspectratio]{gof}
		\caption{\textbf{Goodness-of-fit of \texttt{edgeR} (A), and 
		\texttt{MSstatsTMT} (B) statistical approaches.} The overall
		adequacy of the linear models fit to the data were assessed 
		by plotting the residual deviance for all proteins as a 
		quantile-quantile plot (McCarthy \textit{et al.}, (2012)). 
		\textbf{(A)} For analysis with \texttt{edgeR}, The normalized
		protein data from \texttt{MSstatsTMT} were fit with a negative
		binomal generalized linear model of the form: 
		\texttt{Abundance} $\sim$ \texttt{Mixture + Condition}.
		Where \texttt{Mixure} is an additive blocking factor that 
		accounts for variablity between experiments. 
		The NB framework used by \texttt{edgeR} utilizes a dispersion 
			parameter 
		to account for mean-variance relationships in the data.
		The dispersion parameter can take several forms including:
                'common', 'trended', and 'tagwise'. We plot the deviance
		stattistics for the data fit with each of
		the three disperions parameters against their 
		theoretical normal quantiles using the \texttt{edgeR::gof}
		function. \textbf{(B)} For analysis with \texttt{MSstatsTMT},
		the normalized protein data were fit with a linear mixed-effects 
		model (LMM) of the form: 
		\texttt{Abundance} $\sim$ \texttt{0 + Condition + (1|Mixture)}. 
		Where \texttt{Mixture} represents the mixed-effect
		of \texttt{Mixture}. The residual deviance and degrees of 
		freedom were extracted from the fitted models, z-score
		normalized, and plotted as in (A). Proteins with a significantly 
		poor fit are indicated as outliers in blue 
		(Holm-adjusted P-value $<$ 0.05).}
		\label{fig:gof}
	\end{center}
	\end{fullwidth}
\end{figure}


\section{Statistical Inference with MSstatsTMT}

Given our experimental design, \texttt{MSstatsTMT} fits an
appropriate linear-mixed model expressing the major sources of variation in our
experiment.  The quantile-quantile plot in \FIG{gof} indicates that the data are
well described by \texttt{MSstatsTMT's} LMM, which does not depend
upon the negative binomial assumption.  These plots emphasize the overall
lack-of-fit for proteomics data fit with the \texttt{edgeR} model.\\ 

The strength of LMMs lies in their flexibility. In mixed-models, the response
variable is taken to be a function of both fixed- and random-effects. 
If the set of possible levels of a covariate is fixed and reproducible, then the
factor is modeled as a fixed-effect parameter.  In contrast, if the levels of an
observation reflect a sampling of the set of all possible levels, then the
covariate is modeled as a random-effect.  Random or mixed-effects represent
categorical variables that reflect experimental or observational units within
the dataset (Bates2015).  As such, mixed-effect parameters account for the
variation occurring among the lower levels of an upper level unit in the data
(Bates2015).  Using LMMs we can untangle the variance attributable to the
biological effect we are interested in from the experimental and biological
covariates which mask this response.\\

Huang \textit{et al.} created \texttt{MSstatsTMT}, an R package for data
normalization and hypothesis testing in multiplex TMT proteomics experiments. 
They outline a common vocabulary for describing the experimental design of 
a general TMT MS experiment.\\

An experiment consists of \texttt{m = 1} ... \texttt{M}\ concatenations 
of isobarically labeled samples or \texttt{Mixtures}. 
This mixture is then analyzed by the mass spectrometer in a mass 
spectrometry \texttt{Run} to quantify protein abundance. This mixture is often
fractionated into multiple liquid chromotography \texttt{Fractions} to decrease
sample complexity, and thereby increase the depth of proteome coverage. 
Within a mixture, each of the unique TMT channels is dedicated to the 
analysis of \texttt{c = 1} ... \texttt{C}\ individual biological or treatment 
\texttt{Conditions}.  There may then be \texttt{b = 1} or more \texttt{B}\ 
biological replicates or \texttt{Subjects}. Finally, a single TMT mixture may be 
repeatedly analyzed in \texttt{t = 1} ... \texttt{T}\ technical replicate mass 
spectrometry runs.\\

The following equation is a LMM formula which describes protein abundance in an
experiment composed of  \texttt{M} mixtures, \texttt{T} technical replicates of
mixture, \texttt{C} conditions, and \texttt{B} biological subjects.\\

\begin{equation} % equation:full
	\label{eq:full} 	Y_{mcbt} = \mu + Mixture_m + TechRep(Mixture)_{m(t)} + Condition_c + 
	Subject_b + \epsilon_{mcbt}
\end{equation}

\begin{equation} % eq:constraints
  \begin{gathered}
	\label{eq:constraints}  
	\sum_{c=1}^{C} Condition_c = 0 \\
	Subject_{mcb} \stackrel{iid}{\sim} N(0,\sigma^2_S) \\
	Mixture_m \stackrel{iid}{\sim} N(0,\sigma^2_M) \\
	TechRep(Mixture)_{t(m)} \stackrel{iid}{\sim} N(0,\sigma^2_T) \\
	\epsilon{mtcb} \stackrel{iid}{\sim} N(0,\sigma^2) \\
  \end{gathered}
\end{equation}

The model's constraints (\ref{eq:constraints}) distinguish fixed- and
mixed-effect components of variation in the response. \texttt{Mixture} is a
mixed-effect and represents the variation between TMT mixtures. By definition
mixed-effects are assumed to be independent and normally distributed (iid).
\texttt{TechRep(Mixture)} represents random variation between replicates of a
single MS \texttt{Run}.  The term \texttt{Subject} cooresponds to each unique
biological replicate and represents biological variation among the levels of the
fixed-effect term \texttt{Condition}. The term $\epsilon_{mtcb}$, is a
mixed-effect representing both biological and technical variation, quantifying
any remaining error.\\

If a component of the model is not estimable, then it is removed. 
For example, if there is no technical replication of mixture \texttt{(T=0)}, 
then the model is reduced to: \\

\begin{equation}
	\label{eq:reduced} % equation -- reduced
	Y_{mcbt} = \mu + Mixture_m + Condition_c + Subject_b + \epsilon_{mcb}
\end{equation}

\texttt{MSstatsTMT} performs protein-wise comparisons between pairs of 
\texttt{Conditions} by comparing the estimates obtained from the fit LMM. 
We are interested in testing the null hypothesis:\\

\begin{equation}
	\label{eq:null} % equation -- null
	H0 : l^T * \beta = 0. 
\end{equation}

Where $l^T$ is a vector of $\sum$=1 specifying the positive and negative
coefficients of a contrast. $\beta$ is the model-based estimates of the 
levels of \texttt{Condition}. A test statistic for such a two-way contrasts is 
given by Kutzenova \textit{et al.,}(Kutzenova2017):\\

\begin{equation} 
	\label{eq:tstatistic} % equation -- tstatistic
	t = \frac{l^T \hat{\beta}}{\sqrt{l \sigma^2 \hat{V} l^T}}
\end{equation}

We obtain the models estimates $\hat{\beta}$, error $\sigma^2$, and
variance-covariance matrix $\hat{V}$ from the model fitted by restricted maximum
likelihood. Given a contrast, $l^T$, the numerator of equation 
(\ref{eq:tstatistic}) is the fold change of a comparison. 
In the denominator, the product of $\sigma^2$ and $\hat{V}$ is the scaled 
variance-covariance matrix describing error estimates of the model's 
fixed- and mixed-effect parameters. Together the denominator 
represents the standard error of the comparison.\\

The degrees of freedom for the contrast are derived using the Satterthwaite
moment of approximation method (Satterthwaite1946, Kutzenova2017).  Finally,
a p-value is calculated given the t-statistic and degrees of freedom. 
P-values for the protein-wise tests are adjusted using the Benjamini-Hochberg 
FDR method (Benjamini1995,Huang2020).\\

\section{SWIP-TMT Experimental Design}

\begin{figure}[h] %% figure x -- design 
  \begin{fullwidth}
  \begin{center}
	  \includegraphics[width=0.9\paperwidth,keepaspectratio]{design}
	  \caption{\textbf{Experimental Design.} We performed three 16-plex TMT
	  experiments. Each TMT mixture is a concatenation of 16 labeled
	  samples. In each experiment we analyzed seven subcellular
	  \texttt{BioFractions} prepared from the brain of a single Control
	  and 'Mutant' mouse. In all, we analyzed three \texttt{Subjects} from 
	  each {Condition}. Each \texttt{Mixture} includes two \texttt{Channels}
	  dedicated to the analysis of a common quality control (QC) sample for
	  normalization between MS runs.}
	  \label{fig:design}
  \end{center}
  \end{fullwidth}
\end{figure}

In our experiment, the fixed-effect term \texttt{Condition} in equation
\ref{eq:reduced} represents the fourteen combinations of \texttt{Genotype} and
\texttt{BioFraction} obtained from subcellular fractionation of Control and
SWIP\textsuperscript{P1019R} mouse brain. We refer to these as
\texttt{BioFractions} to distinguish them from an MS \texttt{Fraction}. 

Each 16-plex TMT mixture contains seven repeated measurements made from each
biological subject (\FIG{design}).  To account for this repeated measures
design, we should include the random-effect term \texttt{Subject}.  However, in
our experimental design, \texttt{Mixture} is confounded with \texttt{Subject}.
In each \texttt{Mixture} we analyzed all seven \texttt{BioFractions} from a
single \texttt{Subject} (a Control or Mutant mouse). Thus we can choose to
account for the effect of \texttt{Mixture} or \texttt{Subject}, but not both. We
choose to account for variability of \texttt{Mixture} based on the assumption
that the variance associated with this experimental batch effect is greater than
the intra-Subject variance inherent in the repeated measures of each subject.\\

We omit the un-estimable terms \texttt{TechRep(Mixture)} and \texttt{Subject}
from equation (\ref{eq:full}) and the reduced model is then:

\begin{equation}
	\label{eq:fx0}
	Y_{mcbt} = \mu + Mixture_m + Condition_c + \epsilon_{mcb}
\end{equation}


\begin{figure}[ht!] %% figure x -- contrasts
  \begin{fullwidth}
  \begin{center}
	  \includegraphics[width=0.9\paperwidth,keepaspectratio]{contrasts}
	  \caption{\textbf{Statistical Comparisons.} We assessed two types of
	  contrasts. Each row of the matrix specifies a contrast between
	  positive and negative coefficients in the mixed-effects model fit to
	  each protein. Contrasts1-7 are intra-BioFraction contrasts that
	  specify the pairwise comparisons of Control and Mutant groups for a
	  single fraction. In Contrast 8 we compare Mutant-Control and asses
	  the overall difference of Control and Mutant conditions.  Each
	  contrast is a vector of sum 1.}
	  \label{fig:contrasts}
  \end{center}
  \end{fullwidth}
\end{figure}


\section{Protein-level comparisions}

Following data preprocessing, summarization, and normalization, statistical
inference by \texttt{MSstatsTMT} can be summarized in two steps:\\

\begin{itemize}
	\item Fit each protein with an appropriate LMM based on the design.
	\item Given the fitted model, assess a contrast of interest.
\end{itemize}

Using \texttt{MSstatsTMT} we assesssed two types of protein comparisons:\\

\begin{itemize}
	\item \texttt{intra-BioFraction} comparisons (7)
	\item \texttt{Mutant-Control} contrast  (1)
\end{itemize}

\texttt{Intra-BioFraction} comparisons are the seven pairwise comparisons of 
Control and Mutant protein abundance for each subcellular
\texttt{BioFraction}. We also assessed differential abundance for the 
overall \texttt{Mutant-Control} comparison. Each of these contrasts is 
represented by a vector, $l^T_{1-8}$, which specifies a comparison between 
coefficients in the LMM (\ref{eq:fx0}).
\FIG{contrasts} illustrates a matrix defining all eight unique comparisons.\\


\texttt{MSstatsTMT} attempts to automatically parse the experimental design and
fit the appropriate LMM to each protein in the dataset. In order to understand 
and extend the function of \texttt{MSstatsTMT}, 
we extracted \texttt{MSstatsTMT's} core model-fitting and statistical 
testing steps and illustrate them here.\\

At the core of the model fitting-step is the
R package \texttt{lme4} which implements mixed-effects models with its function
\texttt{lmer}(Bates2015). The package \texttt{lmerTest} extends \texttt{lme4's}
functionality and enables the computation of 
Sattertwaite degrees of freedom (Kutzenova2017).\\

As an example, we illustrate the analysis of WASHC4. First, we fit 
the model (\ref{eq:fx0}) to the normalized protein data from
\texttt{MSstatsTMT}.


%% R ---------------------------------------------------------------------------
\begin{knitrout}
\definecolor{shadecolor}{rgb}{0.969, 0.969, 0.969}\color{fgcolor}\begin{kframe}
\begin{alltt}
\hlcom{# load dependencies}
\hlkwd{library}\hlstd{(dplyr)}
\hlkwd{library}\hlstd{(lmerTest)}

\hlcom{#library(SwipProteomics)}
\hlkwd{data}\hlstd{(swip)}
\hlkwd{data}\hlstd{(msstats_prot)}

\hlcom{# formula to be fit to WASHC4, aka SWIP:}
\hlstd{fx0} \hlkwb{<-} \hlstr{'Abundance ~ 0 + Genotype:BioFraction + (1|Mixture)'}

\hlcom{# fit the LMM}
\hlstd{fm0} \hlkwb{<-} \hlkwd{lmer}\hlstd{(fx0, msstats_prot} \hlopt \hlkwd{subset}\hlstd{(Protein} \hlopt{==} \hlstd{swip))}

\hlcom{# examine the model's summary}
\hlkwd{summary}\hlstd{(fm0,} \hlkwc{ddf} \hlstd{=} \hlstr{"Satterthwaite"}\hlstd{)} \hlopt \hlstd{knitr}\hlopt{::}\hlkwd{kable}\hlstd{()}
\end{alltt}


{\ttfamily\noindent\bfseries\color{errorcolor}{\#\# Error in as.data.frame.default(x): cannot coerce class 'c("{}summary.lmerModLmerTest"{}, "{}summary.merMod"{})' to a data.frame}}\end{kframe}
\end{knitrout}


%% latex -----------------------------------------------------------------------
The model's estimates ($\beta$) represent our best estimate of the mean protein
abundance in the fourteen conditions of \texttt{Genotype:BioFraction}. 
To illustrate an \texttt{intra-BioFraction} comparison, we 
define a contrast comparing the \texttt{Mutant:F7} and \texttt{Control:F7}
conditions.


%% R --------------------------------------------------------------------------
\begin{knitrout}
\definecolor{shadecolor}{rgb}{0.969, 0.969, 0.969}\color{fgcolor}\begin{kframe}
\begin{alltt}
\hlcom{# create a contrast}
\hlstd{coeff} \hlkwb{<-} \hlstd{lme4}\hlopt{::}\hlkwd{fixef}\hlstd{(fm0)}
\hlstd{contrast7} \hlkwb{<-} \hlkwd{setNames}\hlstd{(}\hlkwd{rep}\hlstd{(}\hlnum{0}\hlstd{,}\hlkwd{length}\hlstd{(coeff)),} \hlkwc{nm} \hlstd{=} \hlkwd{names}\hlstd{(coeff))}
\hlstd{contrast7[}\hlstr{"GenotypeMutant:BioFractionF7"}\hlstd{]} \hlkwb{<-} \hlopt{+}\hlnum{1} \hlcom{# positive coeff}
\hlstd{contrast7[}\hlstr{"GenotypeControl:BioFractionF7"}\hlstd{]} \hlkwb{<-} \hlopt{-}\hlnum{1} \hlcom{# negative coeff}

\hlcom{# evaluate contrast}
\hlkwd{lmerTestContrast}\hlstd{(fm0, contrast7)} \hlopt \hlstd{knitr}\hlopt{::}\hlkwd{kable}\hlstd{()}
\end{alltt}
\end{kframe}
\begin{tabular}{l|r|r|r|r|r|r|l}
\hline
Contrast & log2FC & percentControl & SE & Tstatistic & Pvalue & DF & isSingular\\
\hline
GenotypeMutant:BioFractionF7-GenotypeControl:BioFractionF7 & -1.689393 & 0.3100573 & 0.1514779 & -11.15273 & 0 & 26 & FALSE\\
\hline
\end{tabular}


\end{knitrout}


%% latex ----------------------------------------------------------------------
The function \texttt{lmerTestContrast} performs the statstical comparison given
a fitted model and a convrast vector defining a comparison between the models
coefficients. While the work done by this function 
is the same as the work done internally by \texttt{MSstatsTMT's}
\texttt{groupComparisonsTMT} function, \texttt{lmerTestContrast} is more
flexible. Provided the correct contrast, we easily assess the overall
\texttt{Mutant-Control} comparison.\\


%% R --------------------------------------------------------------------------
\begin{knitrout}
\definecolor{shadecolor}{rgb}{0.969, 0.969, 0.969}\color{fgcolor}\begin{kframe}
\begin{alltt}

\hlcom{# use convenience function to contruct a contrast}
contrast8 <- \hlkwd{getContrast}(fm0, \hlstr{"Mutant"},\hlstr{"Control"}))

\hlcom{# assess the comparison}
\hlkwd{lmerTestContrast}(fm0, contrast8) %>% knitr::\hlkwd{kable}()
\end{alltt}


{\ttfamily\noindent\bfseries\color{errorcolor}{\#\# Error: <text>:3:50: unexpected ')'\\\#\# 2: \# use convenience function to contruct a contrast\\\#\# 3: contrast8 <- getContrast(fm0, "{}Mutant"{},"{}Control"{}))\\\#\#\ \ \ \ \ \ \ \ \ \ \ \ \ \ \ \ \ \ \ \ \ \ \ \ \ \ \ \ \ \ \ \ \ \ \ \ \ \ \ \ \ \ \ \ \ \ \ \ \ \ \ \  \textasciicircum{}}}\end{kframe}
\end{knitrout}


%% latex -----------------------------------------------------------------------
\section{LMM Goodness-of-fit}

It is useful to consider the goodness-of-fit of our LMM. A straight forward
measure of a LMM's quality is the Nakagawa coefficient of 
determination (Nakagawa2013,Nakagawa2017). Nakagawa's conditional $R^2$ is 
interpreted as the total variance explained by a LMM ($R^2_{total}$).
The marginal $R^2$ is interpreted as the variance explained by the LMM's 
fixed-effects ($R^2_{fixed}$). We implment Nakagawa's coeffficient of 
determination using the \texttt{r.squaredGLMM} function taken from the 
\texttt{MuMin} package (WangMerkel2018).\\


%% R --------------------------------------------------------------------------
\begin{knitrout}
\definecolor{shadecolor}{rgb}{0.969, 0.969, 0.969}\color{fgcolor}\begin{kframe}
\begin{alltt}
\hlcom{# assess gof with Nakagawa coefficient of determination}
\hlkwd{r.squaredGLMM.merMod}\hlstd{(fm0)} \hlopt \hlstd{knitr}\hlopt{::}\hlkwd{kable}\hlstd{()}
\end{alltt}
\end{kframe}
\begin{tabular}{r|r}
\hline
R2m & R2c\\
\hline
0.9353344 & 0.949433\\
\hline
\end{tabular}


\end{knitrout}


%% latex -----------------------------------------------------------------------
The total variation explained, $R^2_{c}$, for the LMM fit to WASHC4 is 
\texttt{0.949}. The variance explained by fixed-effects, represents a large
fraction of this total ($R^2{m}$=0.935). Only about 1.5\% of the remaing
variance is attributable to residuals and the mixed-effect \texttt{Mixture}.\\


\section{Module-level analysis}

We wish to extend the LMM framework developed by \texttt{MSstatsTMT} to perform 
inference at the level of protein groups or modules.
That is, for module-level comparisons, we are interested in the overall affect 
of Genotype on a group of proteins. Where modules are groups of covarying 
proteins which represent biological niches defined by proteins that 
localized together in subcellular space.\\

Here we hypothesize that the proteins within a module, which are a subset of the
overall proteome,  are a part of a common group, a module, with a common mean
effect. Proteins within a module are correlated observations which we model as a
mixed-effect as we are primarily interested in making inference about the
overall distribution of the responses for a module rather than among its
sublevels. The following LMM includes the additional mixed effect term
\texttt{Protein}, capturing variance among a module's constintuent proteins.\\

\begin{equation} 
  \begin{gathered}\label{eq:fx1} % equation -- fx1
	Y_{mcbt} = \mu + Mixture_m + Condition_c + Protein_p + \epsilon_{mcb}\\
	Protein_p \stackrel{iid}{\sim} N(0,\sigma^2_P) \\
  \end{gathered}
\end{equation}

The term \texttt{Protein} quantifies the variance $\sigma_P$ attributable to all
proteins in a module.  As a means of example, we demonstrate an ideal module, by
fitting LMM (\ref{eq:fx1}) to the five WASH complex proteins.  As before, we
calculate the coefficient of determination for LMM's with the
\texttt{r.squaredGLMM} function (WangMerkel2018).\\


%% R --------------------------------------------------------------------------
\begin{knitrout}
\definecolor{shadecolor}{rgb}{0.969, 0.969, 0.969}\color{fgcolor}\begin{kframe}
\begin{alltt}
\hlcom{# the module-level formula to be fit:}
\hlstd{fx1} \hlkwb{<-} \hlstr{'Abundance ~ 0 + Condition + (1|Mixture) + (1|Protein)'}

\hlcom{# load WASH Complex proteins}
\hlkwd{data}\hlstd{(washc_prots)}

\hlstd{fm1} \hlkwb{<-} \hlkwd{lmer}\hlstd{(fx1, msstats_prot} \hlopt \hlkwd{subset}\hlstd{(Protein} \hlopt \hlstd{washc_prots))}

\hlkwd{r.squaredGLMM.merMod}\hlstd{(fm1)} \hlopt \hlstd{knitr}\hlopt{::}\hlkwd{kable}\hlstd{()}
\end{alltt}
\end{kframe}
\begin{tabular}{r|r}
\hline
R2m & R2c\\
\hline
0.7620866 & 0.8928053\\
\hline
\end{tabular}


\end{knitrout}


%% latex -----------------------------------------------------------------------
Again, we consider the total variance explained as a measure of the model's 
overall quality. Our model explains 89.2\% of the total variance among these
five proteins. The fixed-effect term \texttt{Genotype:BioFraction} explains 
the majority of variance ($R^2_m=0.762$). The remaining 13.0\% variance 
is attributable to a combination of mixed-effects \texttt{Mixture} and 
\texttt{Protein} as well as the residual variance.\\

We assess the overall \texttt{Mutant-Control} difference between responses of 
of 'Mutant' and Control groups as before.


%% latex -----------------------------------------------------------------------
%The test statstistic (\ref{eq:tstatistic}) accounts for variance of the fixed-
%and mixed-effects. It is useful to consider the variance ($\sigma^2$) of the 
%individual mixed-effect terms. These can be assessed with \texttt{lme4's} 


%% latex -----------------------------------------------------------------------

\begin{figure}[hb!] %% figure 4 -- Variance Partition
  \begin{fullwidth}
  \begin{center}
	  \includegraphics[width=0.9\paperwidth,keepaspectratio]{variance}
	  \caption{\textbf{Analysis of Variance Components.} 
	  The proportion of variance explained by Genotype, BioFraction,
	  Mixture, and remaining residual error (subplot error) for all
	  proteins. Note while the contribution of Mixture seems negligiable,
	  its average for all proteins is approximately twice the average
	  percent variance explained by Genotype. BioFraction explains the
	  majority of the variance for all proteins. Analysis done with
	  \texttt{variancePartition::calcVarPart}.}
	  \label{fig:variance}
  \end{center}
  \end{fullwidth}
\end{figure}


The R package \texttt{variancePartition} enables us to calculate the percent
variance explained by a LMM's parameters. To do so, it expects all terms to be
mixed-effects. \FIG{variance}.


%% R --------------------------------------------------------------------------
\begin{knitrout}
\definecolor{shadecolor}{rgb}{0.969, 0.969, 0.969}\color{fgcolor}\begin{kframe}
\begin{alltt}
\hlcom{# load variancePartition}
\hlkwd{library}\hlstd{(variancePartition)}

\hlcom{# calculate partitioned variance}
\hlstd{form} \hlkwb{<-} \hlstr{"Abundance ~ (1|Genotype) + (1|BioFraction) + (1|Mixture) + (1|Protein)"}
\hlstd{fit} \hlkwb{<-} \hlkwd{lmer}\hlstd{(form,} \hlkwc{data} \hlstd{= msstats_prot} \hlopt \hlkwd{filter}\hlstd{(Protein} \hlopt \hlstd{washc_prots))}

\hlkwd{calcVarPart}\hlstd{(fit)}
\end{alltt}
\begin{verbatim}
## BioFraction    Genotype     Mixture     Protein   Residuals 
## 0.032960635 0.822069159 0.002843637 0.074146798 0.067979772
\end{verbatim}
\end{kframe}
\end{knitrout}


%% tex -------------------------------------------------------------------------
We can see that the majority of the variance explained by the LMM fit to the
WASH complex is attributable to \texttt{Genotype}. The mixed-effect terms
\texttt{Protein} and \texttt{Mixture} account for a small fraction of the 
overall variance explained by the model.\\

As our overall goal is to identify groups or modules of proteins that strongly
covary together, our clustering approach should maximize the variance explained
by a module's fixed-effect parameters (Genotype + BioFraction) while minimizing 
the variance among its individual proteins. 
An ideal module is a perfect summary of its protein constituents, 
$PVE_{Protein}=0$. We use this idea of a module's quality to supervise our 
clustering approach.\\

\begin{equation}
	Quality_{Module}=\frac{PVE_{Genotype} + PVE_{BioFraction}}{PVE_{Protein}}
\end{equation}


\section{Network Construction}

Using our \textbf{SWIP-TMT} dataset, we aim to identify modules or groups of
proteins that covary together across subcellular space. Prior to building the
co-variation network, other sources of variation should be removed. Although
\texttt{MSstatsTMT} handles the batch effect inherent in experiments with
multiple TMT mixtures, it is necessary to remove this effect prior to building
the network. We removed the effect of \texttt{Mixture} using
\texttt{limma::RemoveBatchEffect}.  These adjusted data are used for network
construction and plotting but not statistical modeling.\\

Prior to network construction, we removed protein models with poor fit 
($R^2_{total}<0.7$; n=791 proteins). Removing this noisey proteins facilitation
module identification and improves overall module quality.\\

The final network was constructed using data from both Control and Mutant 
samples after adjusting for batch (Mixture). The final dataset included 
42 samples and 6,119 proteins. The protein covariation network was build by
calculating the Pearson correlation for all pairwise comparisons of proteins.\\

We performed network enhancement to remove biological noise from the network.
This step is essential for module detection. Network enhancement reweights the
network's edges and has the overall effect of making the network sparse.
Conceptually this step is related to the soft-thresholding approach taken by 
WGCNA or WPCNA analysis workflows (REFS), but has the befinit of not assuming
that the network has an overall scale free topology. 
Without reweighting or enhancing the network, most extant clustering 
algorithms fail to detect communities in the dataset. 
Network enhancment has the effect of making the
network sparse and facilitates the identification of network structure (FIG).\\

\begin{figure}[h] %% figure [x] -- impute.pdf
  \begin{fullwidth}
  \begin{center}
	  \includegraphics[width=0.9\paperwidth,keepaspectratio]{impute}
	  \caption{\textbf{Missing value imputation and PSM outlier removal.}
	  \textbf{A} \textbf{B} \textbf{C} \textbf{D} }
	  \label{fig:impute}
  \end{center}
  \end{fullwidth}
\end{figure}


(\FIG{impute})

\begin{figure}[h] %% figure [x] -- normalization.pdf
  \begin{fullwidth}
  \begin{center}
	  \includegraphics[width=0.9\paperwidth,keepaspectratio]{normalization}
	  \caption{\textbf{Data Normalization and PCA.} \textbf{A} \textbf{B} }
	  \label{fig:normalization}
  \end{center}
  \end{fullwidth}
\end{figure}


(\FIG{normalization})


\begin{figure}[h] %% figure [x] -- washc4.pdf
  \begin{fullwidth}
  \begin{center}
	  \includegraphics[width=0.9\paperwidth,keepaspectratio]{washc4}
	  \caption{\textbf{Data Normalization and PCA.} \textbf{A} \textbf{B} }
	  \label{fig:washc4}
  \end{center}
  \end{fullwidth}
\end{figure}


(\FIG{washc4})

\end{document}
