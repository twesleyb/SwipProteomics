% DISCUSSION

Taken together, the data presented here support a mechanistic model whereby
SWIPP1019R causes a loss of WASH complex function, resulting in endo-lysosomal
disruption and accumulation of neurodegenerative markers, such as upregulation
of unfolded protein response modulators and lysosomal enzymes, as well as
build-up of lipofuscin and cleaved caspase-3 over time. To our knowledge, this
study provides the first mechanistic evidence of WASH complex impairment having
direct and indirect organellar effects that lead to cognitive deficits and
progressive motor impairments (Figure 8).

Using in vivo proximity-based proteomics in wild-type mouse brain, we identify
that the WASH complex interacts with the CCC (COMMD9 and CCDC93) and Retriever
(VPS35L) cargo selective complexes (Bartuzi et al., 2016; Singla et al., 2019).
Interestingly, we did not find significant enrichment of the Retromer sorting
complex, a well-known WASH interactor, suggesting that it may play a minor role
in neuronal WASH-mediated cargo sorting (Figure 1). These data are supported by
our TMT proteomics and covariation network analyses of SWIPP1019R mutant brain,
which clustered the WASH, CCC, and Retriever complexes together in M19, but not
the Retromer complex, which was found in endosomal module M14 (Figure 2 and
Figure 2-figure supplement 3A). Systems-level protein covariation analyses also
revealed that disruption of these WASH-CCC-Retriever interactions may have
multiple downstream effects on the endosomal machinery, since endosomal modules
displayed significant changes in SWIPP1019R brain (including both M19, Figure 2,
as well as M14, Figure 2-figure supplement 3A), with corresponding decreases in
the abundance of endosomal proteins including Retromer subunits (VPS29 and
VPS35), associated sorting nexins (e.g. SNX17 and SNX27), known WASH interactors
(e.g. RAB21 and FKBP15), and cargos (e.g. LRP1 and ITGA3) (Figure 2-figure
supplements 2 and 3) (Del Olmo et al., 2019; Farfán et al., 2013; Fedoseienko et
al., 2018; Halff et al., 2019; Harbour et al., 2012; McNally et al., 2017; Pan
et al., 2010; Ye et al., 2020; Zimprich et al., 2011). While previous studies
have indicated that Retromer and CCC influence endosomal localization of WASH
(Harbour et al., 2012; Phillips-Krawczak et al., 2015; Singla et al., 2019), our
findings of altered endosomal networks containing decreased Retromer, Retriever,
and CCC protein levels in SWIPP1019R mutant brain point to a possible feedback
mechanism wherein WASH impacts the protein abundance and/or stability of these
interactors. Future studies defining the hierarchical interplay between the
WASH, Retromer, Retriever, and CCC complexes in neurons could provide clarity on
how these mechanisms are organized.

In addition to highlighting the neuronal roles of WASH in CCC- and
Retriever-mediated endosomal sorting, our proteomics approach also identified
protein modules with increased abundance in SWIPP1019R mutant brain. The
proteins in these modules fell into two interesting categories: lysosomal
enzymes and proteins involved in the endoplasmic reticulum (ER) stress response.
Of note, some of the lysosomal enzymes with elevated levels in MUT brain (GRN,
M2; IDS, M2; and GNS, M213; Figure 3) are also implicated in lysosomal storage
disorders, where they generally have decreased, rather than increased, function
or expression (Hopwood et al., 1993; Mok et al., 2003; Schröder et al., 1994;
Ward et al., 2017). We speculate that loss of WASH function in our mutant mouse
model may lead to increased accumulation of cargo and associated machinery at
early endosomes (as seen in Figure 4, enlarged EEA1+ puncta), eventually
overburdening early endosomal vesicles and triggering transition to late
endosomes for subsequent fusion with degradative lysosomes (Figure 8). This
would effectively increase delivery of endosomal substrates to the lysosome
compared to baseline, resulting in enlarged, overloaded lysosomal structures,
and elevated demand for degradative enzymes. For example, since mutant neurons
display increased lysosomal module protein abundance (Figure 3), and larger
lysosomal structures (Figures 4 and 5), they may require higher quantities of
progranulin (GRN, M2; Figure 3) for sufficient lysosomal acidification (Tanaka
et al., 2017).

Our findings that SWIPP1019R results in reduced WASH complex stability and
function, which may ultimately drive lysosomal dysfunction, are supported by
studies in non-mammalian cells. For example, expression of a dominant-negative
form of WASH1 in amoebae impairs recycling of lysosomal V-ATPases (Carnell et
al., 2011) and loss of WASH in Drosophila plasmocytes affects lysosomal
acidification (Gomez et al., 2012; Nagel et al., 2017; Zech et al., 2011).
Moreover, mouse embryonic fibroblasts lacking WASH1 display abnormal lysosomal
morphologies, akin to the structures we observed in cultured SWIPP1019R MUT
neurons (Gomez et al., 2012). 

In addition to lysosomal dysfunction, endoplasmic reticulum (ER) stress is
commonly observed in neurodegenerative states, where accumulation of misfolded
proteins disrupts cellular proteostasis (Cai et al., 2016; Hetz and Saxena,
2017; Montibeller and de Belleroche, 2018). This cellular strain triggers the
adaptive unfolded protein response (UPR), which attempts to restore cellular
homeostasis by increasing the cell’s capacity to retain misfolded proteins
within the ER, remedy misfolded substrates, and trigger degradation of
persistently misfolded species. Involved in this process are ER chaperones that
we identified as increased in SWIPP1019R mutant brain including BiP (HSPA5),
calreticulin (CALR), calnexin (CANX), and the protein disulfide isomerase family
members (PDIA1, PDIA4, PDIA6) (M83; Figure 2-supplement 3B) (Garcia-Huerta et
al., 2016). Many of these proteins were identified in the ER protein module
found to be significantly altered in MUT mouse brain (M83), supporting a
network-level change in the ER stress response (Figure 2-supplement 3B). One
notable exception to this trend was endoplasmin (HSP90B1, M136), which exhibited
significantly decreased abundance in SWIPP1019R mutant brain (Table S2). This is
surprising given that endoplasmin has been shown to coordinate with BiP in
protein folding (Sun et al., 2019), however it may highlight a possible
compensatory mechanism. Additionally, prolonged UPR can stimulate autophagic
pathways in neurons, where misfolded substrates are delivered to the lysosome
for degradation (Cai et al., 2016). These data highlight a relationship between
ER and endo-lysosomal disturbances as an exciting avenue for future research. 

Strikingly, we observed modules enriched for resident proteins corresponding to
all 10 of the major subcellular compartments mapped by Geladaki et al. (2019;
nucleus, mitochondria, golgi, ER, peroxisome, proteasome, plasma membrane,
lysosome, cytoplasm, and ribosome; Supplementary File 1). The greatest
dysregulations we observed were in lysosomal, endosomal, ER, and synaptic
modules, supporting the hypothesis that SWIPP1019R primarily results in
disrupted endo-lysosomal trafficking. While analysis of these dysregulated
modules informs the pathobiology of SWIPP1019R, our spatial proteomics approach
also identified numerous biologically cohesive modules, which remained unaltered
(Supplementary File 1). Given that many of these modules contained proteins of
unknown function, we anticipate that future analyses of these modules and their
protein constituents have great potential to inform our understanding of protein
networks and their influence on neuronal cell biology. 

	It has become clear that preservation of the endo-lysosomal system is
critical to neuronal function, as mutations in mediators of this process are
implicated in neurological diseases such as Parkinson’s disease, Huntington’s
disease, Alzheimer’s disease, Frontotemporal Dementia, Neuronal Ceroid
Lipofuscinoses (NCLs), and Hereditary Spastic Paraplegia (Baker et al., 2006;
Connor-Robson et al., 2019; Edvardson et al., 2012; Follett et al., 2019; Harold
et al., 2009; Mukherjee et al., 2019; Pal et al., 2006; Quadri et al., 2013;
Seshadri et al., 2010; Tachibana et al., 2019; Valdmanis et al., 2007). These
genetic links to predominantly neurodegenerative conditions have supported the
proposition that loss of endo-lysosomal integrity can have compounding effects
over time and contribute to progressive disease pathologies. In particular, NCLs
are lysosomal storage disorders primarily found in children, with heterogenous
presentations and multigenic causations (Mukherjee et al., 2019). The majority
of genes implicated in NCLs affect lysosomal enzymatic function or transport of
proteins to the lysosome (Mukherjee et al., 2019; Poët et al., 2006;
Ramirez-Montealegre and Pearce, 2005; Yoshikawa et al., 2002). Most patients
present with marked neurological impairments, such as learning disabilities,
motor abnormalities, vision loss, and seizures, and have the unifying feature of
lysosomal lipofuscin accumulation upon pathological examination (Mukherjee et
al., 2019). While the human SWIPP1019R mutation has not been classified as an
NCL (Ropers et al., 2011), findings from our mutant mouse model suggest that
loss of WASH complex function leads to phenotypes bearing strong resemblance to
NCLs, including lipofuscin accumulation (Figures 4-7). As a result, our mouse
model could provide the opportunity to study these pathologies at a mechanistic
level, while also enabling preclinical development of treatments for their human
counterparts. 

Currently there is an urgent need for greater mechanistic investigations of
neurodegenerative disorders, particularly in the domain of endo-lysosomal
trafficking. Despite the continual increase in identification of human
disease-associated genes, our molecular understanding of how their protein
equivalents function and contribute to pathogenesis remains limited. Here we
employ a systems-level analysis of proteomic datasets to uncover biological
perturbations linked to SWIPP1019R. We demonstrate the power of combining in
vivo proteomics and systems network analyses with in vitro and in vivo
functional studies to uncover relationships between genetic mutations and
molecular disease pathologies. Applying this platform to study organellar
dysfunction in other neurodegenerative and neurodevelopmental disorders may
facilitate the identification of convergent disease pathways driving brain
disorders. 
