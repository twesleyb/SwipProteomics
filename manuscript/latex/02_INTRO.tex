% INTRODUCTION

Neurons maintain precise control of their subcellular proteome using a 
sophisticated network of vesicular trafficking pathways that shuttle cargo 
throughout their elaborate processes. Endosomes function as a central hub in 
this vesicular relay system by coordinating protein sorting between multiple 
cellular compartments, including surface receptor endocytosis and recycling, 
as well as degradative shunting to the lysosome. 
How endosomal trafficking is modulated in neurons remains a vital area of 
research due to the unique degree of spatial segregation between organelles 
in neurons, and its strong implication in neurodevelopmental and 
neurodegenerative diseases.  

In non-neuronal cells, an evolutionarily conserved complex, the Wiskott-Aldrich
Syndrome protein and SCAR Homology (WASH) complex, coordinates endosomal
trafficking (derivery 2010, linardopoulou 2007). WASH is
composed of five core protein components: WASHC1 (aka WASH1), WASHC2 (aka
FAM21), WASHC3 (aka CCDC53), WASHC4 (aka SWIP), and WASHC5 (aka Strumpellin)
(encoded by genes Washc1-Washc5, respectively), which are broadly expressed in
multiple organ systems (Alekhina et al., 2017; Kustermann et al., 2018; McNally
et al., 2017; Simonetti and Cullen, 2019; Thul et al., 2017).  The WASH complex
plays a central role in non-neuronal endosomal trafficking by activating
Arp2/3-dependent actin branching at the outer surface of endosomes to influence
cargo sorting and vesicular scission (Gomez and Billadeau, 2009; Lee et al.,
2016; Phillips-Krawczak et al., 2015; Piotrowski et al., 2013; Simonetti and
Cullen, 2019). WASH also interacts with at least three main cargo adaptor
complexes — the Retromer, Retriever, and COMMD/CCDC22/CCDC93 (CCC) complexes —
all of which associate with distinct sorting nexins to select specific cargo and
enable their trafficking to other cellular locations (Binda et al., 2019; Farfán
et al., 2013; McNally et al., 2017; Phillips-Krawczak et al., 2015; Seaman and
Freeman, 2014; Singla et al., 2019). Loss of the WASH complex in non-neuronal
cells has detrimental effects on endosomal structure and function, as its loss
results in aberrant endosomal tubule elongation and cargo mislocalization
(Bartuzi et al., 2016; Derivery et al., 2009; Gomez et al., 2012; Gomez and
Billadeau, 2009; Phillips-Krawczak et al., 2015; Piotrowski et al., 2013).
However, whether the WASH complex performs an endosomal trafficking role in
neurons remains an open question, as no studies have addressed neuronal WASH
function to date. 

Consistent with the association between the endosomal trafficking system and
pathology, dominant missense mutations in WASHC5 (protein: Strumpellin) are
associated with hereditary spastic paraplegia (SPG8) (De Bot et al., 2013;
Valdmanis et al., 2007), and autosomal recessive point mutations in WASHC4
(protein: SWIP) and WASHC5 are associated with syndromic and non-syndromic
intellectual disabilities (Assoum et al., 2020; Elliott et al., 2013; Ropers et
al., 2011). In particular, an autosomal recessive mutation in WASHC4 (c.3056C>G;
p.Pro1019Arg) was identified in a cohort of children with non-syndromic
intellectual disability (Ropers et al., 2011). Cell lines derived from these
patients exhibited decreased abundance of WASH proteins, leading the authors to
hypothesize that the observed cognitive deficits in SWIPP1019R patients resulted
from disruption of neuronal WASH signaling (Ropers et al., 2011). However,
whether this mutation leads to perturbations in neuronal endosomal integrity, or
how this might result in cellular changes associated with disease, are unknown.
