@article{Alekhina2017,
annote = {Good review of what we know about each WASH family member right now},
author = {Alekhina, O and Burstein, F and Billadeau, DD},
doi = {10.1242/jcs.199570},
journal = {J Cell Sci},
pages = {2235--2241},
title = {{Cellular function of WASP family proteins at a glance.}},
volume = {130},
year = {2017}
}
@article{Assoum2020,
abstract = {In 2011, KIAA1033/WASHC4 was associated with autosomal recessive intellectual disability (ARID) in a large consanguineous family comprising seven affected individuals with moderate ID and short stature. Since then, no other cases of KIAA1033 variants have been reported. Here we describe three additional patients (from two unrelated families) with syndromic ID due to compound heterozygous KIAA1033 variants ascertained by exome sequencing (ES). Two sisters, aged 4 and 5.5 years, had a stop-gain and a missense variants, each inherited from one parent (p.(Gln442*) and p.(Asp1048Gly)). Both had learning disabilities, macrocephaly, dysmorphic features, skeletal anomalies, and subependymal heterotopic nodules. In addition, the younger sibling had a congenital absence of the right internal carotid and bilateral sensorineural hearing loss. The third patient was aged 34 years and had two missense variants, one inherited from each parent (p.(Lys1079Arg) and p.(His503Arg)). This patient presented with mild ID, short stature, and microcephaly. KIAA1033 encodes a large protein (WASHC4), which is part of the WASH complex. The WASH complex is involved in the regulation of the fission of tubules that serve as transport intermediates during endosome sorting. Another member of the WASH complex, KIAA0196/WASHC5, has already been implicated in ARID with brain and cardiac malformations, under the designation of 3C or Ritscher-Schinzel syndrome (MIM{\#}20210). ES has proved efficient for finding replications of genes with insufficient data in the literature to be defined as new OMIM genes. We conclude that KIAA1033 is responsible for a heterogeneous ARID phenotype, and additional description will be needed to refine the clinical phenotype.},
author = {Assoum, Mirna and Bruel, Ange Line and Crenshaw, Melissa L and Delanne, Julian and Wentzensen, Ingrid M and McWalter, Kirsty and Dent, Karin M and Vitobello, Antonio and Kuentz, Paul and Thevenon, Julien and Duffourd, Yannis and Thauvin-Robinet, Christel and Faivre, Laurence},
doi = {10.1002/ajmg.a.61487},
file = {::},
issn = {15524833},
journal = {American Journal of Medical Genetics, Part A},
keywords = {KIAA1033/WASHC4,exome sequencing,intellectual disability},
number = {4},
pages = {792--797},
pmid = {31953988},
title = {{Novel KIAA1033/WASHC4 mutations in three patients with syndromic intellectual disability and a review of the literature}},
volume = {182},
year = {2020}
}
@article{Baker2006,
abstract = {Frontotemporal dementia (FTD) is the second most common cause of dementia in people under the age of 65 years. A large proportion of FTD patients (35-50{\%}) have a family history of dementia, consistent with a strong genetic component to the disease. In 1998, mutations in the gene encoding the microtubule- associated protein tau (MAPT) were shown to cause familial FTD with parkinsonism linked to chromosome 17q21 (FTDP-17). The neuropathology of patients with defined MAPT mutations is characterized by cytoplasmic neurofibrillary inclusions composed of hyperphosphorylated tau. However, in multiple FTD families with significant evidence for linkage to the same region on chromosome 17q21 (D17S1787-D17S806), mutations in MAPT have not been found and the patients consistently lack tau-immunoreactive inclusion pathology. In contrast, these patients have ubiquitin (ub)-immunoreactive neuronal cytoplasmic inclusions and characteristic lentiform ub-immunoreactive neuronal intranuclear inclusions. Here we demonstrate that in these families, FTD is caused by mutations in progranulin (PGRN) that are likely to create null alleles. PGRN is located 1.7 Mb centromeric of MAPT on chromosome 17q21.31 and encodes a 68.5-kDa secreted growth factor involved in the regulation of multiple processes including development, wound repair and inflammation. PGRN has also been strongly linked to tumorigenesis. Moreover, PGRN expression is increased in activated microglia in many neurodegenerative diseases including Creutzfeldt-Jakob disease, motor neuron disease and Alzheimer's disease. Our results identify mutations in PGRN as a cause of neurodegenerative disease and indicate the importance of PGRN function for neuronal survival. {\textcopyright} 2006 Nature Publishing Group.},
author = {Baker, Matt and Mackenzie, Ian R. and Pickering-Brown, Stuart M. and Gass, Jennifer and Rademakers, Rosa and Lindholm, Caroline and Snowden, Julie and Adamson, Jennifer and Sadovnick, A. Dessa and Rollinson, Sara and Cannon, Ashley and Dwosh, Emily and Neary, David and Melquist, Stacey and Richardson, Anna and Dickson, Dennis and Berger, Zdenek and Eriksen, Jason and Robinson, Todd and Zehr, Cynthia and Dickey, Chad A. and Crook, Richard and McGowan, Eileen and Mann, David and Boeve, Bradley and Feldman, Howard and Hutton, Mike},
doi = {10.1038/nature05016},
file = {::},
issn = {14764687},
journal = {Nature},
month = {aug},
number = {7105},
pages = {916--919},
pmid = {16862116},
publisher = {Nature Publishing Group},
title = {{Mutations in progranulin cause tau-negative frontotemporal dementia linked to chromosome 17}},
volume = {442},
year = {2006}
}
@article{Barnett2006,
abstract = {The development of ordered connections or "maps" within the nervous system is a common feature of sensory systems and is crucial for their normal function. NMDA receptors are known to play a key role in the formation of these maps; however, the intracellular signaling pathways that mediate the effects of glutamate are poorly understood. Here, we demonstrate that SynGAP, a synaptic Ras GTPase activating protein, is essential for the anatomical development of whisker-related patterns in the developing somatosensory pathways in rodent forebrain. Mice lacking SynGAP show only partial segregation of barreloids in the thalamus, and thalamocortical axons segregate into rows but do not form whisker-related patches. In cortex, layer 4 cells do not aggregate to form barrels. In Syngap+/- animals, barreloids develop normally, and thalamocortical afferents segregate in layer 4, but cell segregation is retarded. SynGAP is not necessary for the development of whisker-related patterns in the brainstem. Immunoelectron microscopy for SynGAP from layer 4 revealed a postsynaptic localization with labeling in developing postsynaptic densities (PSDs). Biochemically, SynGAP associates with the PSD in a PSD-95-independent manner, and Psd-95-/- animals develop normal barrels. These data demonstrate an essential role for SynGAP signaling in the activity-dependent development of whisker-related maps selectively in forebrain structures indicating that the intracellular pathways by which NMDA receptor activation mediates map formation differ between brain regions and developmental stage. Copyright {\textcopyright} 2006 Society for Neuroscience.},
author = {Barnett, Mark W. and Watson, Ruth F. and Vitalis, Tania and Porter, Karen and Komiyama, Noboru H. and Stoney, Patrick N. and Gillingwater, Thomas H. and Grant, Seth G.N. and Kind, Peter C.},
doi = {10.1523/JNEUROSCI.3164-05.2006},
file = {::},
issn = {02706474},
journal = {Journal of Neuroscience},
keywords = {Barrels,Development,NMDA receptors,PSD-95,Somatosensory,SynGAP},
month = {feb},
number = {5},
pages = {1355--1365},
pmid = {16452659},
title = {{Synaptic Ras GTPase activating protein regulates pattern formation in the trigeminal system of mice}},
volume = {26},
year = {2006}
}
@article{Bartels2019,
abstract = {Background/Aims: Endoplasmic reticulum (ER)-resident proteins with a C-terminal KDEL ER-retention sequence are captured in the Golgi apparatus by KDEL receptors (KDELRs). The binding of such proteins to these receptors induces their retrograde transport. Nevertheless, some KDEL proteins, such as Protein Disulfide Isomerases (PDIs), are found at the cell surface. PDIs target disulfide bridges in the extracellular domains of proteins, such as integrins or A Disintegrin And Metalloprotease 17 (ADAM17) leading to changes in the structure and function of these molecules. Integrins become activated and ADAM17 inactivated upon disulfide isomerization. The way that PDIs escape from retrograde transport and reach the plasma membrane remains far from clear. Various mechanisms might exist, depending on whether a local cell surface association or a more global secretion is required. Methods: To get a more detailed insight in the transport of PDIs to the cell surface, methods such as cell surface biotinylation, flow cytometric analysis, immunoprecipitation, fluorescence microscopy as well as labeling of cells with fluorescence labled recombinant PDIA6 was performed. Results: Here, we show that the C-terminal KDEL ER retention sequence is sufficient to prevent secretion of PDIA6 into the extracellular space but is mandatory for its association with the cell surface. The cell surface trafficking of PDIA1, PDIA3, and PDIA6 is dependent on KDELR1, which travels in a dynamic manner to the cell surface. This transport is assumed to result in PDI cell surface association, which differs from PDI inducible secretion into the extracellular space. Distinct PDIs differ in their trafficking properties. Endogenous KDELR1, detectable at the cell surface, might be involved not only in the transport of cell-surface-associated PDIs, but also in their retrieval and internalization from the extracellular space. Conclusion: Beside their ER retention motive PDIs travel to the cell surface. Here they target different proteins to render their function. To escape the ER PDIs travel via various pathways. One of them depends on the KDELR1, which can transport its target to the cell surface, where it is to be expected to release its cargo in close vicinity to its target molecules. Hence, the KDEL sequence is needed for cell surface association of PDIs, such as PDIA6.},
author = {Bartels, Anne Kathrin and G{\"{o}}ttert, Sascha and Desel, Christine and Sch{\"{a}}fer, Miriam and Krossa, Sebastian and Scheidig, Axel J. and Gr{\"{o}}tzinger, Joachim and Lorenzen, Inken},
doi = {10.33594/000000059},
file = {::},
issn = {14219778},
journal = {Cellular Physiology and Biochemistry},
keywords = {ADAM17,Cell surface,Extracellular PDIs,KDEL receptor},
number = {4},
pages = {850--868},
pmid = {30958660},
publisher = {Cell Physiol Biochem Press GmbH {\&} Co KG},
title = {{KDEL receptor 1 contributes to cell surface association of protein disulfide isomerases}},
volume = {52},
year = {2019}
}
@article{Bartuzi2016,
abstract = {The low-density lipoprotein receptor (LDLR) plays a pivotal role in clearing atherogenic circulating low-density lipoprotein (LDL) cholesterol. Here we show that the COMMD/CCDC22/CCDC93 (CCC) and the Wiskott-Aldrich syndrome protein and SCAR homologue (WASH) complexes are both crucial for endosomal sorting of LDLR and for its function. We find that patients with X-linked intellectual disability caused by mutations in CCDC22 are hypercholesterolaemic, and that COMMD1-deficient dogs and liver-specific Commd1 knockout mice have elevated plasma LDL cholesterol levels. Furthermore, Commd1 depletion results in mislocalization of LDLR, accompanied by decreased LDL uptake. Increased total plasma cholesterol levels are also seen in hepatic COMMD9-deficient mice. Inactivation of the CCC-associated WASH complex causes LDLR mislocalization, increased lysosomal degradation of LDLR and impaired LDL uptake. Furthermore, a mutation in the WASH component KIAA0196 (strumpellin) is associated with hypercholesterolaemia in humans. Altogether, this study provides valuable insights into the mechanisms regulating cholesterol homeostasis and LDLR trafficking.},
author = {Bartuzi, Paulina and Billadeau, Daniel D. and Favier, Robert and Rong, Shunxing and Dekker, Daphne and Fedoseienko, Alina and Fieten, Hille and Wijers, Melinde and Levels, Johannes H. and Huijkman, Nicolette and Kloosterhuis, Niels and {Van Der Molen}, Henk and Brufau, Gemma and Groen, Albert K. and Elliott, Alison M. and Kuivenhoven, Jan Albert and Plecko, Barbara and Grangl, Gernot and McGaughran, Julie and Horton, Jay D. and Burstein, Ezra and Hofker, Marten H. and {Van De Sluis}, Bart},
doi = {10.1038/ncomms10961},
file = {::},
issn = {20411723},
journal = {Nature Communications},
month = {mar},
pmid = {26965651},
publisher = {Nature Publishing Group},
title = {{CCC- and WASH-mediated endosomal sorting of LDLR is required for normal clearance of circulating LDL}},
volume = {7},
year = {2016}
}
@article{Beare2009,
abstract = {Advances in spinal cord injury (SCI) research are dependent on quality animal models, which in turn rely on sensitive outcome measures able to detect functional differences in animals following injury. To date, most measurements of dysfunction following SCI rely either on the subjective rating of observers or the slow throughput of manual gait assessment. The present study compares the gait of normal and contusion-injured mice using the TreadScan{\textregistered} system. TreadScan utilizes a transparent treadmill belt and a high-speed camera to capture the footprints of animals and automatically analyze gait characteristics. Adult female C57Bl/6 mice were introduced to the treadmill prior to receiving either a standardized mild, moderate, or sham contusion spinal cord injury. TreadScan gait analyses were performed weekly for 10 weeks and compared with scores on the Basso Mouse Scale (BMS). Results indicate that this software successfully differentiates sham animals from injured animals on a number of gait characteristics, including hindlimb swing time, stride length, toe spread, and track width. Differences were found between mild and moderate contusion injuries, indicating a high degree of sensitivity within the system. Rear track width, a measure of the animal's hindlimb base of support, correlated strongly both with spared white matter percentage and with terminal BMS. TreadScan allows for an objective and rapid behavioral assessment of locomotor function following mild-moderate contusive SCI, where the majority of mice still exhibit hindlimb weight support and plantar paw placement during stepping. {\textcopyright} 2009, Mary Ann Liebert, Inc.},
author = {Beare, Jason E. and Morehouse, Johnny R. and Devries, William H. and Enzmann, Gaby U. and Burke, Darlene A. and Magnuson, David S.K. and Whittemore, Scott R.},
doi = {10.1089/neu.2009.0914},
file = {::},
issn = {08977151},
journal = {Journal of Neurotrauma},
keywords = {Behavioral assessments,Locomotor function,Spinal cord injury (SCI)},
month = {nov},
number = {11},
pages = {2045--2056},
title = {{Gait analysis in normal and spinal contused mice using the treadscan system}},
url = {http://www.ncbi.nlm.nih.gov/pubmed/19886808 http://www.pubmedcentral.nih.gov/articlerender.fcgi?artid=PMC2813489},
volume = {26},
year = {2009}
}
@article{Billadeau2010,
abstract = {We recently showed that the Wiskott-Aldrich syndrome protein (WASP) family member, WASH, localizes to endosomal subdomains and regulates endocytic vesicle scission in an Arp2/3-dependent manner. Mechanisms regulating WASH activity are unknown. Here we show that WASH functions in cells within a 500 kDa core complex containing Strumpellin, FAM21, KIAA1033 (SWIP), and CCDC53. Although recombinant WASH is constitutively active toward the Arp2/3 complex, the reconstituted core assembly is inhibited, suggesting that it functions in cells to regulate actin dynamics through WASH. FAM21 interacts directly with CAPZ and inhibits its actin-capping activity. Four of the five core components show distant (approximately 15{\%} amino acid sequence identify) but significant structural homology to components of a complex that negatively regulates the WASP family member, WAVE. Moreover, biochemical and electron microscopic analyses show that the WASH and WAVE complexes are structurally similar. Thus, these two distantly related WASP family members are controlled by analogous structurally related mechanisms. Strumpellin is mutated in the human disease hereditary spastic paraplegia, and its link to WASH suggests that misregulation of actin dynamics on endosomes may play a role in this disorder.},
annote = {*Has cryoEM data on WASH complex and WAVE complex!},
author = {Billadeau, D. D. and Jia, D. and Rosen, M. K. and Gomez, T. S. and Umetani, J. and Metlagel, Z. and Otwinowski, Z.},
doi = {10.1073/pnas.0913293107},
issn = {0027-8424},
journal = {Proceedings of the National Academy of Sciences},
title = {{WASH and WAVE actin regulators of the Wiskott-Aldrich syndrome protein (WASP) family are controlled by analogous structurally related complexes}},
year = {2010}
}
@article{Binda2019,
abstract = {Retromer is an evolutionarily conserved endosomal trafficking complex that mediates the retrieval of cargo proteins from a degradative pathway for sorting back to the cell surface. To promote cargo recycling, the core retromer trimer of VPS (vacuolar protein sorting)26, VPS29 and VPS35 recognises cargo either directly, or through an adaptor protein, the most well characterised of which is the PDZ [postsynaptic density 95 (PSD95), disk large, zona occludens] domain-containing sorting nexin SNX27. Neuroligins (NLGs) are postsynaptic trans-synaptic scaffold proteins that function in the clustering of postsynaptic proteins to maintain synaptic stability. Here, we show that each of the NLGs (NLG1-3) bind to SNX27 in a direct PDZ ligand-dependent manner. Depletion of SNX27 from neurons leads to a decrease in levels of each NLG protein and, for NLG2, this occurs as a result of enhanced lysosomal degradation. Notably, while depletion of the core retromer component VPS35 leads to a decrease in NLG1 and NLG3 levels, NLG2 is unaffected, suggesting that, for this cargo, SNX27 acts independently of retromer. Consistent with loss of SNX27 leading to enhanced lysosomal degradation of NLG2, knockdown of SNX27 results in fewer NLG2 clusters in cultured neurons, and loss of SNX27 or VPS35 reduces the size and number of gephyrin clusters. Together, these data indicate that NLGs are SNX27-retromer cargoes and suggest that SNX27-retromer controls inhibitory synapse number, at least in part through trafficking of NLG2.},
author = {Binda, Caroline S. and Nakamura, Yasuko and Henley, Jeremy M. and Wilkinson, Kevin A.},
doi = {10.1042/BCJ20180504},
issn = {14708728},
journal = {Biochemical Journal},
month = {jan},
number = {2},
pages = {293--306},
pmid = {30602588},
publisher = {Portland Press Ltd},
title = {{Sorting nexin 27 rescues neuroligin 2 from lysosomal degradation to control inhibitory synapse number}},
volume = {476},
year = {2019}
}
@misc{Blackstone2011,
abstract = {Voluntary movement is a fundamental way in which animals respond to, and interact with, their environment. In mammals, the main CNS pathway controlling voluntary movement is the corticospinal tract, which encompasses connections between the cerebral motor cortex and the spinal cord. Hereditary spastic paraplegias (HSPs) are a group of genetic disorders that lead to a length-dependent, distal axonopathy of fibres of the corticospinal tract, causing lower limb spasticity and weakness. Recent work aimed at elucidating the molecular cell biology underlying the HSPs has revealed the importance of basic cellular processes - especially membrane trafficking and organelle morphogenesis and distribution - in axonal maintenance and degeneration. {\textcopyright} 2011 Macmillan Publishers Limited. All rights reserved.},
author = {Blackstone, Craig and O'Kane, Cahir J. and Reid, Evan},
booktitle = {Nature Reviews Neuroscience},
doi = {10.1038/nrn2946},
file = {::},
issn = {1471003X},
month = {jan},
number = {1},
pages = {31--42},
publisher = {NIH Public Access},
title = {{Hereditary spastic paraplegias: Membrane traffic and the motor pathway}},
volume = {12},
year = {2011}
}
@article{Brunk2002,
abstract = {The accumulation of lipofuscin within postmitotic cells is a recognized hallmark of aging occuring with a rate inversely related to longevity. Lipofuscin is an intralysosomal, polymeric substance, primarily composed of cross-linked protein residues, formed due to iron-catalyzed oxidative processes. Because it is undegradable and cannot be removed via exocytosis, lipofuscin accumulation in postmitotic cells is inevitable, whereas proliferative cells efficiently dilute it during division. The rate of lipofuscin formation can be experimentally manipulated. In cell culture models, oxidative stress (e.g., exposure to 40{\%} ambient oxygen or low molecular weight iron) promotes lipofuscin accumulation, whereas growth at 8{\%} oxygen and treatment with antioxidants or iron-chelators diminish it. Lipofuscin is a fluorochrome and may sensitize lysosomes to visible light, a process potentially important for the pathogenesis of age-related macular degeneration. Lipofuscin-associated iron sensitizes lysosomes to oxidative stress, jeopardizing lysosomal stability and causing apoptosis due to release of lysosomal contents. Lipofuscin accumulation may also diminish autophagocytotic capacity by acting as a sink for newly produced lysosomal enzymes and, therefore, interfere with recycling of cellular components. Lipofuscin, thus, may be much more directly related to cellular degeneration at old age than was hitherto believed. {\textcopyright} 2002 Elsevier Science Inc.},
author = {Brunk, Ulf T. and Terman, Alexei},
doi = {10.1016/S0891-5849(02)00959-0},
issn = {08915849},
journal = {Free Radical Biology and Medicine},
keywords = {Age pigment,Aging,Autophagocytosis,Centrophenoxine,Free radicals,Lysosomes,Mitochondria,Oxidative stress},
month = {sep},
number = {5},
pages = {611--619},
pmid = {12208347},
publisher = {Elsevier Inc.},
title = {{Lipofuscin: Mechanisms of age-related accumulation and influence on cell function}},
url = {https://pubmed.ncbi.nlm.nih.gov/12208347/},
volume = {33},
year = {2002}
}
@misc{Cai2016,
abstract = {The common underlying feature of most neurodegenerative diseases such as Alzheimer disease (AD), prion diseases, Parkinson disease (PD), and amyotrophic lateral sclerosis (ALS) involves accumulation of misfolded proteins leading to initiation of endoplasmic reticulum (ER) stress and stimulation of the unfolded protein response (UPR). Additionally, ER stress more recently has been implicated in the pathogenesis of HIV-associated neurocognitive disorders (HAND). Autophagy plays an essential role in the clearance of aggregated toxic proteins and degradation of the damaged organelles. There is evidence that autophagy ameliorates ER stress by eliminating accumulated misfolded proteins. Both abnormal UPR and impaired autophagy have been implicated as a causative mechanism in the development of various neurodegenerative diseases. This review highlights recent advances in the field on the role of ER stress and autophagy in AD, prion diseases, PD, ALS and HAND with the involvement of key signaling pathways in these processes and implications for future development of therapeutic strategies.},
author = {Cai, Yu and Arikkath, Jyothi and Yang, Lu and Guo, Ming Lei and Periyasamy, Palsamy and Buch, Shilpa},
booktitle = {Autophagy},
doi = {10.1080/15548627.2015.1121360},
file = {::},
issn = {15548635},
keywords = {Alzheimer disease,Amyotrophic lateral sclerosis and HIV-associated n,Autophagy,ER stress,Neurodegenerative disorders,Parkinson disease,Prion diseases},
month = {jan},
number = {2},
pages = {225--244},
publisher = {Taylor and Francis Inc.},
title = {{Interplay of endoplasmic reticulum stress and autophagy in neurodegenerative disorders}},
volume = {12},
year = {2016}
}
@article{Calvo2016,
abstract = {Mitochondria are complex organelles that house essential pathways involved in energy metabolism, ion homeostasis, signalling and apoptosis. To understand mitochondrial pathways in health and disease, it is crucial to have an accurate inventory of the organelle's protein components. In 2008, we made substantial progress toward this goal by performing in-depth mass spectrometry of mitochondria from 14 organs, epitope tagging/microscopy and Bayesian integration to assemble MitoCarta (www.broadinstitute.org/pubs/MitoCarta): an inventory of genes encoding mitochondrial-localized proteins and their expression across 14 mouse tissues. Using the same strategy we have now reconstructed this inventory separately for human and for mouse based on (i) improved gene transcript models, (ii) updated literature curation, including results from proteomic analyses of mitochondrial subcompartments, (iii) improved homology mapping and (iv) updated versions of all seven original data sets. The updated human MitoCarta2.0 consists of 1158 human genes, including 918 genes in the original inventory as well as 240 additional genes. The updated mouse MitoCarta2.0 consists of 1158 genes, including 967 genes in the original inventory plus 191 additional genes. The improved MitoCarta 2.0 inventory provides a molecular framework for system-level analysis of mammalian mitochondria.},
author = {Calvo, Sarah E. and Clauser, Karl R. and Mootha, Vamsi K.},
doi = {10.1093/nar/gkv1003},
issn = {13624962},
journal = {Nucleic Acids Research},
pmid = {26450961},
title = {{MitoCarta2.0: An updated inventory of mammalian mitochondrial proteins}},
year = {2016}
}
@article{Carnell2011,
abstract = {WASP and SCAR homologue (WASH) is a recently identified and evolutionarily conserved regulator of actin polymerization. In this paper, we show that WASH coats mature Dictyostelium discoideum lysosomes and is essential for exocytosis of indigestible material. A related process, the expulsion of the lethal endosomal pathogen Cryptococcus neoformans from mammalian macrophages, also uses WASH-coated vesicles, and cells expressing dominant negative WASH mutants inefficiently expel C. neoformans. D. discoideum WASH causes filamentous actin (F-actin) patches to form on lysosomes, leading to the removal of vacuolar adenosine triphosphatase (V-ATPase) and the neutralization of lysosomes to form postlysosomes. Without WASH, no patches or coats are formed, neutral postlysosomes are not seen, and indigestible material such as dextran is not exocytosed. Similar results occur when actin polymerization is blocked with latrunculin. V-ATPases are known to bind avidlyto F-actin. Our data imply a new mechanism, actin-mediatedsorting, in which WASH and the Arp2/3 complex polymerize actin on vesicles to drive the separation and recycling of proteins such as the V-ATPase. {\textcopyright} 2011 Carnell et al.},
author = {Carnell, Michael and Zech, Tobias and Calaminus, Simon D. and Ura, Seiji and Hagedorn, Monica and Johnston, Simon A. and May, Robin C. and Soldati, Thierry and Machesky, Laura M. and Insall, Robert H.},
doi = {10.1083/jcb.201009119},
file = {::},
issn = {00219525},
journal = {Journal of Cell Biology},
month = {may},
number = {5},
pages = {831--839},
pmid = {21606208},
publisher = {The Rockefeller University Press},
title = {{Actin polymerization driven by WASH causes V-ATPase retrieval and vesicle neutralization before exocytosis}},
volume = {193},
year = {2011}
}
@article{Chen2011,
abstract = {Retrograde synaptic signaling by endocannabinoids (eCBs) is a widespread mechanism for activity-dependent inhibition of synaptic strength in the brain. Although prevalent, the conditions for eliciting eCB-mediated synaptic depression vary among brain circuits. As yet, relatively little is known about the molecular mechanisms underlying this variation, although the initial signaling events are likely dictated by postsynaptic proteins. SAP90/PSD-95-associated proteins (SAPAPs) are a family of postsynaptic proteins unique to excitatory synapses. Using Sapap3 knock-out (KO) mice, we find that, in the absence of SAPAP3, striatal medium spiny neuron (MSN) excitatory synapses exhibit eCB-mediated synaptic depression under conditions that do not normally activate this process. The anomalous synaptic plasticity requires type 5 metabotropic glutamate receptors (mGluR5s), which we find are dysregulated in Sapap3KOMSNs. Both surface expression and activity of mGluR5s are increased in Sapap3 KO MSNs, suggesting that enhanced mGluR5 activity may drive the anomalous synaptic plasticity. In direct support of this possibility, we find that, in wild-type (WT) MSNs, pharmacological enhancement of mGluR5 by a positive allosteric modulator is sufficient to reproduce the increased synaptic depression seen in Sapap3 KO MSNs. The same pharmacologic treatment, however, fails to elicit further depression inKOMSNs. Under conditions that are sufficient to engage eCB-mediated synaptic depression in WT MSNs, Sapap3 deletion does not alter the magnitude of the response. These results identify a role for SAPAP3 in the regulation of postsynaptic mGluRs and eCB-mediated synaptic plasticity. SAPAPs, through their effect on mGluR activity, may serve as regulatory molecules gating the threshold for inducing eCB-mediated synaptic plasticity. {\textcopyright} 2011 the authors.},
author = {Chen, Meng and Wan, Yehong and Ade, Kristen and Ting, Jonathan and Feng, Guoping and Calakos, Nicole},
doi = {10.1523/JNEUROSCI.1701-11.2011},
file = {::},
issn = {02706474},
journal = {Journal of Neuroscience},
month = {jun},
number = {26},
pages = {9563--9573},
pmid = {21715621},
title = {{Sapap3 deletion anomalously activates short-term endocannabinoid-mediated synaptic plasticity}},
volume = {31},
year = {2011}
}
@article{Cheng2018,
abstract = {Despite widespread distribution of LAMP1 and the heterogeneous nature of LAMP1-labeled compartments, LAMP1 is routinely used as a lysosomal marker, and LAMP1-positive organelles are often referred to as lysosomes. In this study, we use immunoelectron microscopy and confocal imaging to provide quantitative analysis of LAMP1 distribution in various autophagic and endolysosomal organelles in neurons. Our study demonstrates that a significant portion of LAMP1-labeled organelles do not contain detectable lysosomal hydrolases including cathepsins D and B and glucocerebrosidase. A bovine serum albumin-gold pulse-chase assay followed by ultrastructural analysis suggests a heterogeneity of degradative capacity in LAMP1-labeled endolysosomal organelles. Gradient fractionation displays differential distribution patterns of LAMP1/2 and cathepsins D/B in neurons. We further reveal that LAMP1 intensity in familial amyotrophic lateral sclerosis-linked motor neurons does not necessarily reflect lysosomal deficits in vivo. Our study suggests that labeling a set of lysosomal hydrolases combined with various endolysosomal markers would be more accurate than simply relying on LAMP1/2 staining to assess neuronal lysosome distribution, trafficking, and functionality under physiological and pathological conditions.},
author = {Cheng, Xiu Tang and Xie, Yu Xiang and Zhou, Bing and Huang, Ning and Farfel-Becker, Tamar and Sheng, Zu Hang},
doi = {10.1083/jcb.201711083},
file = {::},
issn = {15408140},
journal = {Journal of Cell Biology},
month = {sep},
number = {9},
pages = {3127--3139},
pmid = {29695488},
publisher = {Rockefeller University Press},
title = {{Characterization of LAMP1-labeled nondegradative lysosomal and endocytic compartments in neurons}},
volume = {217},
year = {2018}
}
@article{Clement2012,
abstract = {Mutations that cause intellectual disability (ID) and autism spectrum disorder (ASD) are commonly found in genes that encode for synaptic proteins. However, it remains unclear how mutations that disrupt synapse function impact intellectual ability. In the SYNGAP1 mouse model of ID/ASD, we found that dendritic spine synapses develop prematurely during the early postnatal period. Premature spine maturation dramatically enhanced excitability in the developing hippocampus, which corresponded with the emergence of behavioral abnormalities. Inducing SYNGAP1 mutations after critical developmental windows closed had minimal impact on spine synapse function, whereas repairing these pathogenic mutations in adulthood did not improve behavior and cognition. These data demonstrate that SynGAP protein acts as a critical developmental repressor of neural excitability that promotes the development of life-long cognitive abilities. We propose that the pace of dendritic spine synapse maturation in early life is a critical determinant of normal intellectual development. ?? 2012 Elsevier Inc.},
archivePrefix = {arXiv},
arxivId = {NIHMS150003},
author = {Clement, James P. and Aceti, Massimiliano and Creson, Thomas K. and Ozkan, Emin D. and Shi, Yulin and Reish, Nicholas J. and Almonte, Antoine G. and Miller, Brooke H. and Wiltgen, Brian J. and Miller, Courtney A. and Xu, Xiangmin and Rumbaugh, Gavin},
doi = {10.1016/j.cell.2012.08.045},
eprint = {NIHMS150003},
isbn = {0092-8674},
issn = {00928674},
journal = {Cell},
number = {4},
pages = {709--723},
pmid = {23141534},
title = {{Pathogenic SYNGAP1 mutations impair cognitive development by disrupting maturation of dendritic spine synapses}},
volume = {151},
year = {2012}
}
@article{Connor-Robson2019,
abstract = {Background: Mutations in LRRK2 are the most common cause of autosomal dominant Parkinson's disease, and the relevance of LRRK2 to the sporadic form of the disease is becoming ever more apparent. It is therefore essential that studies are conducted to improve our understanding of the cellular role of this protein. Here we use multiple models and techniques to identify the pathways through which LRRK2 mutations may lead to the development of Parkinson's disease. Methods: A novel integrated transcriptomics and proteomics approach was used to identify pathways that were significantly altered in iPSC-derived dopaminergic neurons carrying the LRRK2-G2019S mutation. Western blotting, immunostaining and functional assays including FM1-43 analysis of synaptic vesicle endocytosis were performed to confirm these findings in iPSC-derived dopaminergic neuronal cultures carrying either the LRRK2-G2019S or the LRRK2-R1441C mutation, and LRRK2 BAC transgenic rats, and post-mortem human brain tissue from LRRK2-G2019S patients. Results: Our integrated -omics analysis revealed highly significant dysregulation of the endocytic pathway in iPSC-derived dopaminergic neurons carrying the LRRK2-G2019S mutation. Western blot analysis confirmed that key endocytic proteins including endophilin I-III, dynamin-1, and various RAB proteins were downregulated in these cultures and in cultures carrying the LRRK2-R1441C mutation, compared with controls. We also found changes in expression of 25 RAB proteins. Changes in endocytic protein expression led to a functional impairment in clathrin-mediated synaptic vesicle endocytosis. Further to this, we found that the endocytic pathway was also perturbed in striatal tissue of aged LRRK2 BAC transgenic rats overexpressing either the LRRK2 wildtype, LRRK2-R1441C or LRRK2-G2019S transgenes. Finally, we found that clathrin heavy chain and endophilin I-III levels are increased in human post-mortem tissue from LRRK2-G2019S patients compared with controls. Conclusions: Our study demonstrates extensive alterations across the endocytic pathway associated with LRRK2 mutations in iPSC-derived dopaminergic neurons and BAC transgenic rats, as well as in post-mortem brain tissue from PD patients carrying a LRRK2 mutation. In particular, we find evidence of disrupted clathrin-mediated endocytosis and suggest that LRRK2-mediated PD pathogenesis may arise through dysregulation of this process.},
author = {Connor-Robson, Natalie and Booth, Heather and Martin, Jeffrey G. and Gao, Benbo and Li, Kejie and Doig, Natalie and Vowles, Jane and Browne, Cathy and Klinger, Laura and Juhasz, Peter and Klein, Christine and Cowley, Sally A. and Bolam, Paul and Hirst, Warren and Wade-Martins, Richard},
doi = {10.1016/j.nbd.2019.04.005},
file = {::},
issn = {1095953X},
journal = {Neurobiology of Disease},
keywords = {Clathrin,Endocytosis,Endophilin,LRRK2,Parkinson's disease,Rabs},
month = {jul},
pages = {512--526},
pmid = {30954703},
publisher = {Academic Press Inc.},
title = {{An integrated transcriptomics and proteomics analysis reveals functional endocytic dysregulation caused by mutations in LRRK2}},
volume = {127},
year = {2019}
}
@article{DeBot2013,
abstract = {SPG8 is a rare autosomal dominant hereditary spastic paraplegia (AD-HSP), with only six SPG8 families described so far. Our purpose was to screen for KIAA0196 (SPG8) mutations in AD-HSP patients and to investigate their phenotype. Extensive family investigation was performed after positive KIAA0196 mutation analysis, which was part of an on-going mutation screening effort in AD-HSP patients. A novel pathogenic KIAA0196 mutation p.(Gly696Ala) was identified in two AD-HSP patients, who subsequently were shown to belong to a single large Dutch pedigree with more than 10 affected family members. The phenotype consisted of a pure HSP with ages at onset between 20 and 60 years, distally reduced vibration sense in the legs in all, and urinary urgency in seven out of 10 patients. Frequent features were exercise- or emotion-induced increase of spasticity and gait problems and chronic nonspecific lower back and joint pains. We have identified a fourth pathogenic KIAA0196 mutation in a Dutch HSP-family, the seventh family worldwide, with a less severe clinical course than described before. {\textcopyright} 2013 Springer-Verlag Berlin Heidelberg.},
author = {{De Bot}, Susanne T. and Vermeer, Sascha and Buijsman, Wendy and Heister, Angelien and Voorendt, Marsha and Verrips, Aad and Scheffer, Hans and Kremer, Hubertus P.H. and {Van De Warrenburg}, Bart P.C. and Kamsteeg, Erik Jan},
doi = {10.1007/s00415-013-6870-x},
issn = {03405354},
journal = {Journal of Neurology},
keywords = {Hereditary spastic paraplegia,KIAA0196-gene,SPG8,Strumpellin},
month = {jul},
number = {7},
pages = {1765--1769},
publisher = {J Neurol},
title = {{Pure adult-onset Spastic Paraplegia caused by a novel mutation in the KIAA0196 (SPG8) gene}},
volume = {260},
year = {2013}
}
@article{DelOlmo2019,
abstract = {RAB GTPases are central modulators of membrane trafficking. They are under the dynamic regulation of activating guanine exchange factors (GEFs) and inactivating GTPase-activating proteins (GAPs). Once activated, RABs recruit a large spectrum of effectors to control trafficking functions of eukaryotic cells. Multiple proteomic studies, using pull-down or yeast two-hybrid approaches, have identified a number of RAB interactors. However, due to the in vitro nature of these approaches and inherent limitations of each technique, a comprehensive definition of RAB interactors is still lacking. By comparing quantitative affinity purifications of GFP:RAB21 with APEX2-mediated proximity labeling of RAB4a, RAB5a, RAB7a, and RAB21, we find that APEX2 proximity labeling allows for the comprehensive identification of RAB regulators and interactors. Importantly, through biochemical and genetic approaches, we establish a novel link between RAB21 and the WASH and retromer complexes, with functional consequences on cargo sorting. Hence, APEX2-mediated proximity labeling of RAB neighboring proteins represents a new and efficient tool to define RAB functions.},
author = {{Del Olmo}, Tomas and Lauzier, Annie and Normandin, Caroline and Larcher, Rapha{\"{e}}lle and Lecours, Mia and Jean, Dominique and Lessard, Louis and Steinberg, Florian and Boisvert, Fran{\c{c}}ois-Michel and Jean, Steve},
doi = {10.15252/embr.201847192},
issn = {1469-221X},
journal = {EMBO reports},
keywords = {APEX2,Methods {\&} Resources,RAB GTPases,WASH complex Subject Categories Membrane {\&} Intrace,clathrin-independent endocytosis,retromer},
number = {2},
pmid = {30610016},
title = {{APEX2‐mediated RAB proximity labeling identifies a role for RAB21 in clathrin‐independent cargo sorting}},
volume = {20},
year = {2019}
}
@article{Derivery2010,
abstract = {WASH is the Arp2/3 activating protein that is localized at the surface of endosomes, where it induces the formation of branched actin networks. This activity of WASH favors, in collaboration with dynamin, the fission of transport intermediates from endosomes, and hence regulates endosomal trafficking of several cargos. We have purified a novel stable multiprotein complex containing WASH, the WASH complex, and we examine here the evolutionary conservation of its seven subunits across diverse eukaryotic phyla. This analysis supports the idea that the invention of the WASH complex has involved the incorporation of an independent complex, the CapZ $\alpha$/$\beta$ heterodimer, forming the so-called Capping Protein (CP), as illustrated by the yeasts S. cerevisiae and S. pombe, which possess the CP heterodimer but no other subunits of the WASH complex. The alignements of the orthologous genes that we have generated give a view on the conservation of the different subunits and on their organization into domains. Moreover, we propose here a unique nomenclature for the different subunits to prevent future confusions in the field. {\textcopyright} 2010 Landes Bioscience.},
author = {Derivery, Emmanuel and Gautreau, Alexis},
doi = {10.4161/cib.3.3.11185},
file = {::},
issn = {19420889},
journal = {Communicative and Integrative Biology},
keywords = {Arp2/3 complex,CAPZ,CCDC53,CP,Endosome,KIAA0196,KIAA0592,KIAA1033,Strumpellin,VPEF},
month = {may},
number = {3},
pages = {227--230},
publisher = {Taylor {\&} Francis},
title = {{Evolutionary conservation of the WASH complex, an actin polymerization machine involved in endosomal fission}},
volume = {3},
year = {2010}
}
@article{Derivery2009,
annote = {First paper to show role of WASH complex at sorting and recycling endosomes
- believed to be 7 subunits (not 5)},
author = {Derivery, Emmanuel and Sousa, C and Gautier, JJ and Lombard, B and Loew, D and Gautreau, A},
doi = {10.1016/j.devcel.2009.09.010},
journal = {Dev Cell},
number = {5},
pages = {712--23},
pmid = {19922875},
title = {{The Arp2/3 activator WASH controls the fission of endosomes through a large multiprotein complex.}},
volume = {17},
year = {2009}
}
@misc{Dutta2016,
abstract = {Since the late 18th century, the murine model has been widely used in biomedical research (about 59{\%} of total animals used) as it is compact, cost-effective, and easily available, conserving almost 99{\%} of human genes and physiologically resembling humans. Despite the similarities, mice have a diminutive lifespan compared to humans. In this study, we found that one human year is equivalent to nine mice days, although this is not the case when comparing the lifespan of mice versus humans taking the entire life at the same time without considering each phase separately. Therefore, the precise correlation of age at every point in their lifespan must be determined. Determining the age relation between mice and humans is necessary for setting up experimental murine models more analogous in age to humans. Thus, more accuracy can be obtained in the research outcome for humans of a specific age group, although current outcomes are based on mice of an approximate age. To fill this gap between approximation and accuracy, this review article is the first to establish a precise relation between mice age and human age, following our previous article, which explained the relation in ages of laboratory rats with humans in detail.},
author = {Dutta, Sulagna and Sengupta, Pallav},
booktitle = {Life Sciences},
doi = {10.1016/j.lfs.2015.10.025},
issn = {18790631},
keywords = {Age,Developmental biology,Human age,Laboratory mice,Mice age,Physiology},
month = {may},
pages = {244--248},
publisher = {Elsevier Inc.},
title = {{Men and mice: Relating their ages}},
url = {https://pubmed.ncbi.nlm.nih.gov/26596563/},
volume = {152},
year = {2016}
}
@article{Edvardson2012,
abstract = {Parkinson disease is caused by neuronal loss in the substantia nigra which manifests by abnormality of movement, muscle tone, and postural stability. Several genes have been implicated in the pathogenesis of Parkinson disease, but the underlying molecular basis is still unknown for {\~{}}70{\%} of the patients. Using homozygosity mapping and whole exome sequencing we identified a deleterious mutation in DNAJC6 in two patients with juvenile Parkinsonism. The mutation was associated with abnormal transcripts and marked reduced DNAJC6 mRNA level. DNAJC6 encodes the HSP40 Auxilin, a protein which is selectively expressed in neurons and confers specificity to the ATPase activity of its partner Hcs70 in clathrin uncoating. In Auxilin null mice it was previously shown that the abnormally increased retention of assembled clathrin on vesicles and in empty cages leads to impaired synaptic vesicle recycling and perturbed clathrin mediated endocytosis. Endocytosis function, studied by transferring uptake, was normal in fibroblasts from our patients, likely because of the presence of another J-domain containing partner which co-chaperones Hsc70-mediated uncoating activity in non-neuronal cells. The present report underscores the importance of the endocytic/lysosomal pathway in the pathogenesis of Parkinson disease and other forms of Parkinsonism. {\textcopyright} 2012 Edvardson et al.},
author = {Edvardson, Simon and Cinnamon, Yuval and Ta-Shma, Asaf and Shaag, Avraham and Yim, Yang In and Zenvirt, Shamir and Jalas, Chaim and Lesage, Suzanne and Brice, Alexis and Taraboulos, Albert and Kaestner, Klaus H. and Greene, Lois E. and Elpeleg, Orly},
doi = {10.1371/journal.pone.0036458},
file = {::},
issn = {19326203},
journal = {PLoS ONE},
month = {may},
number = {5},
title = {{A deleterious mutation in DNAJC6 encoding the neuronal-specific clathrin-uncoating Co-chaperone auxilin, is associated with juvenile parkinsonism}},
volume = {7},
year = {2012}
}
@article{Elliott2013,
abstract = {Background Ritscher-Schinzel syndrome (RSS) is a clinically heterogeneous disorder characterised by distinctive craniofacial features in addition to cerebellar and cardiac anomalies. It has been described in different populations and is presumed to follow autosomal recessive inheritance. In an effort to identify the underlying genetic cause of RSS, affected individuals from a First Nations (FN) community in northern Manitoba, Canada, were enrolled in this study. Methods Homozygosity mapping by SNP array and Sanger sequencing of the candidate genes in a 1Mb interval on chromosome 8q24.13 were performed on genomic DNA from eight FN RSS patients, eight of their parents and five unaffected individuals (control subjects) from this geographic isolate. Results All eight patients were homozygous for a novel splice site mutation in KIAA0196. RNA analysis revealed an approximate eightfold reduction in the relative amount of a KIAA0196 transcript lacking exon 27. A 60{\%} reduction in the amount of strumpellin protein was observed on western blot. Conclusions We have identified a mutation in KIAA0196 as the cause of the form of RSS characterised in our cohort. The ubiquitous expression and highly conserved nature of strumpellin, the product of KIAA0196, is consistent with the complex and multisystem nature of this disorder.},
author = {Elliott, Alison M. and Simard, Louise R. and Coghlan, Gail and Chudley, Albert E. and Chodirker, Bernard N. and Greenberg, Cheryl R. and Burch, Tanya and Ly, Valentina and Hatch, Grant M. and Zelinski, Teresa},
doi = {10.1136/jmedgenet-2013-101715},
issn = {00222593},
journal = {Journal of Medical Genetics},
month = {dec},
number = {12},
pages = {819--822},
publisher = {BMJ Publishing Group Ltd},
title = {{A novel mutation in KIAA0196: Identification of a gene involved in Ritscher-Schinzel/3C syndrome in a First Nations cohort}},
volume = {50},
year = {2013}
}
@article{Eng1994,
abstract = {Fabry disease, an X‐linked inborn error of glycosphingolipid catabolism, results from mutations in the $\alpha$‐galactosidase A gene at Xq22.1. Studies of the mutations in unrelated Fabry families have identified a variety of lesions indicating the molecular genetic heterogeneity underlying the disease. Forty‐nine different mutations have been described including five partial gene deletions, one partial gene duplication, nine small deletions and insertions, three splice junction consensus site alterations, and 31 coding region single base substitutions. Most mutations resulted in the classical disease phenotype; however, five missense mutations were detected in atypical hemizygotes who were asymptomatic or had symptoms confined to the heart, including N215S, which was described in three unrelated atypical males. Most mutations were confined to a single pedigree with the exception of N215S, R227Q, R227X, R342Q, and R342X, which were each found in several unrelated families. Five of the 14 coding region CpG dinucleotides were sites of point mutations including the CpGs in codons 227 and 342, which were each mutated in both orientations. The identification of the mutation in a given Fabry family permits precise prenatal diagnosis and heterozygote detection of other family members with this X‐linked recessive disease. Studies of additional Fabry families will provide information on the nature and frequency of the mutations causing this disease as well as potential insights into the structure/ function relationships of this lysosomal hydrolase. {\textcopyright} 1994 Wiley‐Liss, Inc. Copyright {\textcopyright} 1994 Wiley‐Liss, Inc., A Wiley Company},
author = {Eng, Christine M. and Desnick, Robert J.},
doi = {10.1002/humu.1380030204},
issn = {10981004},
journal = {Human Mutation},
keywords = {Fabry disease,X‐Chromosome,$\alpha$‐Galactosidase A},
number = {2},
pages = {103--111},
publisher = {Hum Mutat},
title = {{Molecular basis of fabry disease: Mutations and polymorphisms in the human $\alpha$‐galactosidase A gene}},
url = {https://pubmed.ncbi.nlm.nih.gov/7911050/},
volume = {3},
year = {1994}
}
@article{English2000,
abstract = {Membrane type 4 matrix metalloproteinase (MT4-MMP) shows the least sequence homology to the other MT-MMPs, suggesting a distinct function for this protein. We have isolated a complete cDNA corresponding to the mouse homologue which includes the signal peptide and a complete pro-domain, features that were lacking from the human form originally isolated. Mouse MT4-MMP (mMT4-MMP) expressed in COS-7 cells is located at the cell surface but does not show ability to activate pro-MMP2. The pro-catalytic domain was expressed in Escherichia coli as insoluble inclusions and active enzyme recovered after refolding. Activity of the isolated catalytic domain against synthetic peptides commonly used for MMP enzyme assays could be inhibited by TIMP1, -2, and -3. The recombinant mMT4-MMP catalytic domain was also unable to activate pro-MMP2 and was very poor at hydrolyzing components of the extracellular matrix with the exception of fibrinogen and fibrin. mMT4-MMP was able to hydrolyze efficiently a peptide consisting of the pro-tumor necrosis factor $\alpha$ (TNF$\alpha$) cleavage site, a glutathione S-transferase-pro- TNF$\alpha$ fusion protein, and was found to shed pro-TNF$\alpha$ when co-transfected in COS-7 cells. MT4-MMP was detected by Western blot in monocyte/macrophage cell lines which in combination with its fibrinolytic and TNF$\alpha$-converting activity suggests a role in inflammation.},
author = {English, William R. and Puente, Xose S. and Freije, Jos{\'{e}} M.P. and Kn{\"{a}}uper, Vera and Amour, Augustin and Merryweather, Ann and L{\'{o}}pez-Ot{\'{i}}n, Carlos and Murphy, Gillian},
doi = {10.1074/jbc.275.19.14046},
file = {::},
issn = {00219258},
journal = {Journal of Biological Chemistry},
keywords = {ADAM Proteins,ADAM17 Protein,Amino Acid,Animals,Base Sequence,Catalysis,Cell Line,Complementary,DNA,Enzyme Activation,Enzyme Precursors / metabolism*,G Murphy,Humans,Leukocytes / enzymology,MEDLINE,Matrix Metalloproteinases,Matrix Metalloproteinases*,Membrane-Associated,Metalloendopeptidases / genetics,Metalloendopeptidases / metabolism*,Mice,Molecular Sequence Data,NCBI,NIH,NLM,National Center for Biotechnology Information,National Institutes of Health,National Library of Medicine,Non-U.S. Gov't,Post-Translational,Protease Inhibitors / pharmacology,Protein Folding,Protein Processing,PubMed Abstract,Research Support,Sequence Homology,Tumor Necrosis Factor-alpha / metabolism*,W R English,X S Puente,doi:10.1074/jbc.275.19.14046,pmid:10799478},
month = {may},
number = {19},
pages = {14046--14055},
pmid = {10799478},
publisher = {J Biol Chem},
title = {{Membrane type 4 matrix metalloproteinase (MMP17) has tumor necrosis factor-$\alpha$ convertase activity but does not activate pro-MMP2}},
url = {https://pubmed.ncbi.nlm.nih.gov/10799478/},
volume = {275},
year = {2000}
}
@article{Farfan2013,
abstract = {Synopsis: SNX17 is an early endosomal protein that allows efficient recycling of membrane receptors, such as LRP1 and LDLR, containing an NPxY motif in the cytoplasmic domain. When expressed in polarized epithelial cells, both receptors distribute basolaterally and recycle through the basolateral sorting endosome (BSE). This study describes in detail the SNX17-binding domain present in the LRP1's cytoplasmic domain and demonstrates that in polarized epithelial cells the function of SNX17 is restricted to the basolateral cargo recycling through the BSE. SNX17 is an early endosomal protein that allows efficient recycling of membrane receptors, such as LRP1 and LDLR, containing an NPxY motif in the cytoplasmic domain. When expressed in polarized epithelial cells, both receptors distribute basolaterally and recycle through the basolateral sorting endosome (BSE). This study describes in detail the SNX17-binding domain present in the LRP1's cytoplasmic domain and demonstrates that in polarized epithelial cells the function of SNX17 is restricted to the basolateral cargo recycling through the BSE. Sorting nexin 17 (SNX17) is an adaptor protein present in early endosomal antigen 1 (EEA1)-positive sorting endosomes that promotes the efficient recycling of low-density lipoprotein receptor-related protein 1 (LRP1) to the plasma membrane through recognition of the first NPxY motif in the cytoplasmic tail of this receptor. The interaction of LRP1 with SNX17 also regulates the basolateral recycling of the receptor from the basolateral sorting endosome (BSE). In contrast, megalin, which is apically distributed in polarized epithelial cells and localizes poorly to EEA1-positive sorting endosomes, does not interact with SNX17, despite containing three NPxY motifs, indicating that this motif is not sufficient for receptor recognition by SNX17. Here, we identified a cluster of 32 amino acids within the cytoplasmic domain of LRP1 that is both necessary and sufficient for SNX17 binding. To delineate the function of this SNX17-binding domain, we generated chimeric proteins in which the SNX17-binding domain was inserted into the cytoplasmic tail of megalin. This insertion mediated the binding of megalin to SNX17 and modified the cell surface expression and recycling of megalin in non-polarized cells. However, the polarized localization of chimeric megalin was not modified in polarized Madin-Darby canine kidney cells. These results provide evidence regarding the molecular and cellular mechanisms underlying the specificity of SNX17-binding receptors and the restricted function of SNX17 in the BSE. {\textcopyright} 2013 John Wiley {\&} Sons A/S. Published by John Wiley {\&} Sons Ltd.},
author = {Farf{\'{a}}n, Pamela and Lee, Jiyeon and Larios, Jorge and Sotelo, Pablo and Bu, Guojun and Marzolo, Mar{\'{i}}a Paz},
doi = {10.1111/tra.12076},
file = {::},
issn = {13989219},
journal = {Traffic},
keywords = {LRP1,Megalin,Polarized cells,Recycling endosome,SNX17,Sorting endosome},
month = {jul},
number = {7},
pages = {823--838},
title = {{A Sorting Nexin 17-Binding Domain Within the LRP1 Cytoplasmic Tail Mediates Receptor Recycling Through the Basolateral Sorting Endosome}},
volume = {14},
year = {2013}
}
@article{Fedoseienko2018,
abstract = {Rationale: COMMD (copper metabolism MURR1 domain)-containing proteins are a part of the CCC (COMMD-CCDC22 [coiled-coil domain containing 22]-CCDC93 [coiled-coil domain containing 93]) complex facilitating endosomal traffcking of cell surface receptors. Hepatic COMMD1 inactivation decreases CCDC22 and CCDC93 protein levels, impairs the recycling of the LDLR (low-density lipoprotein receptor), and increases plasma lowdensity lipoprotein cholesterol levels in mice. However, whether any of the other COMMD members function similarly as COMMD1 and whether perturbation in the CCC complex promotes atherogenesis remain unclear. Objective: The main aim of this study is to unravel the contribution of evolutionarily conserved COMMD proteins to plasma lipoprotein levels and atherogenesis. Methods and Results: Using liver-specifc Commd1, Commd6, or Commd9 knockout mice, we investigated the relation between the COMMD proteins in the regulation of plasma cholesterol levels. Combining biochemical and quantitative targeted proteomic approaches, we found that hepatic COMMD1, COMMD6, or COMMD9 defciency resulted in massive reduction in the protein levels of all 10 COMMDs. This decrease in COMMD protein levels coincided with destabilizing of the core (CCDC22, CCDC93, and chromosome 16 open reading frame 62 [C16orf62]) of the CCC complex, reduced cell surface levels of LDLR and LRP1 (LDLR-related protein 1), followed by increased plasma low-density lipoprotein cholesterol levels. To assess the direct contribution of the CCC core in the regulation of plasma cholesterol levels, Ccdc22 was deleted in mouse livers via CRISPR/Cas9-mediated somatic gene editing. CCDC22 defciency also destabilized the complete CCC complex and resulted in elevated plasma low-density lipoprotein cholesterol levels. Finally, we found that hepatic disruption of the CCC complex exacerbates dyslipidemia and atherosclerosis in ApoE3∗Leiden mice. Conclusions: Collectively, these fndings demonstrate a strong interrelationship between COMMD proteins and the core of the CCC complex in endosomal LDLR traffcking. Hepatic disruption of either of these CCC components causes hypercholesterolemia and exacerbates atherosclerosis. Our results indicate that not only COMMD1 but all other COMMDs and CCC components may be potential targets for modulating plasma lipid levels in humans.},
author = {Fedoseienko, Alina and Wijers, Melinde and Wolters, Justina C. and Dekker, Daphne and Smit, Marieke and Huijkman, Nicolette and Kloosterhuis, Niels and Klug, Helene and Schepers, Aloys and {Van Dijk}, Ko Willems and Levels, Johannes H.M. and Billadeau, Daniel D. and Hofker, Marten H. and {Van Deursen}, Jan and Westerterp, Marit and Burstein, Ezra and Kuivenhoven, Jan Albert and {Van De Sluis}, Bart},
doi = {10.1161/CIRCRESAHA.117.312004},
file = {::},
issn = {15244571},
journal = {Circulation Research},
keywords = {Atherosclerosis,Endosome,Hypercholesterolemia,Liver,Mice transgenic},
number = {12},
pages = {1648--1660},
publisher = {Lippincott Williams and Wilkins},
title = {{The COMMD family regulates plasma LDL levels and attenuates atherosclerosis through stabilizing the CCC complex in endosomal LDLR traffcking}},
volume = {122},
year = {2018}
}
@article{Follett2019,
abstract = {DNAJC13 (RME-8)is a core co-chaperone that facilitates membrane recycling and cargo sorting of endocytosed proteins. DNAJ/Hsp40 (heat shock protein 40)proteins are highly conserved throughout evolution and mediate the folding of nascent proteins, and the unfolding, refolding or degradation of misfolded proteins while assisting in associated-membrane translocation. DNAJC13 is one of five DNAJ ‘C' class chaperone variants implicated in monogenic parkinsonism. Here we examine the effect of the DNAJC13 disease-linked mutation (p.Asn855Ser)on its interacting partners, focusing on sorting nexin 1 (SNX1)membrane dynamics in primary cortical neurons derived from a novel Dnajc13 p.Asn855Ser knock-in (DKI)mouse model. Dnajc13 p.Asn855Ser mutant and wild type protein expression were equivalent in mature heterozygous cultures (DIV21). While SNX1-positive puncta density, area, and WASH-retromer assembly were comparable between cultures derived from DKI and wild type littermates, the formation of SNX1-enriched tubules in DKI neuronal cultures was significantly increased. Thus, Dnajc13 p.Asn855Ser disrupts SNX1 membrane-tubulation and trafficking, analogous to results from RME-8 depletion studies. The data suggest the mutation confers a dominant-negative gain-of-function in RME-8. Implications for the pathogenesis of Parkinson's disease are discussed.},
author = {Follett, Jordan and Fox, Jesse D. and Gustavsson, Emil K. and Kadgien, Chelsie and Munsie, Lise N. and Cao, Li Ping and Tatarnikov, Igor and Milnerwood, Austen J. and Farrer, Matthew J.},
doi = {10.1016/j.neulet.2019.04.043},
issn = {18727972},
journal = {Neuroscience Letters},
keywords = {DNAJC13,Parkinson's disease,RME-8,Retromer,SNX1},
month = {jul},
pages = {114--122},
pmid = {31082451},
publisher = {Elsevier Ireland Ltd},
title = {{DNAJC13 p.Asn855Ser, implicated in familial parkinsonism, alters membrane dynamics of sorting nexin 1}},
volume = {706},
year = {2019}
}
@article{Garcia-Huerta2016,
abstract = {Proteins along the secretory pathway are co-translationally translocated into the lumen of the endoplasmic reticulum (ER) as unfolded polypeptide chains. Afterwards, they are usually modified with N-linked glycans, correctly folded and stabilized by disulfide bonds. ER chaperones and folding enzymes control these processes. The accumulation of unfolded proteins in the ER activates a signaling response, termed the unfolded protein response (UPR). The hallmark of this response is the coordinated transcriptional up-regulation of ER chaperones and folding enzymes. In order to discuss the importance of the proper folding of certain substrates we will address the role of ER chaperones in normal physiological conditions and examine different aspects of its contribution in neurodegenerative disease. This article is part of a Special Issue entitled SI:ER stress.},
author = {Garcia-Huerta, Paula and Bargsted, Leslie and Rivas, Alexis and Matus, Soledad and Vidal, Rene L},
doi = {10.1016/j.brainres.2016.04.070},
file = {::},
issn = {1872-6240},
journal = {Brain research},
keywords = {ER chaperones,ER stress,Neurodegenerative disease,Protein aggregation,UPR},
month = {oct},
number = {Pt B},
pages = {580--587},
pmid = {27134034},
publisher = {Elsevier B.V.},
title = {{ER chaperones in neurodegenerative disease: Folding and beyond.}},
url = {http://www.ncbi.nlm.nih.gov/pubmed/27134034},
volume = {1648},
year = {2016}
}
@article{Geladaki2019,
abstract = {The study of protein localisation has greatly benefited from high-throughput methods utilising cellular fractionation and proteomic profiling. Hyperplexed Localisation of Organelle Proteins by Isotope Tagging (hyperLOPIT) is a well-established method in this area. It achieves high-resolution separation of organelles and subcellular compartments but is relatively time- and resource-intensive. As a simpler alternative, we here develop Localisation of Organelle Proteins by Isotope Tagging after Differential ultraCentrifugation (LOPIT-DC) and compare this method to the density gradient-based hyperLOPIT approach. We confirm that high-resolution maps can be obtained using differential centrifugation down to the suborganellar and protein complex level. HyperLOPIT and LOPIT-DC yield highly similar results, facilitating the identification of isoform-specific localisations and high-confidence localisation assignment for proteins in suborganellar structures, protein complexes and signalling pathways. By combining both approaches, we present a comprehensive high-resolution dataset of human protein localisations and deliver a flexible set of protocols for subcellular proteomics.},
author = {Geladaki, Aikaterini and {Ko{\v{c}}evar Britov{\v{s}}ek}, Nina and Breckels, Lisa M. and Smith, Tom S. and Vennard, Owen L. and Mulvey, Claire M. and Crook, Oliver M. and Gatto, Laurent and Lilley, Kathryn S.},
doi = {10.1038/s41467-018-08191-w},
issn = {20411723},
journal = {Nature Communications},
title = {{Combining LOPIT with differential ultracentrifugation for high-resolution spatial proteomics}},
year = {2019}
}
@article{Giurgiu2019,
abstract = {CORUM is a database that provides a manually curated repository of experimentally characterized protein complexes from mammalian organisms, mainly human (67{\%}), mouse (15{\%}) and rat (10{\%}). Given the vital functions of these macromolecular machines, their identification and functional characterization is foundational to our understanding of normal and disease biology. The new CORUM 3.0 release encompasses 4274 protein complexes offering the largest and most comprehensive publicly available dataset of mammalian protein complexes. The CORUM dataset is built from 4473 different genes, representing 22{\%} of the protein coding genes in humans. Protein complexes are described by a protein complex name, subunit composition, cellular functions as well as the literature references. Information about stoichiometry of subunits depends on availability of experimental data. Recent developments include a graphical tool displaying known interactions between subunits. This allows the prediction of structural interconnections within protein complexes of unknown structure. In addition, we present a set of 58 protein complexes with alternatively spliced subunits. Those were found to affect cellular functions such as regulation of apoptotic activity, protein complex assembly or define cellular localization. CORUM is freely accessible at http://mips.helmholtz-muenchen.de/corum/.},
author = {Giurgiu, Madalina and Reinhard, Julian and Brauner, Barbara and Dunger-Kaltenbach, Irmtraud and Fobo, Gisela and Frishman, Goar and Montrone, Corinna and Ruepp, Andreas},
doi = {10.1093/nar/gky973},
issn = {13624962},
journal = {Nucleic Acids Research},
pmid = {30357367},
title = {{CORUM: The comprehensive resource of mammalian protein complexes - 2019}},
year = {2019}
}
@article{Glascock2011,
abstract = {Despite the protective role that blood brain barrier plays in shielding the brain, it limits the access to the central nervous system (CNS) which most often results in failure of potential therapeutics designed for neurodegenerative disorders. Neurodegenerative diseases such as Spinal Muscular Atrophy (SMA), in which the lower motor neurons are affected, can benefit greatly from introducing the therapeutic agents into the CNS. The purpose of this video is to demonstrate two different injection paradigms to deliver therapeutic materials into neonatal mice soon after birth. One of these methods is injecting directly into cerebral lateral ventricles (Intracerebroventricular) which results in delivery of materials into the CNS through the cerebrospinal fluid. The second method is a temporal vein injection (intravenous) that can introduce different therapeutics into the circulatory system, leading to systemic delivery including the CNS. Widespread transduction of the CNS is achievable if an appropriate viral vector and viral serotype is utilized. Visualization and utilization of the temporal vein for injection is feasible up to postnatal day 6. However, if the delivered material is intended to reach the CNS, these injections should take place while the blood brain barrier is more permeable due to its immature status, preferably prior to postnatal day 2. The fully developed blood brain barrier greatly limits the effectiveness of intravenous delivery. Both delivery systems are simple and effective once the surgical aptitude is achieved. They do not require any extensive surgical devices and can be performed by a single person. However, these techniques are not without challenges. The small size of postnatal day 2 pups and the subsequent small target areas can make the injections difficult to perform and initially challenging to replicate. {\textcopyright} 2011 Journal of Visualized Experiments.},
author = {Glascock, Jacqueline J. and Osman, Erkan Y. and Coady, Tristan H. and Rose, Ferrill F. and Shababi, Monir and Lorson, Christian L.},
doi = {10.3791/2968},
file = {::},
issn = {1940087X},
journal = {Journal of Visualized Experiments},
keywords = {Brain,CNS,Injection,Issue 56,Medicine,Motor neuron,Mouse,Neuroscience,Temporal vein,Ventricles},
number = {56},
pmid = {21988897},
publisher = {Journal of Visualized Experiments},
title = {{Delivery of therapeutic agents through intracerebroventricular (ICV) and intravenous (IV) injection in mice}},
year = {2011}
}
@article{Gomez2009,
abstract = {The Arp2/3 complex regulates endocytosis, sorting, and trafficking, yet the Arp2/3-stimulating factors orchestrating these distinct events remain ill defined. WASH (Wiskott-Aldrich Syndrome Protein and SCAR Homolog) is an Arp2/3 activator with unknown function that was duplicated during primate evolution. We demonstrate that WASH associates with tubulin and localizes to early endosomal subdomains, which are enriched in Arp2/3, F-actin, and retromer components. Although WASH localized with activated receptors, it was not essential for endocytosis. However, WASH did regulate retromer-mediated retrograde CI-MPR trafficking, which required its association with endosomes, Arp2/3-directed F-actin regulation, and tubulin interaction. Moreover, WASH exists in a multiprotein complex containing FAM21, which links WASH to endosomes and is required for WASH-dependent retromer-mediated sorting. Significantly, without WASH, retromer tubulation was exaggerated, supporting a model wherein WASH links retromer-mediated cargo containing tubules to microtubules for Golgi-directed trafficking and generates F-actin-driven force for tubule scission. {\textcopyright} 2009 Elsevier Inc. All rights reserved.},
author = {Gomez, Timothy S. and Billadeau, Daniel D.},
doi = {10.1016/j.devcel.2009.09.009},
file = {::},
issn = {15345807},
journal = {Developmental Cell},
keywords = {CELLBIO},
month = {nov},
number = {5},
pages = {699--711},
title = {{A FAM21-Containing WASH Complex Regulates Retromer-Dependent Sorting}},
volume = {17},
year = {2009}
}
@article{Gomez2012,
abstract = {The Arp2/3-activator Wiskott-Aldrich syndrome protein and Scar homologue (WASH) is suggested to regulate actin-dependent membrane scission during endosomal sorting, but its cellular roles have not been fully elucidated. To investigate WASH function, we generated tamoxifen-inducible WASH-knockout mouse embryonic fibroblasts (WASHout MEFs). Of interest, although EEA1(+) endosomes were enlarged, collapsed, and devoid of filamentous-actin and Arp2/3 in WASHout MEFs, we did not observe elongated membrane tubules emanating from these disorganized endomembranes. However, collapsed WASHout endosomes harbored segregated subdomains, containing either retromer cargo recognition complex-associated proteins or EEA1. In addition, we observed global collapse of LAMP1(+) lysosomes, with some lysosomal membrane domains associated with endosomes. Both epidermal growth factor receptor (EGFR) and transferrin receptor (TfnR) exhibited changes in steady-state cellular localization. EGFR was directed to the lysosomal compartment and exhibited reduced basal levels in WASHout MEFs. However, although TfnR was accumulated with collapsed endosomes, it recycled normally. Moreover, EGF stimulation led to efficient EGFR degradation within enlarged lysosomal structures. These results are consistent with the idea that discrete receptors differentially traffic via WASH-dependent and WASH-independent mechanisms and demonstrate that WASH-mediated F-actin is requisite for the integrity of both endosomal and lysosomal networks in mammalian cells.},
author = {Gomez, Timothy S. and Gorman, Jacquelyn A. and de Narvajas, Amaia Artal Martinez and Koenig, Alexander O. and Billadeau, Daniel D.},
doi = {10.1091/mbc.e12-02-0101},
file = {::},
issn = {19394586},
journal = {Molecular biology of the cell},
number = {16},
pages = {3215--3228},
publisher = {American Society for Cell Biology},
title = {{Trafficking defects in WASH-knockout fibroblasts originate from collapsed endosomal and lysosomal networks.}},
volume = {23},
year = {2012}
}
@article{Goosens2001,
abstract = {A large body of literature implicates the amygdala in Pavlovian fear conditioning. In this study, we examined the contribution of individual amygdaloid nuclei to contextual and auditory fear conditioning in rats. Prior to fear conditioning, rats received a large electrolytic lesion of the amygdala in one hemisphere, and a nucleus-specific neurotoxic lesion in the contralateral hemisphere. Neurotoxic lesions targeted either the lateral nucleus (LA), basolateral and basomedial nuclei (basal nuclei), or central nucleus (CE) of the amygdala. LA and CE lesions attenuated freezing to both contextual and auditory conditional stimuli (CSs). Lesions of the basal nuclei produced deficits in contextual and auditory fear conditioning only when the damage extended into the anterior divisions of the basal nuclei; damage limited to the posterior divisions of the basal nuclei did not significantly impair conditioning to either auditory or contextual CS. These effects were typically not lateralized, although neurotoxic lesions of the posterior divisions of the basal nuclei had greater effects on contextual fear conditioning when the contralateral electrolytic lesion was placed in the right hemisphere. These results indicate that there is significant overlap within the amygdala in the neural pathways mediating fear conditioning to contextual and acoustic CS, and that these forms of learning are not anatomically dissociable at the level of amygdaloid nuclei.},
author = {Goosens, Ki A. and Maren, Stephen},
doi = {10.1101/lm.37601},
file = {::},
issn = {10720502},
journal = {Learning and Memory},
number = {3},
pages = {148--155},
pmid = {11390634},
title = {{Contextual and auditory fear conditioning are mediated by the lateral, basal, and central amygdaloid nuclei in rats}},
url = {http://www.ncbi.nlm.nih.gov/pubmed/11390634 http://www.pubmedcentral.nih.gov/articlerender.fcgi?artid=PMC311374},
volume = {8},
year = {2001}
}
@article{Halff2019,
abstract = {GABAA receptors mediate fast inhibitory transmission in the brain, and their number can be rapidly up- or downregulated to alter synaptic strength. Neuroligin-2 plays a critical role in the stabilization of synaptic GABAA receptors and the development and maintenance of inhibitory synapses. To date, little is known about how the amount of neuroligin-2 at the synapse is regulated and whether neuroligin-2 trafficking affects inhibitory signaling. Here, we show that neuroligin-2, when internalized to endosomes, co-localizes with SNX27, a brain-enriched cargo-adaptor protein that facilitates membrane protein recycling. Direct interaction between the PDZ domain of SNX27 and PDZ-binding motif in neuroligin-2 enables membrane retrieval of neuroligin-2, thus enhancing synaptic neuroligin-2 clusters. Furthermore, SNX27 knockdown has the opposite effect. SNX27-mediated up- and downregulation of neuroligin-2 surface levels affects inhibitory synapse composition and signaling strength. Taken together, we show a role for SNX27-mediated recycling of neuroligin-2 in maintenance and signaling of the GABAergic synapse.},
author = {Halff, Els F. and Szulc, Blanka R. and Lesept, Flavie and Kittler, Josef T.},
doi = {10.1016/j.celrep.2019.10.096},
file = {::},
issn = {22111247},
journal = {Cell Reports},
keywords = {E/I balance,GABA,NLGN2,SNX27,endocytosis,inhibitory signaling,neuroligin,neuronal plasticity,retromer},
month = {nov},
number = {9},
pages = {2599--2607.e6},
publisher = {Elsevier B.V.},
title = {{SNX27-Mediated Recycling of Neuroligin-2 Regulates Inhibitory Signaling}},
volume = {29},
year = {2019}
}
@article{Hallett2008,
abstract = {An important tool for studying the regulation of synapses is a rapid and reliable means of separating synaptic and intracellular proteins. This unit presents a technique for analysis of brain tissue which relies on differential centrifugation to separate proteins present at synaptic sites from those found in intracellular cytoplasmic and vesicular pools. The method is efficient in that only small amounts of tissue, such as might be obtained from a small region of a rodent brain, are required. It is reproducible and, in conjunction with immunoblot or immunoprecipitation techniques, can produce reliable quantitative data. The protocol will be of interest to those conducting a variety of different studies related to the localization and trafficking of brain receptors and signaling molecules. {\textcopyright} 2008 by John Wiley {\&} Sons, Inc.},
author = {Hallett, Penelope J. and Collins, Tiffany L. and Standaert, David G. and Dunah, Anthone W.},
doi = {10.1002/0471142301.ns0116s42},
issn = {19348584},
journal = {Current Protocols in Neuroscience},
keywords = {Immunoprecipitation,Phosphorylation,Receptor,Subcellular,Trafficking},
number = {SUPPL. 42},
title = {{Biochemical fractionation of brain tissue for studies of receptor distribution and trafficking}},
year = {2008}
}
@article{Harbour2012,
abstract = {The retromer complex is a conserved endosomal protein sorting complex that sorts membrane proteins into nascent endosomal tubules. The recognition of membrane proteins is mediated by the cargo-selective retromer complex, a stable trimer of the Vps35 (vacuolar protein sorting 35), Vps29 and Vps26 proteins. We have recently reported that the cargo-selective retromer complex associates with the WASH (Wiskott-Aldrich syndrome homologue) complex, a multimeric protein complex that regulates tubule dynamics at endosomes. In the present study, we show that the retromer-WASH complex interaction occurs through the long unstructured 'tail' domain of the WASH complex-Fam21 protein binding to Vps35, an interaction that is necessary and sufficient to target the WASH complex to endosomes. The Fam21-tail also binds to FKBP15 (FK506- binding protein 15), a protein associated with ulcerative colitis, to mediate the membrane association of FKBP15. Elevated Fam21- tail expression inhibits the association of the WASH complex with retromer, resulting in increased cytoplasmicWASHcomplex. Additionally, overexpression of the Fam21-tail results in cellspreading defects, implicating the activity of theWASH complex in regulating the mobilization of membrane into the endosome-to-cell surface pathway. {\textcopyright} The Authors Journal compilation {\textcopyright} 2012 Biochemical Society.},
author = {Harbour, Michael E. and Breusegem, Sophia Y. and Seaman, Matthew N.J.},
doi = {10.1042/BJ20111761},
file = {::},
issn = {02646021},
journal = {Biochemical Journal},
keywords = {Endosome,FK506-binding protein 15 (FKBP15),Retromer,Sorting,Tubule,Wiskott-Aldrich syndrome homologue (WASH) complex},
month = {feb},
number = {1},
pages = {209--220},
title = {{Recruitment of the endosomal WASH complex is mediated by the extended 'tail' of Fam21 binding to the retromer protein Vps35}},
volume = {442},
year = {2012}
}
@article{Harold2009,
abstract = {We undertook a two-stage genome-wide association study (GWAS) of Alzheimer's disease (AD) involving over 16,000 individuals, the most powerful AD GWAS to date. In stage 1 (3,941 cases and 7,848 controls), we replicated the established association with the apolipoprotein E (APOE) locus (most significant SNP, rs2075650, P = 1.8 × 10 157) and observed genome-wide significant association with SNPs at two loci not previously associated with the disease: at the CLU (also known as APOJ) gene (rs11136000, P = 1.4 × 10 9) and 5′ to the PICALM gene (rs3851179, P = 1.9 × 10 8). These associations were replicated in stage 2 (2,023 cases and 2,340 controls), producing compelling evidence for association with Alzheimer's disease in the combined dataset (rs11136000, P = 8.5 × 10 10, odds ratio = 0.86; rs3851179, P = 1.3 × 10 9, odds ratio = 0.86). {\textcopyright} 2009 Nature America, Inc. All rights reserved.},
author = {Harold, Denise and Abraham, Richard and Hollingworth, Paul and Sims, Rebecca and Gerrish, Amy and Hamshere, Marian L. and Pahwa, Jaspreet Singh and Moskvina, Valentina and Dowzell, Kimberley and Williams, Amy and Jones, Nicola and Thomas, Charlene and Stretton, Alexandra and Morgan, Angharad R. and Lovestone, Simon and Powell, John and Proitsi, Petroula and Lupton, Michelle K. and Brayne, Carol and Rubinsztein, David C. and Gill, Michael and Lawlor, Brian and Lynch, Aoibhinn and Morgan, Kevin and Brown, Kristelle S. and Passmore, Peter A. and Craig, David and McGuinness, Bernadette and Todd, Stephen and Holmes, Clive and Mann, David and Smith, A. David and Love, Seth and Kehoe, Patrick G. and Hardy, John and Mead, Simon and Fox, Nick and Rossor, Martin and Collinge, John and Maier, Wolfgang and Jessen, Frank and Sch{\"{u}}rmann, Britta and Heun, Reinhard and {Van Den Bussche}, Hendrik and Heuser, Isabella and Kornhuber, Johannes and Wiltfang, Jens and Dichgans, Martin and Fr{\"{o}}lich, Lutz and Hampel, Harald and H{\"{u}}ll, Michael and Rujescu, Dan and Goate, Alison M. and Kauwe, John S.K. and Cruchaga, Carlos and Nowotny, Petra and Morris, John C. and Mayo, Kevin and Sleegers, Kristel and Bettens, Karolien and Engelborghs, Sebastiaan and {De Deyn}, Peter P. and Livingston, Gill and Bass, Nicholas J. and Gurling, Hugh and McQuillin, Andrew and Gwilliam, Rhian and Deloukas, Panagiotis and Al-Chalabi, Ammar and Shaw, Christopher E. and Tsolaki, Magda and Singleton, Andrew B. and Guerreiro, Rita and M{\"{u}}hleisen, Thomas W. and N{\"{o}}then, Markus M. and Moebus, Susanne and J{\"{o}}ckel, Karl Heinz and Klopp, Norman and Wichmann, H. Erich and Carrasquillo, Minerva M. and Pankratz, V. Shane and Younkin, Steven G. and O'Donovan, Michael and Owen, Michael J. and Williams, Julie},
doi = {10.1038/ng.440},
file = {::},
issn = {15461718},
journal = {Nature Genetics},
month = {oct},
number = {10},
pages = {1088--1093},
pmid = {19734902},
publisher = {Nature Publishing Group},
title = {{Genome-wide association study identifies variants at CLU and PICALM associated with Alzheimer's disease}},
volume = {41},
year = {2009}
}
@misc{Hetz2017,
abstract = {The clinical manifestation of neurodegenerative diseases is initiated by the selective alteration in the functionality of distinct neuronal populations. The pathology of many neurodegenerative diseases includes accumulation of misfolded proteins in the brain. In physiological conditions, the proteostasis network maintains normal protein folding, trafficking and degradation; alterations in this network-particularly disturbances to the function of endoplasmic reticulum (ER)-are thought to contribute to abnormal protein aggregation. ER stress triggers a signalling reaction known as the unfolded protein response (UPR), which induces adaptive programmes that improve protein folding and promote quality control mechanisms and degradative pathways or can activate apoptosis when damage is irreversible. In this Review, we discuss the latest advances in defining the functional contribution of ER stress to brain diseases, including novel evidence that relates the UPR to synaptic function, which has implications for cognition and memory. A complex concept is emerging wherein the consequences of ER stress can differ drastically depending on the disease context and the UPR signalling pathway that is altered. Strategies to target specific components of the UPR using small molecules and gene therapy are in development, and promise interesting avenues for future interventions to delay or stop neurodegeneration.},
author = {Hetz, Claudio and Saxena, Smita},
booktitle = {Nature Reviews Neurology},
doi = {10.1038/nrneurol.2017.99},
file = {::},
issn = {17594766},
number = {8},
pages = {477--491},
title = {{ER stress and the unfolded protein response in neurodegeneration}},
url = {www.nature.com/nrneurol},
volume = {13},
year = {2017}
}
@misc{Hohn2013,
abstract = {Oxidative stress plays a crucial role in the development of the aging process and age dependent diseases. Both are closely connected to disturbances of proteostasis by protein oxidation and an impairment of the proteasomal system. The final consequence is the accumulation of highly cross-linked undegradable aggregates such as lipofuscin. These aggregates of damaged proteins are detrimental to normal cell functions. Here we provide an overview about effect of these aggregates on the proteasomal system, followed by transcription factor activation and loss of cell viability. Furthermore, findings on the mechanism of radical genesis, proteasomal inhibition and the required components of lipofuscin formation were resumed. {\textcopyright} 2013 The Authors.},
author = {H{\"{o}}hn, Annika and Grune, Tilman},
booktitle = {Redox Biology},
doi = {10.1016/j.redox.2013.01.006},
file = {::},
issn = {22132317},
keywords = {Aging,Autophagy,Lipofuscin,Oxidative stress,Proteasome},
number = {1},
pages = {140--144},
publisher = {Elsevier B.V.},
title = {{Lipofuscin: Formation, effects and role of macroautophagy}},
volume = {1},
year = {2013}
}
@article{Hopwood1993,
abstract = {A number of mutations in the X‐chromosomal human iduronate‐2‐sulphatase gene have now been identified as the primary genetic defect leading to the clinical condition known as Hunter syndrome or mucopolysaccharidosis type II. The mutations that are tabulated include different deletions, splice‐site and point mutations. From the group of 319 patients thus far studied by Southern analysis, 14 have a full deletion of the gene and 48 have a partial deletion or other gross rearrangements. All patients with full deletions or gross rearrangements have severe clinical presentations. Twenty‐nine different “small” mutations have so far been characterised in a total of 32 patients. These include 4 nonsense and 13 missense mutations, 7 different small deletions from 1 to 3 bp, with most leading to a frameshift and premature chain termination, and 5 different splice‐site mutations also leading to small insertions or deletions in the mRNA. A 60 bp deletion, that results from a new donor splice‐site, has been observed in five unrelated patients with relatively mild clinical phenotypes. This information will not only be useful for MPS II patient and carrier diagnosis, but also will aid in the understanding of the structure and function of iduronate‐2‐sulphatase, and possibly in correlating genotype with phenotype. {\textcopyright} 1993 Wiley‐Liss, Inc. Copyright {\textcopyright} 1993 Wiley‐Liss, Inc., A Wiley Company},
author = {Hopwood, J. J. and Bunge, S. and Morris, C. P. and Wilson, P. J. and Steglich, C. and Beck, M. and Schwinger, E. and Gal, A.},
doi = {10.1002/humu.1380020603},
issn = {10981004},
journal = {Human Mutation},
keywords = {Hunter syndrome,Iduronate‐2‐Sulphatase (IDS),Mucopolysaccharidosis type II (MPS II),Mutation analysis},
number = {6},
pages = {435--442},
publisher = {Hum Mutat},
title = {{Molecular basis of mucopolysaccharidosis type II: Mutations in the iduronate‐2‐sulphatase gene}},
url = {https://pubmed.ncbi.nlm.nih.gov/8111411/},
volume = {2},
year = {1993}
}
@article{Jahic2015,
abstract = {Background: The hereditary spastic paraplegias (HSPs) are rare neurodegenerative gait disorders which are genetically highly heterogeneous. For each single form, eventual consideration of therapeutic strategies requires an understanding of the mechanism by which mutations confer pathogenicity. SPG8 is a dominantly inherited HSP, and associated with rather early onset and rapid progression. A total of nine mutations in KIAA0196, which encodes the WASH regulatory complex (SHRC) member strumpellin, have been reported in SPG8 patients so far. Based on biochemical and cell biological approaches, they have been suggested to act via loss of function-mediated haploinsufficiency. Methods: We generated a deletion-based knockout allele for E430025E21Rik, i.e. the murine homologue of KIAA0196. The consequences on mRNA and protein levels were analyzed by qPCR and Western-blotting, respectively. Motor performance was evaluated by the foot-base angle paradigm. Axon outgrowth and relevant organelle compartments were investigated in primary neuron cultures and primary fibroblast cultures, respectively. A homemade multiplex ligation-dependent probe amplification assay enabling identification of large inactivating KIAA0196 deletion alleles was applied to DNA from 240 HSP index patients. Results: Homozygous but not heterozygous mice showed early embryonic lethality. No transcripts from the knockout allele were detected, and the previously suggested compensation by the wild-type allele upon heterozygosity was disproven. mRNA expression of genes encoding other SHRC members was unaltered, while there was evidence for reduced SHRC abundance at protein level. We did, however, neither observe HSP-related in vivo and ex vivo phenotypes, nor alterations affecting endosomal, lysosomal, or autophagic compartments. KIAA0196 copy number screening excluded large inactivating deletion mutations in HSP patients. The consequences of monoallelic KIAA0196/E430025E21Rik activation thus differ from those observed for dominant HSP genes for which a loss-of-function mechanism is well established. Conclusions: Our data do not support the current view that heterozygous loss of strumpellin/SHRC function leads to haploinsufficiency and, in turn, to HSP. The lethality of homozygous knockout mice, i.e. the effect of complete loss of function, also argues against a dominant negative effect of mutant on wild-type strumpellin in patients. Toxic gain-of-function represents a potential alternative explanation. Confirmation of this therapeutically relevant hypothesis in vivo, however, will require availability of appropriate knockin models.},
author = {Jahic, Amir and Khundadze, Mukhran and Jaenisch, Nadine and Sch{\"{u}}le, Rebecca and Klimpe, Sven and Klebe, Stephan and Frahm, Christiane and Kassubek, Jan and Stevanin, Giovanni and Sch{\"{o}}ls, Ludger and Brice, Alexis and H{\"{u}}bner, Christian A. and Beetz, Christian},
doi = {10.1186/s13023-015-0359-x},
file = {:Users/student/Library/Application Support/Mendeley Desktop/Downloaded/Jahic et al. - 2015 - The spectrum of KIAA0196 variants, and characterization of a murine knockout Implications for the mutational mecha.pdf:pdf},
issn = {17501172},
journal = {Orphanet Journal of Rare Diseases},
keywords = {Hereditary spastic paraplegia,KIAA0196,Knockout,Mouse model,SPG8,Strumpellin},
month = {nov},
number = {1},
pmid = {26572744},
publisher = {BioMed Central Ltd.},
title = {{The spectrum of KIAA0196 variants, and characterization of a murine knockout: Implications for the mutational mechanism in hereditary spastic paraplegia type SPG8 Rare neurological diseases}},
volume = {10},
year = {2015}
}
@article{Jia2010,
abstract = {We recently showed that the Wiskott-Aldrich syndrome protein (WASP) family member,WASH, localizes to endosomal subdomains and regulates endocytic vesicle scission in an Arp2/3-dependent manner. Mechanisms regulating WASH activity are unknown. Here we show that WASH functions in cells within a 500 kDa core complex containing Strumpellin, FAM21, KIAA1033 (SWIP), and CCDC53. Although recombinant WASH is constitutively active toward the Arp2/3 complex, the reconstituted core assembly is inhibited, suggesting that it functions in cells to regulate actin dynamics through WASH. FAM21 interacts directly with CAPZ and inhibits its actin-capping activity. Four of the five core components show distant (approximately 15{\%} amino acid sequence identify) but significant structural homology to components of a complex that negatively regulates the WASP family member, WAVE. Moreover, biochemical and electron microscopic analyses show that the WASH and WAVE complexes are structurally similar. Thus, these two distantly related WASP family members are controlled by analogous structurally related mechanisms. Strumpellin is mutated in the human disease hereditary spastic paraplegia, and its link to WASH suggests that misregulation of actin dynamics on endosomes may play a role in this disorder.},
author = {Jia, Da and Gomez, Timothy S. and Metlagel, Zoltan and Umetani, Junko and Otwinowski, Zbyszek and Rosen, Michael K. and Billadeau, Daniel D.},
doi = {10.1073/pnas.0913293107},
file = {::},
issn = {00278424},
journal = {Proceedings of the National Academy of Sciences of the United States of America},
keywords = {Actin dynamics and capping,Endosome,Hereditary spastic paraplegia,WASH regulatory complex,WAVE regulatory complex},
month = {jun},
number = {23},
pages = {10442--10447},
publisher = {National Academy of Sciences},
title = {{WASH and WAVE actin regulators of the Wiskott-Aldrich syndrome protein (WASP) family are controlled by analogous structurally related complexes}},
url = {/pmc/articles/PMC2890800/?report=abstract https://www.ncbi.nlm.nih.gov/pmc/articles/PMC2890800/},
volume = {107},
year = {2010}
}
@article{Kim2016,
abstract = {The BioID method uses a promiscuous biotin ligase to detect protein-protein associations as well as proximate proteins in living cells. Here we report improvements to the BioID method centered on BioID2, a substantially smaller promiscuous biotin ligase. BioID2 enables more-selective targeting of fusion proteins, requires less biotin supplementation, and exhibits enhanced labeling of proximate proteins. Thus BioID2 improves the efficiency of screening for protein-protein associations. We also demonstrate that the biotinylation range of BioID2 can be considerably modulated using flexible linkers, thus enabling applicationspecific adjustment of the biotin-labeling radius.},
author = {Kim, Dae In and Jensen, Samuel C. and Noble, Kyle A. and Kc, Birendra and Roux, Kenneth H. and Motamedchaboki, Khatereh and Roux, Kyle J.},
doi = {10.1091/mbc.E15-12-0844},
file = {::},
issn = {19394586},
journal = {Molecular Biology of the Cell},
month = {apr},
number = {8},
pages = {1188--1196},
pmid = {26912792},
publisher = {American Society for Cell Biology},
title = {{An improved smaller biotin ligase for BioID proximity labeling}},
volume = {27},
year = {2016}
}
@article{Kim2020,
abstract = {AKT/PKB is downregulated by the ubiquitin-proteasome system (UPS), which plays a key role in cell survival and tumor progression in various types of cancer. The objective of this study was to determine the relationship between the sequential ubiquitination of lysine residues K284 to K214 in AKT and R-HSPA5 (the arginylated form of HSPA5), which contribute to the autophagic/lysosomal degradation of AKT when impaired proteasomal activity induces cellular stress. Results show that proteasome inhibitors (PIs) increased ATE1 (arginyltransferase 1)-mediated R-HSPA5 levels in a reactive oxygen species (ROS)-dependent manner. Further, binding of fully ubiquitinated AKT with R-HSPA5 induced AKT degradation via the autophagy-lysosome pathway. Specifically, the K48 (Lys48)-linked ubiquitinated form of AKT was selectively degraded in the lysosome with R-HSPA5. The deubiquitinase, USP7 (ubiquitin specific peptidase 7), prevented AKT degradation by inhibiting AKT ubiquitination via interaction with AKT. MUL1 (mitochondrial ubiquitin ligase activator of NFKB 1) also played a vital role in the lysosomal degradation of AKT by sequentially ubiquitinating AKT residues K284 to K214 for R-HSPA5-mediated autophagy. Consistent with this finding, despite HSPA5 arginylation, AKT was not degraded in mul1 KO cells. These results suggest that MUL1-mediated sequential ubiquitination of K284 to K214 may serve as a novel mechanism by which AKT is designated for lysosomal degradation. Moreover, binding of R-HSPA5 with fully ubiquitinated AKT is required for the autophagic/lysosomal degradation of AKT. Thus, modulating the MUL1-mediated non-proteasomal proteolysis mechanisms, such as sequential ubiquitination, may prove to be a novel therapeutic approach for cancer treatment. Abbreviations: AKT1: thymoma viral proto-oncogene 1; ATE1: arginyltransferase 1; ATG5: autophagy related 5; CASP3: caspase 3; EGFP: enhanced green fluorescent protein; GAPDH: glyceraldehyde-3-phosphate dehydrogenase; GSK3B; glycogen synthase kinase 3 beta; HA: hemagglutinin; HSPA5/GRP78/BIP: heat shock protein 5; LAMP1: lysosomal-associated membrane protein 1; MAP1LC3B: microtubule-associated protein 1 light chain 3 beta; MEF: mouse embryonic fibroblast; MUL1: mitochondrial ubiquitin ligase activator of NFKB1; NAC: N-acetylcysteine; NEK2: NIMA (never in mitosis gene a)-related expressed kinase 2; NH4Cl: ammonium chloride; PARP1: poly(ADP-ribose) polymerase family, member 1; PI: proteasome inhibitor; R-HSPA5: arginylated HSPA5; ROS: reactive oxygen species; SQSTM1: sequestome 1; Ub: ubiquitin; USP7: ubiquitin specific peptidase 7.},
author = {Kim, Hyo Jeong and Kim, Sun Yong and Kim, Dae Ho and Park, Joon Seong and Jeong, Seong Hyun and Choi, Young Won and Kim, Chul Ho},
doi = {10.1080/15548627.2020.1740529},
issn = {15548635},
journal = {Autophagy},
keywords = {AKT,HSPA5,MUL1,autophagy,lysosome,sequential ubiquitination},
pmid = {32164484},
publisher = {Taylor and Francis Inc.},
title = {{Crosstalk between HSPA5 arginylation and sequential ubiquitination leads to AKT degradation through autophagy flux}},
year = {2020}
}
@article{Kim2003,
abstract = {Synaptic GTPase-activating protein (SynGAP) is a neuronal RasGAP (Ras GTPase-activating protein) that is selectively expressed in brain and highly enriched at excitatory synapses, where it negatively regulates Ras activity and its downstream signaling pathways. To investigate the physiological role of SynGAP in the brain, we have generated mutant mice lacking the SynGAP protein. These mice exhibit postnatal lethality, indicating that SynGAP plays a critical role during neuronal development. In addition, cell biological experiments show that neuronal cultures from mutant mice have more synaptic AMPA receptor clusters, suggesting that SynGAP regulates glutamate receptor synaptic targeting. Moreover, electrophysiological studies demonstrated that heterozygous mutant mice have a specific defect in hippocampal long-term potentiation (LTP). These studies show that the regulation of synaptic Ras signaling by SynGAP is important for proper neuronal development and glutamate receptor trafficking and is critical for the induction of LTP.},
author = {Kim, Jee Hae and Lee, Hey Kyoung and Takamiya, Kogo and Huganir, Richard L.},
doi = {10.1523/jneurosci.23-04-01119.2003},
file = {::},
issn = {02706474},
journal = {Journal of Neuroscience},
keywords = {AMPA receptors,Excitatory synapses,Glutamate,Long-term depression,Long-term potentiation,NMDA receptors,Postsynaptic density,Ras signaling},
month = {feb},
number = {4},
pages = {1119--1124},
pmid = {12598599},
title = {{The role of synaptic GTPase-activating protein in neuronal development and synaptic plasticity}},
volume = {23},
year = {2003}
}
@article{Klaassen2016,
abstract = {Trafficking and biophysical properties of AMPA receptors (AMPARs) in the brain depend on interactions with associated proteins. We identify Shisa6, a single transmembrane protein, as a stable and directly interacting bona fide AMPAR auxiliary subunit. Shisa6 is enriched at hippocampal postsynaptic membranes and co-localizes with AMPARs. The Shisa6 C-terminus harbours a PDZ domain ligand that binds to PSD-95, constraining mobility of AMPARs in the plasma membrane and confining them to postsynaptic densities. Shisa6 expressed in HEK293 cells alters GluA1-and GluA2-mediated currents by prolonging decay times and decreasing the extent of AMPAR desensitization, while slowing the rate of recovery from desensitization. Using gene deletion, we show that Shisa6 increases rise and decay times of hippocampal CA1 miniature excitatory postsynaptic currents (mEPSCs). Shisa6-containing AMPARs show prominent sustained currents, indicating protection from full desensitization. Accordingly, Shisa6 prevents synaptically trapped AMPARs from depression at high-frequency synaptic transmission.},
author = {Klaassen, Remco V. and Stroeder, Jasper and Coussen, Fran{\c{c}}oise and Hafner, Anne Sophie and Petersen, Jennifer D. and Renancio, Cedric and Schmitz, Leanne J.M. and Normand, Elisabeth and Lodder, Johannes C. and Rotaru, Diana C. and Rao-Ruiz, Priyanka and Spijker, Sabine and Mansvelder, Huibert D. and Choquet, Daniel and Smit, August B.},
doi = {10.1038/ncomms10682},
file = {::},
issn = {20411723},
journal = {Nature Communications},
month = {mar},
pmid = {26931375},
publisher = {Nature Publishing Group},
title = {{Shisa6 traps AMPA receptors at postsynaptic sites and prevents their desensitization during synaptic activity}},
volume = {7},
year = {2016}
}
@article{Kustermann2018,
abstract = {VCP/p97 (valosin containing protein) is a key regulator of cellular proteostasis. It orchestrates protein turnover and quality control in vivo, processes fundamental for proper cell function. In humans, mutations in VCP lead to severe myo- and neuro-degenerative disorders such as inclusion body myopathy with Paget disease of the bone and frontotemporal dementia (IBMPFD), amyotrophic lateral sclerosis (ALS) or and hereditary spastic paraplegia (HSP). We analyzed here the in vivo role of Vcp and its novel interactor Washc4/Swip (WASH complex subunit 4) in the vertebrate model zebrafish (Danio rerio). We found that targeted inactivation of either Vcp or Washc4, led to progressive impairment of cardiac and skeletal muscle function, structure and cytoarchitecture without interfering with the differentiation of both organ systems. Notably, loss of Vcp resulted in compromised protein degradation via the proteasome and the macroautophagy/autophagy machinery, whereas Washc4 deficiency did not affect the function of the ubiquitin-proteasome system (UPS) but caused ER stress and interfered with autophagy function in vivo. In summary, our findings provide novel insights into the in vivo functions of Vcp and its novel interactor Washc4 and their particular and distinct roles during proteostasis in striated muscle cells.},
author = {Kustermann, Monika and Manta, Linda and Paone, Christoph and Kustermann, Jochen and Lausser, Ludwig and Wiesner, Cora and Eichinger, Ludwig and Clemen, Christoph S. and Schr{\"{o}}der, Rolf and Kestler, Hans A. and Sandri, Marco and Rottbauer, Wolfgang and Just, Steffen},
doi = {10.1080/15548627.2018.1491491},
file = {::},
issn = {15548635},
journal = {Autophagy},
keywords = {Proteostasis,Vcp,Washc4,striated muscle,zebrafish},
month = {nov},
number = {11},
pages = {1911--1927},
pmid = {30010465},
publisher = {Taylor and Francis Inc.},
title = {{Loss of the novel Vcp (valosin containing protein) interactor Washc4 interferes with autophagy-mediated proteostasis in striated muscle and leads to myopathy in vivo}},
url = {https://www.tandfonline.com/doi/full/10.1080/15548627.2018.1491491},
volume = {14},
year = {2018}
}
@article{Kvainickas2017,
abstract = {The retromer complex, which recycles the cation-independent mannose 6-phosphate receptor (CI-MPR) from endosomes to the trans-Golgi network (TGN), is thought to consist of a cargo-selective VPS26-VPS29-VPS35 trimer and a membrane- deforming subunit of sorting nexin (SNX)-Bin, Amphyphysin, and Rvs (BAR; SNX-BAR) proteins. In this study, we demonstrate that heterodimers of the SNX-BAR proteins, SNX1, SNX2, SNX5, and SNX6, are the cargo-selective elements that mediate the retrograde transport of CI-MPR from endosomes to the TGN independently of the core retromer trimer. Using quantitative proteomics, we also identify the IGF1R, among more potential cargo, as another SNX5 and SNX6 binding receptor that recycles through SNX-BAR heterodimers, but not via the retromer trimer, in a ligand- and activation-dependent manner. Overall, our data redefine the mechanics of retromer-based sorting and call into question whether retromer indeed functions as a complex of SNX-BAR proteins and the VPS26-VPS29-VPS35 trimer.},
author = {Kvainickas, Arunas and Jimenez-Orgaz, Ana and N{\"{a}}gele, Heike and Hu, Zehan and Dengjel, J{\"{o}}rn and Steinberg, Florian},
doi = {10.1083/jcb.201702137},
file = {::},
issn = {15408140},
journal = {Journal of Cell Biology},
month = {nov},
number = {11},
pages = {3677--3693},
publisher = {Rockefeller University Press},
title = {{Cargo-selective SNX-BAR proteins mediate retromer trimer independent retrograde transport}},
volume = {216},
year = {2017}
}
@article{Lee2016,
abstract = {The endosomal network maintains cellular homeostasis by sorting, recycling and degrading endocytosed cargoes. Retromer organizes the endosomal sorting pathway in conjunction with various sorting nexin (SNX) proteins. The SNX27-retromer complex has recently been identified as a major endosomal hub that regulates endosome-to-plasma membrane recycling by preventing lysosomal entry of cargoes. Here, we show that SNX27 directly interacts with FAM21, which also binds retromer, within the Wiskott-Aldrich syndrome protein and SCAR homologue (WASH) complex. This interaction is required for the precise localization of SNX27 at an endosomal subdomain as well as for recycling of SNX27-retromer cargoes. Furthermore, FAM21 prevents cargo transport to the Golgi apparatus by controlling levels of phosphatidylinositol 4-phosphate, which facilitates cargo dissociation at the Golgi. Together, our results demonstrate that the SNX27-retromer-WASH complex directs cargoes to the plasma membrane by blocking their transport to lysosomes and the Golgi.},
author = {Lee, Seongju and Chang, Jaerak and Blackstone, Craig},
doi = {10.1038/ncomms10939},
file = {::},
issn = {20411723},
journal = {Nature Communications},
month = {mar},
publisher = {Nature Publishing Group},
title = {{FAM21 directs SNX27-retromer cargoes to the plasma membrane by preventing transport to the Golgi apparatus}},
volume = {7},
year = {2016}
}
@article{Linardopoulou2007,
abstract = {Subtelomeres are duplication-rich, structurally variable regions of the human genome situated just proximal of telomeres. We report here that the most terminally located human subtelomeric genes encode a previously unrecognized third subclass of the Wiskott-Aldrich Syndrome Protein family, whose known members reorganize the actin cytoskeleton in response to extracellular stimuli. This new subclass, which we call WASH, is evolutionarily conserved in species as diverged as Entamoeba. We demonstrate that WASH is essential in Drosophila. WASH is widely expressed in human tissues, and human WASH protein colocalizes with actin in filopodia and lamellipodia. The VCA domain of human WASH promotes actin polymerization by the Arp2/3 complex in vitro. WASH duplicated to multiple chromosomal ends during primate evolution, with highest copy number reached in humans, whose WASH repertoires vary. Thus, human subtelomeres are not genetic junkyards, and WASH's location in these dynamic regions could have advantageous as well as pathologic consequences. {\textcopyright} 2007 Linardopoulou et al.},
author = {Linardopoulou, Elena V. and Parghi, Sean S. and Friedman, Cynthia and Osborn, Gregory E. and Parkhurst, Susan M. and Trask, Barbara J.},
doi = {10.1371/journal.pgen.0030237},
file = {::},
issn = {15537390},
journal = {PLoS Genetics},
month = {dec},
number = {12},
pages = {2477--2485},
pmid = {18159949},
publisher = {Public Library of Science},
title = {{Human subtelomeric WASH genes encode a new subclass of the WASP family}},
volume = {3},
year = {2007}
}
@article{Lopez2015,
abstract = {HitPredict is a consolidated resource of experimentally identified, physical protein-protein interactions with confidence scores to indicate their reliability. The study of genes and their inter-relationships using methods such as network and pathway analysis requires high quality protein-protein interaction information. Extracting reliable interactions from most of the existing databases is challenging because they either contain only a subset of the available interactions, or a mixture of physical, genetic and predicted interactions. Automated integration of interactions is further complicated by varying levels of accuracy of database content and lack of adherence to standard formats. To address these issues, the latest version of HitPredict provides a manually curated dataset of 398 696 physical associations between 70 808 proteins from 105 species. Manual confirmation was used to resolve all issues encountered during data integration. For improved reliability assessment, this version combines a new score derived from the experimental information of the interactions with the original score based on the features of the interacting proteins. The combined interaction score performs better than either of the individual scores in HitPredict as well as the reliability score of another similar database. HitPredict provides a web interface to search proteins and visualize their interactions, and the data can be downloaded for offline analysis. Data usability has been enhanced by mapping protein identifiers across multiple reference databases. Thus, the latest version of HitPredict provides a significantly larger, more reliable and usable dataset of protein-protein interactions from several species for the study of gene groups.},
author = {L{\'{o}}pez, Yosvany and Nakai, Kenta and Patil, Ashwini},
doi = {10.1093/database/bav117},
issn = {17580463},
journal = {Database},
title = {{HitPredict version 4: Comprehensive reliability scoring of physical protein-protein interactions from more than 100 species}},
year = {2015}
}
@incollection{Mann2015,
abstract = {Among a wide variety of model systems, rodents provide a unique, testable, and translational tool for modeling movement disorders. It is important to first select a model that offers construct validity (i.e., reflects mechanisms underlying pathological features of the disease), then various methods can be used to assess motor dysfunction. Even though motor assessments can provide face validity to models of motor disorders, rodents cannot be expected to faithfully reproduce human deficits. Rather, these motor tests may provide reliable endpoint measures despite the presence of deficits not closely related to those seen in humans. Using multiple motor tests in parallel can help correctly assess motor deficits and assist in interpreting results. The motor tests discussed here assess motor coordination, balance, fine motor skills, general exploration and locomotion, sensorimotor skills, limb bias, grip strength, and gait. Although some tests listed in this chapter are designed for use specifically in either rats or mice, they often can be modified for use in either rodent model.},
author = {Mann, Amandeep and Chesselet, Marie Francoise},
booktitle = {Movement Disorders: Genetics and Models: Second Edition},
doi = {10.1016/B978-0-12-405195-9.00008-1},
isbn = {9780124051959},
keywords = {Balance,Fine motor skills,Motor coordination,Motor tests,Rodent},
month = {jan},
pages = {139--157},
publisher = {Elsevier Inc.},
title = {{Techniques for Motor Assessment in Rodents}},
year = {2015}
}
@article{Mao2015,
abstract = {The Shank genes (SHANK1, 2, 3) encode scaffold proteins highly enriched in postsynaptic densities where they regulate synaptic structure in spiny neurons. Mutations in human Shank genes are linked to autism spectrum disorder and schizophrenia. Shank1 mutant mice exhibit intriguing cognitive phenotypes reminiscent of individuals with autism spectrum disorder. However, the molecular mechanisms leading to the human pathophysiological phenotypes and mouse behaviors have not been elucidated. In this study it is shown that Shank1 protein is highly localized in parvalbumin-expressing (PV+) fast-spiking inhibitory interneurons in the hippocampus. Importantly, a lack of Shank1 in hippocampal CA1 PV+ neurons reduced excitatory synaptic inputs and inhibitory synaptic outputs to pyramidal neurons. Furthermore, it is demonstrated that hippocampal CA1 pyramidal neurons in Shank1 mutant mice exhibit a shift in the excitatory and inhibitory balance (E-I balance), a pathophysiological hallmark of autism spectrum disorder. The mutant mice also exhibit lower expression of gephyrin (a scaffold component of inhibitory synapses), supporting the dysregulation of E-I balance in the hippocampus. These results suggest that Shank1 scaffold in PV+ interneurons regulates excitatory synaptic strength and participates in the maintenance of E-I balance in excitatory neurons.},
archivePrefix = {arXiv},
arxivId = {15334406},
author = {Mao, Wenjie and Watanabe, Takuya and Cho, Sukhee and Frost, Jeffrey L. and Truong, Tina and Zhao, Xiaohu and Futai, Kensuke},
doi = {10.1111/ejn.12877},
eprint = {15334406},
isbn = {1460-9568 (Electronic)$\backslash$r0953-816X (Linking)},
issn = {14609568},
journal = {European Journal of Neuroscience},
keywords = {Excitatory synapse,GABAergic synapse,Hippocampus,Interneurons,Synaptic transmission},
number = {8},
pages = {1025--1035},
pmid = {25816842},
title = {{Shank1 regulates excitatory synaptic transmission in mouse hippocampal parvalbumin-expressing inhibitory interneurons}},
volume = {41},
year = {2015}
}
@article{Maruzs2015,
abstract = {The retromer is an evolutionarily conserved coat complex that consists of Vps26, Vps29, Vps35 and a heterodimer of sorting nexin (Snx) proteins in yeast. Retromer mediates the recycling of transmembrane proteins from endosomes to the trans-Golgi network, including receptors that are essential for the delivery of hydrolytic enzymes to lysosomes. Besides its function in lysosomal enzyme receptor recycling, involvement of retromer has also been proposed in a variety of vesicular trafficking events, including early steps of autophagy and endocytosis. Here we show that the late stages of autophagy and endocytosis are impaired in Vps26 and Vps35 deficient Drosophila larval fat body cells, but formation of autophagosomes and endosomes is not compromised. Accumulation of aberrant autolysosomes and amphisomes in the absence of retromer function appears to be the consequence of decreased degradative capacity, as they contain undigested cytoplasmic material. Accordingly, we show that retromer is required for proper cathepsin L trafficking mainly independent of LERP, the Drosophila homolog of the cation-independent mannose 6-phosphate receptor. Finally, we find that Snx3 and Snx6 are also required for proper autolysosomal degradation in Drosophila larval fat body cells. In this study we characterize the role of the retromer complex in autophagy and endocytosis in Drosophila larval fat body cells.},
author = {Maruzs, Tam{\'{a}}s and Lorincz, P{\'{e}}ter and Szatm{\'{a}}ri, Zsuzsanna and Sz{\'{e}}plaki, Szilvia and S{\'{a}}ndor, Zolt{\'{a}}n and Lakatos, Zsolt and Puska, Gina and Juh{\'{a}}sz, G{\'{a}}bor and Sass, Mikl{\'{o}}s},
doi = {10.1111/tra.12309},
issn = {16000854},
journal = {Traffic},
keywords = {Autophagy,Drosophila,Endocytosis,Lysosomal storage diseases,Lysosome,Retromer},
month = {oct},
number = {10},
pages = {1088--1107},
publisher = {Blackwell Munksgaard},
title = {{Retromer Ensures the Degradation of Autophagic Cargo by Maintaining Lysosome Function in Drosophila}},
volume = {16},
year = {2015}
}
@article{Mason2011,
abstract = {The maintenance of rapid and efficient actin dynamics in vivo requires coordination of filament assembly and disassembly. This regulation requires temporal and spatial integration of signaling pathways by protein complexes. However, it remains unclear how these complexes form and then regulate the actin cytoskeleton. Here, we identify a srGAP2 and forminlike 1 (FMNL1, also known as FRL1 or FRL$\alpha$) complex whose assembly is regulated by Rac signaling. Our data suggest sr-GAP2 regulates FMNL1 in two ways; 1) Rac-mediated activation of FMNL1 leads to the recruitment of srGAP2, which contains a Rac-specific GAP domain; 2) the SH3 domain of srGAP2 binds the formin homology 1 domain of FMNL1 to inhibit FMNL1-mediated actin severing. Thus, srGAP2 can efficiently terminate the upstream activating Rac signal while also opposing an important functional output of FMNL1, namely actin severing. We also show that FMNL1 and srGAP2 localize to the actin-rich phagocytic cup of macrophage-derived cells, suggesting the complex may regulate this Rac- and actin-driven process in vivo. We propose that after Rac-dependent activation of FMNL1, srGAP2 mediates a potent mechanism to limit the duration of Rac action and inhibit formin activity during rapid actin dynamics. {\textcopyright} 2011 by The American Society for Biochemistry and Molecular Biology, Inc.},
author = {Mason, Frank M and Heimsath, Ernest G and Higgs, Henry N and Soderling, Scott H},
doi = {10.1074/jbc.M110.190397},
file = {::},
issn = {00219258},
journal = {Journal of Biological Chemistry},
keywords = {2 The abbreviations used are: GAP,CA,ChFP,DAD,DID,Dia autoregulatory domain,Dia inhibitory domain,FH,FMNL1,GBD,GTPase binding domain,GTPase-activating protein,TIRF,cherry fluorescent protein,constitutively active,formin homology,formin-like 1,total internal reflection fluorescence},
number = {8},
pages = {6577--6586},
pmid = {21148482},
title = {{Bi-modal regulation of a formin by srGAP2}},
url = {http://www.jbc.org/},
volume = {286},
year = {2011}
}
@article{Mayor1993,
abstract = {A central question in the endocytic process concerns the mechanism for sorting of recycling components (such as transferrin or low density lipoprotein receptors) from lysosomally directed components; membrane-associated molecules including receptors are generally directed towards the recycling pathway while the luminal content of sorting endosomes, consisting of the acid-released ligands, are lysosomally targeted. However, it is not known whether recycling membrane receptors follow bulk membrane flow or if these proteins are actively sorted from lysosomally directed material because of specific protein sequences and/or structural features. Using quantitative fluorescence microscopy we have determined the endocytic route and kinetics of traffic of the bulk carrier, membrane lipids, to address this issue directly. We show that N-[N-(7-nitro-2,1,3-benzoxadiazol-4-yl)-epsilon-aminohexanoyl]- sphingosylphosphorylcholine (C6-NBD-SM) in endocytosed as bulk membrane, and it transits the endocytic system kinetically and morphologically identically to fluorescently labeled transferrin in a CHO cell line. With indistinguishable kinetics, the two labeled markers sort from lysosomally destined molecules in peripherally located sorting endosomes, accumulate in a peri-centriolar recycling compartment, and finally exit the cell. Other fluorescently labeled lipids, C6-NBD-phosphatidylcholine and galactosylceramide also traverse the same pathway. The constitutive nature of sorting of bulk membrane towards the recycling pathway and the lysosomal direction of fluid phase implies a geometric basis of sorting.},
author = {Mayor, S. and Presley, J. F. and Maxfield, F. R.},
doi = {10.1083/jcb.121.6.1257},
isbn = {0021-9525},
issn = {00219525},
journal = {Journal of Cell Biology},
pmid = {8509447},
title = {{Sorting of membrane components from endosomes and subsequent recycling to the cell surface occurs by a bulk flow process}},
year = {1993}
}
@article{McCarthy2012,
abstract = {A flexible statistical framework is developed for the analysis of read counts from RNA-Seq gene expression studies. It provides the ability to analyse complex experiments involving multiple treatment conditions and blocking variables while still taking full account of biological variation. Biological variation between RNA samples is estimated separately from the technical variation associated with sequencing technologies. Novel empirical Bayes methods allow each gene to have its own specific variability, even when there are relatively few biological replicates from which to estimate such variability. The pipeline is implemented in the edgeR package of the Bioconductor project. A case study analysis of carcinoma data demonstrates the ability of generalized linear model methods (GLMs) to detect differential expression in a paired design, and even to detect tumour-specific expression changes. The case study demonstrates the need to allow for gene-specific variability, rather than assuming a common dispersion across genes or a fixed relationship between abundance and variability. Genewise dispersions de-prioritize genes with inconsistent results and allow the main analysis to focus on changes that are consistent between biological replicates. Parallel computational approaches are developed to make non-linear model fitting faster and more reliable, making the application of GLMs to genomic data more convenient and practical. Simulations demonstrate the ability of adjusted profile likelihood estimators to return accurate estimators of biological variability in complex situations. When variation is gene-specific, empirical Bayes estimators provide an advantageous compromise between the extremes of assuming common dispersion or separate genewise dispersion. The methods developed here can also be applied to count data arising from DNA-Seq applications, including ChIP-Seq for epigenetic marks and DNA methylation analyses. {\textcopyright} 2011 The Author(s).},
author = {McCarthy, Davis J. and Chen, Yunshun and Smyth, Gordon K.},
doi = {10.1093/nar/gks042},
issn = {03051048},
journal = {Nucleic Acids Research},
title = {{Differential expression analysis of multifactor RNA-Seq experiments with respect to biological variation}},
year = {2012}
}
@article{McNally2017,
abstract = {Following endocytosis into the endosomal network, integral membrane proteins undergo sorting for lysosomal degradation or are retrieved and recycled back to the cell surface. Here we describe the discovery of an ancient and conserved multiprotein complex that orchestrates cargo retrieval and recycling and, importantly, is biochemically and functionally distinct from the established retromer pathway. We have called this complex €retriever'; it is a heterotrimer composed of DSCR3, C16orf62 and VPS29, and bears striking similarity to retromer. We establish that retriever associates with the cargo adaptor sorting nexin 17 (SNX17) and couples to CCC (CCDC93, CCDC22, COMMD) and WASH complexes to prevent lysosomal degradation and promote cell surface recycling of $\alpha$ 5 $\beta$ 1 integrin. Through quantitative proteomic analysis, we identify over 120 cell surface proteins, including numerous integrins, signalling receptors and solute transporters, that require SNX17-retriever to maintain their surface levels. Our identification of retriever establishes a major endosomal retrieval and recycling pathway.},
author = {McNally, Kerrie E. and Faulkner, Rebecca and Steinberg, Florian and Gallon, Matthew and Ghai, Rajesh and Pim, David and Langton, Paul and Pearson, Neil and Danson, Chris M. and N{\"{a}}gele, Heike and Morris, Lindsey L. and Singla, Amika and Overlee, Brittany L. and Heesom, Kate J. and Sessions, Richard and Banks, Lawrence and Collins, Brett M. and Berger, Imre and Billadeau, Daniel D. and Burstein, Ezra and Cullen, Peter J.},
doi = {10.1038/ncb3610},
file = {::},
issn = {14764679},
journal = {Nature Cell Biology},
month = {sep},
number = {10},
pages = {1214--1225},
pmid = {28892079},
publisher = {Nature Publishing Group},
title = {{Retriever is a multiprotein complex for retromer-independent endosomal cargo recycling}},
volume = {19},
year = {2017}
}
@article{Mok2003,
abstract = {Mucopolysaccharidosis type IIID (MPS IIID; Sanfilippo syndrome type D; MIM 252940) is caused by deficiency of the activity of N-acetylglucosamine-6-sulfatase (GNS), which is normally required for degradation of heparan sulfate. The clinical features of MPS IIID include progressive neurodegeneration, with relatively mild somatic symptoms. Biochemical features include accumulation of heparan sulfate and N-acetylglucosamine-6-sulfate in the brain and viscera. To date, diagnosis required a specific lysosomal enzyme assay for GNS activity. From genomic DNA of a subject with MPS IIID, we amplified and sequenced the promoter and 14 exons of GNS. We found a homozygous nonsense mutation in exon 9 (1063C → T), which predicted premature termination of translation (R355X). We also identified two common synonymous coding single-nucleotide polymorphisms and genotyped these in samples from four ethnic groups. This first report of a mutation in GNS resulting in MPS IIID indicates the potential utility of molecular diagnosis for this rare condition. {\textcopyright} 2003 Elsevier Science (USA). All rights reserved.},
author = {Mok, Andrea and Cao, Henian and Hegele, Robert A.},
doi = {10.1016/S0888-7543(02)00014-9},
file = {::},
issn = {08887543},
journal = {Genomics},
keywords = {Inborn errors of metabolism,Lyases,Metabolic disease,Nonsense mutation,Variation (genetics)},
month = {jan},
number = {1},
pages = {1--5},
publisher = {Academic Press Inc.},
title = {{Genomic basis of mucopolysaccharidosis type IIID (MIM 252940) revealed by sequencing of GNS encoding N-acetylglucosamine-6-sulfatase}},
volume = {81},
year = {2003}
}
@article{Monteiro2017,
abstract = {Several large-scale genomic studies have supported an association between cases of autism spectrum disorder and mutations in the genes SH3 and multiple ankyrin repeat domains protein 1 (SHANK1), SHANK2 and SHANK3, which encode a family of postsynaptic scaffolding proteins that are present at glutamatergic synapses in the CNS. An evaluation of human genetic data, as well as of in vitro and in vivo animal model data, may allow us to understand how disruption of SHANK scaffolding proteins affects the structure and function of neural circuits and alters behaviour.},
author = {Monteiro, Patricia and Feng, Guoping},
doi = {10.1038/nrn.2016.183},
issn = {1471-003X},
journal = {Nature Reviews Neuroscience},
number = {3},
pages = {147--157},
pmid = {28179641},
title = {{SHANK proteins: roles at the synapse and in autism spectrum disorder}},
url = {http://www.nature.com/doifinder/10.1038/nrn.2016.183},
volume = {18},
year = {2017}
}
@article{Montibeller2018,
abstract = {The endoplasmic reticulum (ER) plays an important role in maintenance of proteostasis through the unfolded protein response (UPR), which is strongly activated in most neurodegenerative disorders. UPR signalling pathways mediated by IRE1$\alpha$ and ATF6 play a crucial role in the maintenance of ER homeostasis through the transactivation of an array of transcription factors. When activated, these transcription factors induce the expression of genes involved in protein folding and degradation with pro-survival effects. However, the specific contribution of these transcription factors to different neurodegenerative diseases remains poorly defined. Here, we characterised 44 target genes strongly influenced by XBP1 and ATF6 and quantified the expression of a subset of genes in the human post-mortem spinal cord from amyotrophic lateral sclerosis (ALS) cases and in the frontal and temporal cortex from frontotemporal lobar degeneration (FTLD) and Alzheimer's disease (AD) cases and controls. We found that IRE1$\alpha$-XBP1 and ATF6 pathways were strongly activated both in ALS and AD. In ALS, XBP1 and ATF6 activation was confirmed by a substantial increase in the expression of both known and novel target genes involved particularly in co-chaperone activity and ER-associated degradation (ERAD) such as DNAJB9, SEL1L and OS9. In AD cases, a distinct pattern emerged, where targets involved in protein folding were more prominent, such as CANX, PDIA3 and PDIA6. These results reveal that both overlapping and disease-specific patterns of IRE1$\alpha$-XBP1 and ATF6 target genes are activated in AD and ALS, which may be relevant to the development of new therapeutic strategies. [Figure not available: see fulltext.].},
author = {Montibeller, L. and de Belleroche, J.},
doi = {10.1007/s12192-018-0897-y},
file = {::},
issn = {14661268},
journal = {Cell Stress and Chaperones},
keywords = {AD,ALS,ER stress,ERAD,Folding,UPR},
month = {sep},
number = {5},
pages = {897--912},
publisher = {Cell Stress and Chaperones},
title = {{Amyotrophic lateral sclerosis (ALS) and Alzheimer's disease (AD) are characterised by differential activation of ER stress pathways: focus on UPR target genes}},
volume = {23},
year = {2018}
}
@article{Moon2016,
abstract = {Peripheral processes that mediate beneficial effects of exercise on the brain remain sparsely explored. Here, we show that a muscle secretory factor, cathepsin B (CTSB) protein, is important for the cognitive and neurogenic benefits of running. Proteomic analysis revealed elevated levels of CTSB in conditioned medium derived from skeletal muscle cell cultures treated with AMP-kinase agonist AICAR. Consistently, running increased CTSB levels in mouse gastrocnemius muscle and plasma. Furthermore, recombinant CTSB application enhanced expression of brain-derived neurotrophic factor (BDNF) and doublecortin (DCX) in adult hippocampal progenitor cells through a mechanism dependent on the multifunctional protein P11. In vivo, in CTSB knockout (KO) mice, running did not enhance adult hippocampal neurogenesis and spatial memory function. Interestingly, in Rhesus monkeys and humans, treadmill exercise elevated CTSB in plasma. In humans, changes in CTSB levels correlated with fitness and hippocampus-dependent memory function. Our findings suggest CTSB as a mediator of effects of exercise on cognition.},
author = {Moon, Hyo Youl and Becke, Andreas and Berron, David and Becker, Benjamin and Sah, Nirnath and Benoni, Galit and Janke, Emma and Lubejko, Susan T. and Greig, Nigel H. and Mattison, Julie A. and Duzel, Emrah and van Praag, Henriette},
doi = {10.1016/j.cmet.2016.05.025},
issn = {19327420},
journal = {Cell Metabolism},
keywords = {cathepsin B,exercise,hippocampus,humans,memory,mice,muscle},
title = {{Running-Induced Systemic Cathepsin B Secretion Is Associated with Memory Function}},
year = {2016}
}
@misc{Moreno-Garcia2018,
abstract = {Despite aging being by far the greatest risk factor for highly prevalent neurodegenerative disorders, the molecular underpinnings of age-related brain changes are still not well understood, particularly the transition from normal healthy brain aging to neuropathological aging. Aging is an extremely complex, multifactorial process involving the simultaneous interplay of several processes operating at many levels of the functional organization. The buildup of potentially toxic protein aggregates and their spreading through various brain regions has been identified as a major contributor to these pathologies. One of the most striking morphologic changes in neurons during normal aging is the accumulation of lipofuscin (LF) aggregates, as well as, neuromelanin pigments. LF is an autofluorescent lipopigment formed by lipids, metals and misfolded proteins, which is especially abundant in nerve cells, cardiac muscle cells and skin. Within the Central Nervous System (CNS), LF accumulates as aggregates, delineating a specific senescence pattern in both physiological and pathological states, altering neuronal cytoskeleton and cellular trafficking and metabolism, and being associated with neuronal loss, and glial proliferation and activation. Traditionally, the accumulation of LF in the CNS has been considered a secondary consequence of the aging process, being a mere bystander of the pathological buildup associated with different neurodegenerative disorders. Here, we discuss recent evidence suggesting the possibility that LF aggregates may have an active role in neurodegeneration. We argue that LF is a relevant effector of aging that represents a risk factor or driver for neurodegenerative disorders.},
author = {Moreno-Garc{\'{i}}a, Alexandra and Kun, Alejandra and Calero, Olga and Medina, Miguel and Calero, Miguel},
booktitle = {Frontiers in Neuroscience},
doi = {10.3389/fnins.2018.00464},
file = {::},
issn = {1662453X},
keywords = {Aging,Amyloid,Autofluorescence,Lipofuscin,Neurodegeneration,Oxidative stress,Protein deposits},
month = {jul},
number = {JUL},
pages = {464},
publisher = {Frontiers Media S.A.},
title = {{An overview of the role of lipofuscin in age-related neurodegeneration}},
volume = {12},
year = {2018}
}
@article{Mucha2010,
abstract = {Network science is an interdisciplinary endeavor, with methods and applications drawn from across the natural, social, and information sciences. A prominent problem in network science is the algorithmic detection of tightly connected groups of nodes known as communities. We developed a generalized framework of network quality functions that allowed us to study the community structure of arbitrary multislice networks, which are combinations of individual networks coupled through links that connect each node in one network slice to itself in other slices. This framework allows studies of community structure in a general setting encompassing networks that evolve over time, have multiple types of links (multiplexity), and have multiple scales.},
archivePrefix = {arXiv},
arxivId = {0911.1824},
author = {Mucha, Peter J. and Richardson, Thomas and Macon, Kevin and Porter, Mason A. and Onnela, Jukka Pekka},
doi = {10.1126/science.1184819},
eprint = {0911.1824},
issn = {00368075},
journal = {Science},
month = {may},
number = {5980},
pages = {876--878},
publisher = {American Association for the Advancement of Science},
title = {{Community structure in time-dependent, multiscale, and multiplex networks}},
url = {https://science.sciencemag.org/content/328/5980/876 https://science.sciencemag.org/content/328/5980/876.abstract},
volume = {328},
year = {2010}
}
@misc{Mukherjee2019,
abstract = {Neuronal Ceroid Lipofuscinoses (NCLs), commonly known as Batten disease, constitute a group of the most prevalent neurodegenerative lysosomal storage disorders (LSDs). Mutations in at least 13 different genes (called CLNs) cause various forms of NCLs. Clinically, the NCLs manifest early impairment of vision, progressive decline in cognitive and motor functions, seizures and a shortened lifespan. At the cellular level, all NCLs show intracellular accumulation of autofluorescent material (called ceroid) and progressive neuron loss. Despite intense studies the normal physiological functions of each of the CLN genes remain poorly understood. Consequently, the development of mechanism-based therapeutic strategies remains challenging. Endolysosomal dysfunction contributes to pathogenesis of virtually all LSDs. Studies within the past decade have drastically changed the notion that the lysosomes are merely the terminal degradative organelles. The emerging new roles of the lysosome include its central role in nutrient-dependent signal transduction regulating metabolism and cellular proliferation or quiescence. In this review, we first provide a brief overview of the endolysosomal and autophagic pathways, lysosomal acidification and endosome-lysosome and autophagosome-lysosome fusions. We emphasize the importance of these processes as their dysregulation leads to pathogenesis of many LSDs including the NCLs. We also describe what is currently known about each of the 13 CLN genes and their products and how understanding the emerging new roles of the lysosome may clarify the underlying pathogenic mechanisms of the NCLs. Finally, we discuss the current and emerging therapeutic strategies for various NCLs.},
author = {Mukherjee, Anil B. and Appu, Abhilash P. and Sadhukhan, Tamal and Casey, Sydney and Mondal, Avisek and Zhang, Zhongjian and Bagh, Maria B.},
booktitle = {Molecular Neurodegeneration},
doi = {10.1186/s13024-018-0300-6},
file = {::},
issn = {17501326},
keywords = {Batten Disease,Lysosomal Storage Disease,Neurodegeneration,Neuronal Ceroid Lipofuscinosis},
month = {jan},
number = {1},
publisher = {BioMed Central Ltd.},
title = {{Emerging new roles of the lysosome and neuronal ceroid lipofuscinoses}},
volume = {14},
year = {2019}
}
@article{Nagel2017,
abstract = {The Wiskott-Aldrich syndrome protein and SCAR homolog (WASH; also known as Washout in flies) is a conserved actin-nucleationpromoting factor controlling Arp2/3 complex activity in endosomal sorting and recycling. Previous studies have identified WASH as an essential regulator in Drosophila development. Here, we show that homozygous wash mutant flies are viable and fertile.We demonstrate that Drosophila WASH has conserved functions in integrin receptor recycling and lysosome neutralization. WASH generates actin patches on endosomes and lysosomes, thereby mediating both aforementioned functions. Consistently, loss of WASH function results in cell spreading and cell migration defects of macrophages, and an increased lysosomal acidification that affects efficient phagocytic and autophagic clearance. WASH physically interacts with the vacuolar (V)-ATPase subunit Vha55 that is crucial to establish and maintain lysosome acidification. As a consequence, starved flies that lack WASH function show a dramatic increase in acidic autolysosomes, causing a reduced lifespan. Thus, our data highlight a conserved role for WASH in the endocytic sorting and recycling of membrane proteins, such as integrins and the V-ATPase, that increase the likelihood of survival under nutrient deprivation.},
author = {Nagel, Benedikt M. and Bechtold, Meike and Rodriguez, Luis Garcia and Bogdan, Sven},
doi = {10.1242/jcs.193086},
file = {::},
issn = {14779137},
journal = {Journal of Cell Science},
keywords = {Actin cytoskeleton,Arp2/3,Cell adhesion,Drosophila,Integrin,Lysosome neutralization,Macrophages,V-ATPase,WASH,Wiskott-Aldrich syndrome},
month = {jan},
number = {2},
pages = {344--359},
pmid = {27884932},
publisher = {Company of Biologists Ltd},
title = {{Drosophila WASH is required for integrin-mediated cell adhesion, cell motility and lysosomal neutralization}},
volume = {130},
year = {2017}
}
@article{Pal2006,
abstract = {The molecular mechanisms underlying the targeting of Huntingtin (Htt) to endosomes and its multifaceted role in endocytosis are poorly understood. In this study, we have identi. ed Htt-associated protein 40 (HAP40) as a novel effector of the small guanosine triphosphatase Rab5, a key regulator of endocytosis. HAP40 mediates the recruitment of Htt by Rab5 onto early endosomes. HAP40 overexpression caused a drastic reduction of early endosomal motility through their displacement from microtubules and preferential association with actin filaments. Remarkably, endogenous HAP40 was up-regulated in fibroblasts and brain tissue from human patients affected by Huntington's disease (HD) as well as in STHdhQ111 striatal cells established from a HD mouse model. These cells consistently displayed altered endosome motility and endocytic activity, which was restored by the ablation of HAP40. In revealing an unexpected link between Rab5, HAP40, and Htt, we uncovered a new mechanism regulating cytoskeleton-dependent endosome dynamics and its dysfunction under pathological conditions. {\textcopyright} The Rockefeller University Press.},
author = {Pal, Arun and Severin, Fedor and Lommer, Barbara and Shevchenko, Anna and Zerial, Marino},
doi = {10.1083/jcb.200509091},
file = {::},
issn = {00219525},
journal = {Journal of Cell Biology},
month = {feb},
number = {4},
pages = {605--618},
pmid = {16476778},
title = {{Huntingtin-HAP40 complex is a novel Rab5 effector that regulates early endosome motility and is up-regulated in Huntington's disease}},
volume = {172},
year = {2006}
}
@article{Pan2010,
abstract = {Ulcerative colitis (UC) is one of the major forms of inflammatory bowel disease with unknown cause. A molecular marker, WAFL, has recently been found to be up-regulated in the inflamed colonic mucosa of UC patients. Towards understanding biological function of WAFL, we analyzed proteins interacting with WAFL in HEK-293 cells by immunoprecipitation and mass spectrometry. Among four proteins found to specifically interact with WAFL, both KIAA0196 and KIAA1033 bind to $\alpha$-appendage of the adaptor protein complex 2 (AP2), which acts as an interaction hub for accessory proteins in endocytosis mediated by clathrin-coated vesicle (CCV). The specific interaction between WAFL and KIAA0196 was also confirmed in human colorectal carcinoma HCT-116 cells by co-immunoprecipitation with specific antibodies. Meta-analyses of the databases of expressed genes suggest that the three genes are co-expressed in many tissues and cell types, and that their molecular function may be classified in the category of 'membrane traffic protein'. Therefore, these results suggest that WAFL may play an important role in endocytosis and subsequent membrane trafficking by interacting with AP2 through KIAA0196 and KIAA1033. {\textcopyright} Ivyspring International Publisher.},
author = {Pan, You Fu and Viklund, Ing Marie and Tsai, Heng Hang and Pettersson, Sven and Maruyama, Ichiro N.},
doi = {10.7150/ijbs.6.163},
file = {::},
issn = {14492288},
journal = {International Journal of Biological Sciences},
keywords = {Inflammatory bowel disease,KIAA0196/strumpellin,KIAA1033,Proteomics,WAFL/FKBP15/FKBP133/KIAA0674,Wiskott-Aldrich syndrome protein},
number = {2},
pages = {163--171},
publisher = {Ivyspring International Publisher},
title = {{The ulcerative colitis marker protein WAFL interacts with accessory proteins in endocytosis}},
volume = {6},
year = {2010}
}
@misc{Patel2018,
abstract = {The state of enzymes in the human body determines the normal physiology or pathology, so all the six classes of enzymes are crucial. Proteases, the hydrolases, can be of several types based on the nucleophilic amino acid or the metal cofactor needed for their activity. Cathepsins are proteases with serine, cysteine, or aspartic acid residues as the nucleophiles, which are vital for digestion, coagulation, immune response, adipogenesis, hormone liberation, peptide synthesis, among a litany of other functions. But inflammatory state radically affects their normal roles. Released from the lysosomes, they degrade extracellular matrix proteins such as collagen and elastin, mediating parasite infection, autoimmune diseases, tumor metastasis, cardiovascular issues, and neural degeneration, among other health hazards. Over the years, the different types and isoforms of cathepsin, their optimal pH and functions have been studied, yet much information is still elusive. By taming and harnessing cathepsins, by inhibitors and judicious lifestyle, a gamut of malignancies can be resolved. This review discusses these aspects, which can be of clinical relevance.},
author = {Patel, Seema and Homaei, Ahmad and El-Seedi, Hesham R. and Akhtar, Nadeem},
booktitle = {Biomedicine and Pharmacotherapy},
doi = {10.1016/j.biopha.2018.05.148},
file = {::},
issn = {19506007},
keywords = {Carcinogenesis,Cathepsin,Extracellular matrix,Immune activation,Inflammation,Inhibitors},
month = {sep},
pages = {526--532},
pmid = {29885636},
publisher = {Elsevier Masson SAS},
title = {{Cathepsins: Proteases that are vital for survival but can also be fatal}},
volume = {105},
year = {2018}
}
@article{Phillips-Krawczak2015,
abstract = {COMMD1 deficiency results in defective copper homeostasis, but the mechanism for this has remained elusive. Here we report that COMMD1 is directly linked to early endosomes through its interaction with a protein complex containing CCDC22, CCDC93, and C16orf62. This COMMD/CCDC22/CCDC93 (CCC) complex interacts with the multisubunit WASH complex, an evolutionarily conserved system, which is required for endosomal deposition of F-actin and cargo trafficking in conjunction with the retromer. Interactions between the WASH complex subunit FAM21, and the carboxyl-terminal ends of CCDC22 and CCDC93 are responsible for CCC complex recruitment to endosomes. We show that depletion of CCC complex components leads to lack of copper-dependent movement of the copper transporter ATP7A from endosomes, resulting in intracellular copper accumulation and modest alterations in copper homeostasis in humans with CCDC22 mutations. This work provides a mechanistic explanation for the role of COMMD1 in copper homeostasis and uncovers additional genes involved in the regulation of copper transporter recycling.},
author = {Phillips-Krawczak, Christine A. and Singla, Amika and Starokadomskyy, Petro and Deng, Zhihui and Osborne, Douglas G. and Li, Haiying and Dick, Christopher J. and Gomez, Timothy S. and Koenecke, Megan and Zhang, Jin San and Dai, Haiming and Sifuentes-Dominguez, Luis F. and Geng, Linda N. and Kaufmann, Scott H. and Hein, Marco Y. and Wallis, Mathew and McGaughran, Julie and Gecz, Jozef and {Van De Sluis}, Bart and Billadeau, Daniel D. and Burstein, Ezra},
doi = {10.1091/mbc.E14-06-1073},
issn = {19394586},
journal = {Molecular Biology of the Cell},
month = {jan},
number = {1},
pages = {91--103},
pmid = {25355947},
publisher = {American Society for Cell Biology},
title = {{COMMD1 is linked to the WASH complex and regulates endosomal trafficking of the copper transporter ATP7A}},
volume = {26},
year = {2015}
}
@article{Ping2018,
abstract = {Patients with Alzheimer's disease (AD) and Parkinson's disease (PD) often have overlap in clinical presentation and brain neuropathology suggesting that these two diseases share common underlying mechanisms. Currently, the molecular pathways linking AD and PD are incompletely understood. Utilizing Tandem Mass Tag (TMT) isobaric labeling and synchronous precursor selection-based MS3 (SPS-MS3) mass spectrometry, we performed an unbiased quantitative proteomic analysis of post-mortem human brain tissues (n=80) from four different groups defined as controls, AD, PD, and co-morbid AD/PD cases across two brain regions (frontal cortex and anterior cingulate gyrus). In total, we identified 11 840 protein groups representing 10 230 gene symbols, which map to ∼65{\%} of the protein coding genes in brain. The utility of including two reference standards in each TMT 10-plex assay to assess intra-A nd inter-batch variance is also described. Ultimately, this comprehensive human brain proteomic dataset serves as a valuable resource for various research endeavors including, but not limited to, the identification of disease-specific protein signatures and molecular pathways that are common in AD and PD.},
author = {Ping, Lingyan and Duong, Duc M. and Yin, Luming and Gearing, Marla and Lah, James J. and Levey, Allan I. and Seyfried, Nicholas T.},
doi = {10.1038/sdata.2018.36},
issn = {20524463},
journal = {Scientific Data},
title = {{Global quantitative analysis of the human brain proteome in Alzheimer's and Parkinson's Disease}},
year = {2018}
}
@article{Piotrowski2013,
abstract = {WASH is an Arp2/3 activator of the Wiskott-Aldrich syndrome protein superfamily that functions during endosomal trafficking processes in collaboration with the retromer and sorting nexins, but its in vivo function has not been examined. To elucidate the physiological role of WASH in T cells, we generated a WASH conditional knockout (WASHout) mouse model. Using CD4(Cre) deletion, we found that thymocyte development and naive T cell activation are unaltered in the absence of WASH. Surprisingly, despite normal T cell receptor (TCR) signaling and interleukin-2 production, WASHout T cells demonstrate significantly reduced proliferative potential and fail to effectively induce experimental autoimmune encephalomyelitis. Interestingly, after activation, WASHout T cells fail to maintain surface levels of TCR, CD28, and LFA-1. Moreover, the levels of the glucose transporter, GLUT1, are also reduced compared to wild-type T cells. We further demonstrate that the loss of surface expression of these receptors in WASHout cells results from aberrant accumulation within the collapsed endosomal compartment, ultimately leading to degradation within the lysosome. Subsequently, activated WASHout T cells experience reduced glucose uptake and metabolic output. Thus, we found that WASH is a newly recognized regulator of TCR, CD28, LFA-1, and GLUT1 endosome-to-membrane recycling. Aberrant trafficking of these key T cell proteins may potentially lead to attenuated proliferation and effector function.},
author = {Piotrowski, J. T. and Gomez, T. S. and Schoon, R. A. and Mangalam, A. K. and Billadeau, D. D.},
doi = {10.1128/MCB.01288-12},
isbn = {1098-5549 (Electronic)$\backslash$r0270-7306 (Linking)},
issn = {0270-7306},
journal = {Molecular and Cellular Biology},
pmid = {23275443},
title = {{WASH Knockout T Cells Demonstrate Defective Receptor Trafficking, Proliferation, and Effector Function}},
year = {2013}
}
@article{Plubell2017,
abstract = {The lack of high-throughput methods to analyze the adipose tissue protein composition limits our understanding of the protein networks responsible for age and diet related metabolic response. We have developed an approach using multiple-dimension liquid chromatography tandem mass spectrometry and extended multiplexing (24 biological samples) with tandem mass tags (TMT) labeling to analyze proteomes of epididymal adipose tissues isolated from mice fed either low or high fat diet for a short or a long-term, and from mice that aged on low versus high fat diets. The peripheral metabolic health (as measured by body weight, adiposity, plasma fasting glucose, insulin, triglycerides, total cholesterol levels, and glucose and insulin tolerance tests) deteriorated with diet and advancing age, with long-term high fat diet exposure being the worst. In response to short-term high fat diet, 43 proteins representing lipid metabolism (e.g. AACS, ACOX1, ACLY) and red-ox pathways (e.g. CPD2, CYP2E, SOD3) were significantly altered (FDR {\textless} 10{\%}). Long-term high fat diet significantly altered 55 proteins associated with immune response (e.g. IGTB2, IFIT3, LGALS1) and rennin angiotensin system (e.g. ENPEP, CMA1, CPA3, AN-PEP). Age-related changes on low fat diet significantly altered only 18 proteins representing mainly urea cycle (e.g. OTC, ARG1, CPS1), and amino acid biosynthesis (e.g. GMT, AKR1C6). Surprisingly, high fat diet driven age-related changes culminated with alterations in 155 proteins involving primarily the urea cycle (e.g. ARG1, CPS1), immune response/complement activation (e.g. C 3, C4b, C 8, C9, CFB, CFH, FGA), extracellular remodeling (e.g. EFEMP1, FBN1, FBN2, LTBP4, FERMT2, ECM1, EMILIN2, ITIH3) and apoptosis (e.g. YAP1, HIP1, NDRG1, PRKCD, MUL1) pathways. Using our adipose tissue tailored approach we have identified both age-related and high fat diet specific proteomic signatures highlighting a pronounced involvement of arginine metabolism in response to advancing age, and branched chain amino acid metabolism in early response to high fat feeding. Data are available via ProteomeXchange with identifier PXD005953.},
author = {Plubell, Deanna L. and Wilmarth, Phillip A. and Zhao, Yuqi and Fenton, Alexandra M. and Minnier, Jessica and Reddy, Ashok P. and Klimek, John and Yang, Xia and David, Larry L. and Pamir, Nathalie},
doi = {10.1074/mcp.M116.065524},
issn = {15359484},
journal = {Molecular and Cellular Proteomics},
title = {{Extended multiplexing of tandem mass tags (TMT) labeling reveals age and high fat diet specific proteome changes in mouse epididymal adipose tissue}},
year = {2017}
}
@article{Poet2006,
abstract = {Mammalian CLC proteins function as Cl- channels or as electrogenic Cl-/H+ exchangers and are present in the plasma membrane and intracellular vesicles. We now show that the ClC-6 protein is almost exclusively expressed in neurons of the central and peripheral nervous systems, with a particularly high expression in dorsal root ganglia. ClC-6 colocalized with markers for late endosomes in neuronal cell bodies. The disruption of ClC-6 in mice reduced their pain sensitivity and caused moderate behavioral abnormalities. Neuronal tissues showed autofluorescence at initial axon segments. At these sites, electron microscopy revealed electron-dense storage material that caused a pathological enlargement of proximal axons. These deposits were positive for several lysosomal proteins and other marker proteins typical for neuronal ceroid lipofuscinosis (NCL), a lysosomal storage disease. However, the lysosomal pH of Clcn6-/- neurons appeared normal. CLCN6 is a candidate gene for mild forms of human NCL. Analysis of 75 NCL patients identified ClC-6 amino acid exchanges in two patients but failed to prove a causative role of CLCN6 in that disease. {\textcopyright} 2006 by The National Academy of Sciences of the USA.},
author = {Po{\"{e}}t, Mallorie and Kornak, Uwe and Schweizer, Michaela and Zdebik, Anselm A. and Scheel, Olaf and Hoelter, Sabine and Wurst, Wolfgang and Schmitt, Anja and Fuhrmann, Jens C. and Planells-Cases, Rosa and Mole, Sara E. and H{\"{u}}bner, Christian A. and Jentsch, Thomas J.},
doi = {10.1073/pnas.0606137103},
file = {::},
issn = {00278424},
journal = {Proceedings of the National Academy of Sciences of the United States of America},
keywords = {Acidification,Anion transport,Batten disease,Channelopathy,Kufs' disease},
month = {sep},
number = {37},
pages = {13854--13859},
publisher = {National Academy of Sciences},
title = {{Lysosomal storage disease upon disruption of the neuronal chloride transport protein ClC-6}},
volume = {103},
year = {2006}
}
@misc{Porter1999,
abstract = {Caspases are crucial mediators of programmed cell death (apoptosis). Among them, caspase-3 is a frequently activated death protease, catalyzing the specific cleavage of many key cellular proteins. However, the specific requirements of this (or any other) caspase in apoptosis have remained largely unknown until now. Pathways to caspase-3 activation have been identified that are either dependent on or independent of mitochondrial cytochrome c release and caspase-9 function. Caspase-9 is essential for normal brain development and is important or essential in other apoptotic scenarios in a remarkable tissue-, cell type- or death stimulus-specific manner. Caspase-3 is also required for some typical hallmarks of apoptosis, and is indispensable for apoptotic chromatin condensation and DNA fragmentation in all cell types examined. Thus, caspase-3 is essential for certain processes associated with the dismantling of the cell and the formation of apoptotic bodies, but it may also function before or at the stage when commitment to loss of cell viability is made.},
author = {Porter, Alan G. and J{\"{a}}nicke, Reiner U.},
booktitle = {Cell Death and Differentiation},
doi = {10.1038/sj.cdd.4400476},
file = {::},
issn = {13509047},
keywords = {Apoptotic morphology,Caspase-3,Cytochrome c,DNA fragmentation},
month = {feb},
number = {2},
pages = {99--104},
publisher = {Nature Publishing Group},
title = {{Emerging roles of caspase-3 in apoptosis}},
url = {http://www.ncbi.nlm.nih.gov/pubmed/10200555},
volume = {6},
year = {1999}
}
@misc{Pottier2016,
abstract = {Frontotemporal lobar degeneration (FTLD) comprises a highly heterogeneous group of disorders clinically associated with behavioral and personality changes, language impairment, and deficits in executive functioning, and pathologically associated with degeneration of frontal and temporal lobes. Some patients present with motor symptoms including amyotrophic lateral sclerosis. Genetic research over the past two decades in FTLD families led to the identification of three common FTLD genes (microtubule-associated protein tau, progranulin, and chromosome 9 open reading frame 72) and a small number of rare FTLD genes, explaining the disease in almost all autosomal dominant FTLD families but only a minority of apparently sporadic patients or patients in whom the family history is less clear. Identification of additional FTLD (risk) genes is therefore highly anticipated, especially with the emerging use of next-generation sequencing. Common variants in the transmembrane protein 106 B were identified as a genetic risk factor of FTLD and disease modifier in patients with known mutations. This review summarizes for each FTLD gene what we know about the type and frequency of mutations, their associated clinical and pathological features, and potential disease mechanisms. We also provide an overview of emerging disease pathways encompassing multiple FTLD genes. We further discuss how FTLD specific issues, such as disease heterogeneity, the presence of an unclear family history and the possible role of an oligogenic basis of FTLD, can pose challenges for future FTLD gene identification and risk assessment of specific variants. Finally, we highlight emerging clinical, genetic, and translational research opportunities that lie ahead. (Figure presented.) Genetic research led to the identification of three common FTLD genes with rare variants (MAPT, GRN, and C9orf72) and a small number of rare genes. Efforts are now ongoing, which aimed at the identification of rare variants with high risk and/or low frequency variants with intermediate effect. Common risk variants have also been identified, such as TMEM106B. This review discusses the current knowledge on FTLD genes and the emerging disease pathways encompassing multiple FTLD genes.},
author = {Pottier, Cyril and Ravenscroft, Thomas A. and Sanchez-Contreras, Monica and Rademakers, Rosa},
booktitle = {Journal of Neurochemistry},
doi = {10.1111/jnc.13622},
issn = {14714159},
keywords = {C9orf72,MAPT,TMEM106B,frontotemporal lobar degeneration,genetics,progranulin},
month = {aug},
pages = {32--53},
publisher = {Blackwell Publishing Ltd},
title = {{Genetics of FTLD: overview and what else we can expect from genetic studies}},
year = {2016}
}
@article{Powell2005,
abstract = {Cellular senescence may be accompanied by accumulation of large aggregates of oxidized proteins, also known as lipofuscin. The hypothesis that cellular accumulation of lipofuscin-like materials (LIP) results in cell death as a result of proteasome inhibition was examined. Rat neonatal cardiomyocytes were incubated with synthetic LIP for up to 48 h. This was accompanied by increases in cellular autofluorescence (207{\%} by 48 h; p {\textless} 0.05) and electron microscopic evidence of internalization of LIP particles. LIP incubation resulted in loss of viability (-46{\%} by 48 h; p {\textless} 0.05) through apoptotic cell death. Although 20S-proteasome activity was increased by 74{\%} after 6 h, both 20S- and 26S-proteasome activities were decreased after 48 h of incubation (-54{\%} (p {\textless} 0.05) and -50{\%}, respectively), accompanied by large increases in ubiquitinated proteins. Several proteasome-regulated proapoptotic proteins, including c-Jun (2.9-fold; p {\textless} 0.05), Bax (1.8-fold; p {\textless} 0.05), and p27kip1 (3.2-fold; p {\textless} 0.05), were observed to be increased by 48 h. Observation of ubiquitinated homologues of Bax and p27kip1 suggested that part of the increase was due to decreased proteasomal degradation of these proteins. The results of this study are consistent with the conclusion that accumulation of LIP results in inhibition of the proteasome, which initiates an apoptotic cascade as a result of dysregulation of several proapoptotic proteins. {\textcopyright} 2005 Elsevier Inc. All rights reserved.},
author = {Powell, Saul R. and Wang, Ping and Divald, Andras and Teichberg, Saul and Haridas, Viraga and McCloskey, Thomas W. and Davies, Kelvin J.A. and Katzeff, Harvey},
doi = {10.1016/j.freeradbiomed.2005.01.003},
issn = {08915849},
journal = {Free Radical Biology and Medicine},
keywords = {Apoptosis,Cell death,Free radicals,Lipofuscin,Proteasome,Senescence},
month = {apr},
number = {8},
pages = {1093--1101},
publisher = {Elsevier Inc.},
title = {{Aggregates of oxidized proteins (lipofuscin) induce apoptosis through proteasome inhibition and dysregulation of proapoptotic proteins}},
url = {https://pubmed.ncbi.nlm.nih.gov/15780767/},
volume = {38},
year = {2005}
}
@article{Quadri2013,
abstract = {Autosomal recessive, early-onset Parkinsonism is clinically and genetically heterogeneous. Here, we report the identification, by homozygosity mapping and exome sequencing, of a SYNJ1 homozygous mutation (p.Arg258Gln) segregating with disease in an Italian consanguineous family with Parkinsonism, dystonia, and cognitive deterioration. Response to levodopa was poor, and limited by side effects. Neuroimaging revealed brain atrophy, nigrostriatal dopaminergic defects, and cerebral hypometabolism. SYNJ1 encodes synaptojanin 1, a phosphoinositide phosphatase protein with essential roles in the postendocytic recycling of synaptic vesicles. The mutation is absent in variation databases and in ethnically matched controls, is damaging according to all prediction programs, and replaces an amino acid that is extremely conserved in the synaptojanin 1 homologues and in SAC1-like domains of other proteins. Sequencing the SYNJ1 ORF in unrelated patients revealed another heterozygous mutation (p.Ser1422Arg), predicted as damaging, in a patient who also carries a heterozygous PINK1 truncating mutation. The SYNJ1 gene is a compelling candidate for Parkinsonism; mutations in the functionally linked protein auxilin cause a similar early-onset phenotype, and other findings implicate endosomal dysfunctions in the pathogenesis. Our data delineate a novel form of human Mendelian Parkinsonism, and provide further evidence for abnormal synaptic vesicle recycling as a central theme in the pathogenesis. By homozygosity mapping and exome sequencing in an Italian consanguineous family with early-onset Parkinsonism, we identified a disease-segregating homozygous SYNJ1 mutation. SYNJ1 encodes synaptojanin 1, a phosphoinositide phosphatase, essential for the post-endocytic recycling of synaptic vesicles. This work delineates a novel form of Mendelian Parkinsonism and provides further evidence for abnormal synaptic vesicle recycling as a central theme in the pathogenesis. {\textcopyright} 2013 WILEY PERIODICALS, INC.},
author = {Quadri, Marialuisa and Fang, Mingyan and Picillo, Marina and Olgiati, Simone and Breedveld, Guido J. and Graafland, Josja and Wu, Bin and Xu, Fengping and Erro, Roberto and Amboni, Marianna and Pappat{\`{a}}, Sabina and Quarantelli, Mario and Annesi, Grazia and Quattrone, Aldo and Chien, Hsin F. and Barbosa, Egberto R. and Oostra, Ben A. and Barone, Paolo and Wang, Jun and Bonifati, Vincenzo},
doi = {10.1002/humu.22373},
issn = {10597794},
journal = {Human Mutation},
keywords = {Dementia,Dystonia,Gene,Mutation,PINK1,Parkinsonism,SYNJ1},
month = {sep},
number = {9},
pages = {1208--1215},
title = {{Mutation in the SYNJ1 gene associated with autosomal recessive, early-onset parkinsonism}},
volume = {34},
year = {2013}
}
@article{Ramirez-Montealegre2005,
abstract = {Mutations in the CLN3 gene, which encodes a lysosomal membrane protein, are responsible for the neurodegenerative disorder juvenile Batten disease. A previous study on the yeast homolog to CLN3, designated Btn1p, revealed a potential role for CLN3 in the transport of arginine into the yeast vacuole, the equivalent organelle to the mammalian lysosome. Lysosomes isolated from lymphoblast cell lines, established from individuals with juvenile Batten disease-bearing mutations in CLN3, but not age-matched controls, demonstrate defective transport of arginine. Furthermore, we show that there is a depletion of arginine in cells derived from individuals with juvenile Batten disease. We have, therefore, characterized lysosomal arginine transport in normal lysosomes and show that it is ATP-, v-ATPase- and cationic-dependent. This and previous studies have shown that both arginine and lysine are transported by the same transport system, designated system c. However, we report that lysosomes isolated from juvenile Batten disease lymphoblasts are only defective for arginine transport. These results suggest that the CLN3 defect in juvenile Batten disease may affect how intracellular levels of arginine are regulated or distributed throughout the cell. This assertion is supported by two other experimental approaches. First, an antibody to CLN3 can block lysosomal arginine transport and second, expression of CLN3 in JNCL cells using a lentiviral vector can restore lysosomal arginine transport. CLN3 may have a role in regulating intracellular levels of arginine possibly through control of the transport of this amino acid into lysosomes. {\textcopyright} The Author 2005. Published by Oxford University Press. All rights reserved.},
author = {Ramirez-Montealegre, Denia and Pearce, David A.},
doi = {10.1093/hmg/ddi406},
issn = {09646906},
journal = {Human Molecular Genetics},
number = {23},
pages = {3759--3773},
title = {{Defective lysosomal arginine transport in juvenile Batten disease}},
url = {https://www.ncbi.nlm.nih.gov/pubmed/?term=16251196},
volume = {14},
year = {2005}
}
@article{Ramirez2006,
abstract = {Neurodegenerative disorders such as Parkinson and Alzheimer disease cause motor and cognitive dysfunction and belong to a heterogeneous group of common and disabling disorders. Although the complex molecular pathophysiology of neurodegeneration is largely unknown, major advances have been achieved by elucidating the genetic defects underlying mendelian forms of these diseases. This has led to the discovery of common pathophysiological pathways such as enhanced oxidative stress, protein misfolding and aggregation and dysfunction of the ubiquitin-proteasome system. Here, we describe loss-of-function mutations in a previously uncharacterized, predominantly neuronal P-type ATPase gene, ATP13A2, underlying an autosomal recessive form of early-onset parkinsonism with pyramidal degeneration and dementia (PARK9, Kufor-Rakeb syndrome). Whereas the wild-type protein was located in the lysosome of transiently transfected cells, the unstable truncated mutants were retained in the endoplasmic reticulum and degraded by the proteasome. Our findings link a class of proteins with unknown function and substrate specificity to the protein networks implicated in neurodegeneration and parkinsonism. {\textcopyright} 2006 Nature Publishing Group.},
author = {Ramirez, Alfredo and Heimbach, Andr{\'{e}} and Gr{\"{u}}ndemann, Jan and Stiller, Barbara and Hampshire, Dan and Cid, L. Pablo and Goebel, Ingrid and Mubaidin, Ammar F. and Wriekat, Abdul Latif and Roeper, Jochen and Al-Din, Amir and Hillmer, Axel M. and Karsak, Meliha and Liss, Birgit and Woods, C. Geoffrey and Behrens, Maria I. and Kubisch, Christian},
doi = {10.1038/ng1884},
file = {::},
issn = {10614036},
journal = {Nature Genetics},
keywords = {Adenosine Triphosphatases / genetics*,Adenosine Triphosphatases / metabolism,Alfredo Ramirez,Andr{\'{e}} Heimbach,Christian Kubisch,Dementia / etiology*,Dementia / genetics,Endoplasmic Reticulum / enzymology,Female,Humans,Lysosomes / enzymology*,MEDLINE,Male,Mesencephalon / enzymology,Mesencephalon / pathology,Mutation*,NCBI,NIH,NLM,National Center for Biotechnology Information,National Institutes of Health,National Library of Medicine,Neurons / enzymology,Neurons / pathology,Non-U.S. Gov't,Parkinsonian Disorders / complications,Parkinsonian Disorders / genetics*,Proton-Translocating ATPases / genetics*,PubMed Abstract,Research Support,doi:10.1038/ng1884,pmid:16964263},
month = {oct},
number = {10},
pages = {1184--1191},
publisher = {Nat Genet},
title = {{Hereditary parkinsonism with dementia is caused by mutations in ATP13A2, encoding a lysosomal type 5 P-type ATPase}},
url = {https://pubmed.ncbi.nlm.nih.gov/16964263/ https://pubmed.ncbi.nlm.nih.gov/16964263/?from{\_}single{\_}result=Hereditary+parkinsonism+with+dementia+is+caused+by+mutations+in+ATP13A2{\%}2C+encoding+a+lysosomal+type+5+P-type+ATPase},
volume = {38},
year = {2006}
}
@book{Regier1993,
abstract = {CLINICAL CHARACTERISTICS GLB1-related disorders comprise two phenotypically distinct lysosomal storage disorders: GM1 gangliosidosis and mucopolysaccharidosis type IVB (MPS IVB). GM1 gangliosidosis includes phenotypes that range from severe to mild. Type I (infantile) begins before age one year; progressive central nervous system dysfunction leads to spasticity, deafness, blindness, and decerebrate rigidity. Life expectancy is two to three years. Type II can be subdivided into the late-infantile form and juvenile form. Type II, late-infantile form begins between ages one and three years; life expectancy is five to ten years. Type II, juvenile form begins between ages three and ten years with insidious plateauing of motor and cognitive development followed by slow regression. Type II may or may not include skeletal dysplasia. Type III begins in the second to third decade with extrapyramidal signs, gait disturbance, and cardiomyopathy; and can be misidentified as Parkinson disease. Intellectual impairment is common late in the disease; skeletal involvement includes short stature, kyphosis, and scoliosis of varying severity. MPS IVB is characterized by skeletal changes, including short stature and skeletal dysplasia. Affected children have no distinctive clinical findings at birth. The severe form is usually apparent between ages one and three years, and the attenuated form in late childhood or adolescence. In addition to skeletal involvement, significant morbidity can result from respiratory compromise, obstructive sleep apnea, valvular heart disease, hearing impairment, corneal clouding, and spinal cord compression. Intellect is normal unless spinal cord compression leads to central nervous system compromise. DIAGNOSIS/TESTING The diagnosis of GLB1-related disorders is suspected in individuals with characteristic clinical, neuroimaging, radiographic, and biochemical findings. The diagnosis is confirmed by either deficiency of $\beta$-galactosidase enzyme activity or biallelic pathogenic variants in GLB1. MANAGEMENT Treatment of manifestations: Best provided by specialists in biochemical genetics, cardiology, orthopedics, and neurology and therapists knowledgeable about GLB1-related disorders; surgery is best performed in centers with surgeons and anesthesiologists experienced in the care of individuals with lysosomal storage disorders; occupational therapy to optimize activities of daily living (including adaptive equipment) and physical therapy to optimize gait and mobility (including orthotics and bracing); early and ongoing interventions to optimize educational and social outcomes. For those with GM1 gangliosidosis: Adequate nutrition to maintain growth; speech therapy to optimize oral motor skills; aggressive seizure control; routine management of risk of aspiration, risk of chronic urinary tract infection, and cardiac involvement; when disease is advanced: hospice services for supportive in-home care. Prevention of secondary complications: Anesthetic precautions to anticipate and manage complications relating to skeletal involvement and airway compromise; routine immunization; bacterial endocarditis prophylaxis in those with cardiac valvular disease. Surveillance: GM1 gangliosidosis: Routine monitoring of growth and nutrition. Assess yearly: quality of life including history and physical examination; seizure risk by a neurologist; cervical spine stability; and hip dislocation risk. Perform every one to three years: electrocardiogram and echocardiogram; eye examination. MPS IVB: Yearly: perform endurance tests to evaluate functional status of the cardiovascular, pulmonary, musculoskeletal, and nervous systems; assess lower extremities for malalignment, hips for dysplasia/subluxation, thoracolumbar spine for kyphosis, and cervical spine for instability; perform eye examination and audiogram. Perform electrocardiogram and echocardiogram every one to three years depending on disease course; assess for obstructive sleep apnea and restrictive lung disease; monitor nutritional status using MPS IVA-specific growth charts. Agents/circumstances to avoid: Psychotropic medications because of the risk of worsening neurologic disease; obesity in those with skeletal dysplasia GENETIC COUNSELING GLB1-related disorders are inherited in an autosomal recessive manner. Each sib of an affected individual has a 25{\%} chance of being affected, a 50{\%} chance of being an asymptomatic carrier, and a 25{\%} chance of being unaffected and not a carrier. Carrier testing for at-risk family members and prenatal diagnosis for pregnancies at increased risk are possible if the pathogenic variants in the family have been identified.},
author = {Regier, Debra S and Tifft, Cynthia J},
booktitle = {GeneReviews{\textregistered}},
keywords = {Beta,GLB1,GM1 Gangliosidosis,Mucopolysaccharidosis Type IVB,Related Disorders,galactosidase},
month = {aug},
pmid = {24156116},
publisher = {University of Washington, Seattle},
title = {{GLB1-Related Disorders}},
url = {http://www.ncbi.nlm.nih.gov/pubmed/24156116},
year = {1993}
}
@article{Ritchie2015,
abstract = {limma is an R/Bioconductor software package that provides an integrated solution for analysing data from gene expression experiments. It contains rich features for handling complex experimental designs and for information borrowing to overcome the problem of small sample sizes. Over the past decade, limma has been a popular choice for gene discovery through differential expression analyses of microarray and high-throughput PCR data. The package contains particularly strong facilities for reading, normalizing and exploring such data. Recently, the capabilities of limma have been significantly expanded in two important directions. First, the package can now perform both differential expression and differential splicing analyses of RNA sequencing (RNA-seq) data. All the downstream analysis tools previously restricted to microarray data are now available for RNA-seq as well. These capabilities allow users to analyse both RNA-seq and microarray data with very similar pipelines. Second, the package is now able to go past the traditional gene-wise expression analyses in a variety of ways, analysing expression profiles in terms of co-regulated sets of genes or in terms of higher-order expression signatures. This provides enhanced possibilities for biological interpretation of gene expression differences. This article reviews the philosophy and design of the limma package, summarizing both new and historical features, with an emphasis on recent enhancements and features that have not been previously described.},
author = {Ritchie, Matthew E. and Phipson, Belinda and Wu, Di and Hu, Yifang and Law, Charity W. and Shi, Wei and Smyth, Gordon K.},
doi = {10.1093/nar/gkv007},
issn = {13624962},
journal = {Nucleic Acids Research},
pmid = {25605792},
title = {{Limma powers differential expression analyses for RNA-sequencing and microarray studies}},
year = {2015}
}
@article{Ritchie2016,
abstract = {Network modules—topologically distinct groups of edges and nodes—that are preserved across datasets can reveal common features of organisms, tissues, cell types, and molecules. Many statistics to identify such modules have been developed, but testing their significance requires heuristics. Here, we demonstrate that current methods for assessing module preservation are systematically biased and produce skewed p values. We introduce NetRep, a rapid and computationally efficient method that uses a permutation approach to score module preservation without assuming data are normally distributed. NetRep produces unbiased p values and can distinguish between true and false positives during multiple hypothesis testing. We use NetRep to quantify preservation of gene coexpression modules across murine brain, liver, adipose, and muscle tissues. Complex patterns of multi-tissue preservation were revealed, including a liver-derived housekeeping module that displayed adipose- and muscle-specific association with body weight. Finally, we demonstrate the broader applicability of NetRep by quantifying preservation of bacterial networks in gut microbiota between men and women.},
author = {Ritchie, Scott C. and Watts, Stephen and Fearnley, Liam G. and Holt, Kathryn E. and Abraham, Gad and Inouye, Michael},
doi = {10.1016/j.cels.2016.06.012},
file = {::},
issn = {24054720},
journal = {Cell Systems},
month = {jul},
number = {1},
pages = {71--82},
pmid = {27467248},
publisher = {Cell Press},
title = {{A Scalable Permutation Approach Reveals Replication and Preservation Patterns of Network Modules in Large Datasets}},
url = {http://www.cell.com/article/S2405471216302174/fulltext http://www.cell.com/article/S2405471216302174/abstract https://www.cell.com/cell-systems/abstract/S2405-4712(16)30217-4},
volume = {3},
year = {2016}
}
@article{Robinson2009,
abstract = {It is expected that emerging digital gene expression (DGE) technologies will overtake microarray technologies in the near future for many functional genomics applications. One of the fundamental data analysis tasks, especially for gene expression studies, involves determining whether there is evidence that counts for a transcript or exon are significantly different across experimental conditions. edgeR is a Bioconductor software package for examining differential expression of replicated count data. An overdispersed Poisson model is used to account for both biological and technical variability. Empirical Bayes methods are used to moderate the degree of overdispersion across transcripts, improving the reliability of inference. The methodology can be used even with the most minimal levels of replication, provided at least one phenotype or experimental condition is replicated. The software may have other applications beyond sequencing data, such as proteome peptide count data. {\textcopyright} The Author(s) 2009. Published by Oxford University Press.},
author = {Robinson, Mark D. and McCarthy, Davis J. and Smyth, Gordon K.},
doi = {10.1093/bioinformatics/btp616},
issn = {14602059},
journal = {Bioinformatics},
pmid = {19910308},
title = {{edgeR: A Bioconductor package for differential expression analysis of digital gene expression data}},
year = {2009}
}
@incollection{Rodriguiz2006,
abstract = {Although most behavioral experiments have been conducted in rats, mice are rapidly becoming the preferred rodent of study in many labs because their genetics are well known, their genome has been sequenced, and they can be genetically manipulated. To date, several different approaches have been used to generate a behavioral phenotype for study.},
author = {Rodriguiz, Ramona Marie and Wetsel, William C.},
booktitle = {Animal Models of Cognitive Impairment},
doi = {10.1201/9781420004335.ch12},
isbn = {9781420004335},
month = {jan},
pages = {223--282},
pmid = {21204369},
publisher = {CRC Press},
title = {{Assessments of cognitive deficits in mutant mice}},
year = {2006}
}
@article{Ropers2011,
abstract = {High-throughput sequencing has greatly facilitated the elucidation of genetic disorders, but compared with X-linked and autosomal dominant diseases, the search for genetic defects underlying autosomal recessive diseases still lags behind. In a large consanguineous family with autosomal recessive intellectual disability (ARID), we have combined homozygosity mapping, targeted exon enrichment and high-throughput sequencing to identify the underlying gene defect. After appropriate single-nucleotide polymorphism filtering, only two molecular changes remained, including a non-synonymous sequence change in the SWIP [Strumpellin and WASH (Wiskott-Aldrich syndrome protein and scar homolog)-interacting protein] gene, a member of the recently discovered WASH complex, which is involved in actin polymerization and multiple endosomal transport processes. Based on high pathogenicity and evolutionary conservation scores as well as functional considerations, this gene defect was considered as causative for ID in this family. In line with this assumption, we could show that this mutation leads to significantly reduced SWIP levels and to destabilization of the entire WASH complex. Thus, our findings suggest that SWIP is a novel gene for ARID.},
author = {Ropers, Fabienne and Derivery, Emmanuel and Hu, Hao and Garshasbi, Masoud and Karbasiyan, Mohsen and Herold, Martin and N{\"{u}}rnberg, Gudrun and Ullmann, Reinhard and Gautreau, Alexis and Sperling, Karl and Varon, Raymonda and Rajab, Anna},
doi = {10.1093/hmg/ddr158},
isbn = {1460-2083 (Electronic)$\backslash$r0964-6906 (Linking)},
issn = {09646906},
journal = {Human Molecular Genetics},
pmid = {21498477},
title = {{Identification of a novel candidate gene for non-syndromic autosomal recessive intellectual disability: The WASH complex member swip}},
year = {2011}
}
@article{Rosenbaum2014,
abstract = {As newly synthesized glycoproteins move through the secretory pathway, the asparagine-linked glycan (N-glycan) undergoes extensive modifications involving the sequential removal and addition of sugar residues. These modifications are critical for the proper assembly, quality control and transport of glycoproteins during biosynthesis. The importance of N-glycosylation is illustrated by a growing list of diseases that result from defects in the biosynthesis and processing of N-linked glycans. The major rhodopsin in Drosophila melanogaster photoreceptors, Rh1, is highly unique among glycoproteins, as the N-glycan appears to be completely removed during Rh1 biosynthesis and maturation. However, much of the deglycosylation pathway for Rh1 remains unknown. To elucidate the key steps in Rh1 deglycosylation in vivo, we characterized mutant alleles of four Drosophila glycosyl hydrolases, namely $\alpha$-mannosidase-II ($\alpha$-Man-II), $\alpha$-mannosidase-IIb ($\alpha$-Man-IIb), a $\beta$-N-acetylglucosaminidase called fused lobes (Fdl), and hexosaminidase 1 (Hexo1). We have demonstrated that these four enzymes play essential and unique roles in a highly coordinated pathway for oligosaccharide trimming during Rh1 biosynthesis. Our results reveal that $\alpha$-Man-II and $\alpha$-Man-IIb are not isozymes like their mammalian counterparts, but rather function at distinct stages in Rh1 maturation. Also of significance, our results indicate that Hexo1 has a biosynthetic role in N-glycan processing during Rh1 maturation. This is unexpected given that in humans, the hexosaminidases are typically lysosomal enzymes involved in N-glycan catabolism with no known roles in protein biosynthesis. Here, we present a genetic dissection of glycoprotein processing in Drosophila and unveil key steps in N-glycan trimming during Rh1 biosynthesis. Taken together, our results provide fundamental advances towards understanding the complex and highly regulated pathway of N-glycosylation in vivo and reveal novel insights into the functions of glycosyl hydrolases in the secretory pathway. {\textcopyright} 2014 Rosenbaum et al.},
author = {Rosenbaum, Erica E. and Vasiljevic, Eva and Brehm, Kimberley S. and Colley, Nansi Jo},
doi = {10.1371/journal.pgen.1004349},
file = {::},
issn = {15537404},
journal = {PLoS Genetics},
number = {5},
pmid = {24785692},
publisher = {Public Library of Science},
title = {{Mutations in Four Glycosyl Hydrolases Reveal a Highly Coordinated Pathway for Rhodopsin Biosynthesis and N-Glycan Trimming in Drosophila melanogaster}},
url = {/pmc/articles/PMC4006722/?report=abstract https://www.ncbi.nlm.nih.gov/pmc/articles/PMC4006722/},
volume = {10},
year = {2014}
}
@article{Schroder1994,
abstract = {Genomic DNA and cDNA from fibroblasts from nine unrelated German patients with X‐linked iduronate‐2‐sulfatase (IDS) deficiency showing variable clinical manifestation were screened for point mutations and small structural aberrations. Direct sequencing revealed a splice mutation skipping exon A, one nonsense mutation, and five missense mutations concerning the exons B, F and I of the IDS gene. Several novel missense mutations were found: A68E, S426X, I485R, Q293H, and D478G. One of the point mutations eliminating a recognition site for the restriction enzyme MspI was used as a direct marker for a prenatal diagnosis. A relationship between type of mutation and clinical picture could not be recognized. {\textcopyright} 1994 Wiley‐Liss, Inc. Copyright {\textcopyright} 1994 Wiley‐Liss, Inc., A Wiley Company},
author = {Schr{\"{o}}der, Winnie and Wulff, Karin and Wehnert, Manfred and Seidlitz, G{\"{u}}nter and Herrmann, Falko H},
doi = {10.1002/humu.1380040206},
issn = {10981004},
journal = {Human Mutation},
keywords = {Direct sequencing,Iduronate‐2‐Sulfatase gene,MPS II,Point mutations},
number = {2},
pages = {128--131},
pmid = {7981716},
title = {{Mutations of the iduronate‐2‐sulfatase (IDS) gene in patients with hunter syndrome (mucopolysaccharidosis II)}},
url = {http://www.ncbi.nlm.nih.gov/pubmed/7981716},
volume = {4},
year = {1994}
}
@article{Seaman2014,
abstract = {The retromer complex mediates endosomal protein sorting by concentrating membrane proteins (cargo) into nascent tubules formed through the action of sorting nexin (SNX) proteins. The WASH complex is recruited to endosomes by binding to the VPS35 subunit of retromer and facilitates cargo protein sorting by promoting formation of endosomally-localized F-actin. The VPS35 protein is mutated in Parkinson disease (PD) and a recent report has revealed that the PD-causing mutation impairs the association of retromer with the WASH complex leading to perturbed endosomal protein sorting. Another important player in endosomal protein sorting is the DNAJC13/RME-8 protein, which associates with SNX1 and has also recently been linked to PD. An additional recent report has now shown that RME-8 also interacts with the WASH complex thus establishing retromer and WASH complex-mediated endosomal protein sorting as a key pathway linked to the pathology of PD and providing new avenues to explore in the search for insights into the disease mechanism.},
author = {Seaman, Matthew NJ and Freeman, Caroline L},
doi = {10.4161/cib.29483},
isbn = {1942-0889 (Electronic)$\backslash$r1942-0889 (Linking)},
issn = {1942-0889},
journal = {Communicative {\&} Integrative Biology},
pmid = {25067992},
title = {{Analysis of the Retromer complex-WASH complex interaction illuminates new avenues to explore in Parkinson disease}},
year = {2014}
}
@article{Seshadri2010,
abstract = {Context: Genome-wide association studies (GWAS) have recently identified CLU, PICALM, and CR1 as novel genes for late-onset Alzheimer disease (AD). Objectives: To identify and strengthen additional loci associated with AD and confirm these in an independent sample and to examine the contribution of recently identified genes to AD risk prediction in a 3-stage analysis of new and previously published GWAS on more than 35 000 persons (8371 AD cases). Design, Setting, and Participants: In stage 1, we identified strong genetic associations (P{\textless}10-3) in a sample of 3006 AD cases and 14 642 controls by combining new data from the population-based Cohorts for Heart and Aging Research in Genomic Epidemiology consortium (1367 AD cases [973 incident]) with previously reported results from the Translational Genomics Research Institute and the Mayo AD GWAS. We identified 2708 single-nucleotide polymorphisms (SNPs) with P{\textless}10-3. In stage 2, we pooled results for these SNPs with the European AD Initiative (2032 cases and 5328 controls) to identify 38 SNPs (10 loci) with P{\textless}10-5. In stage 3, we combined data for these 10 loci with data from the Genetic and Environmental Risk in AD consortium (3333 cases and 6995 controls) to identify 4 SNPs with P{\textless}1.7 × 10-8. These 4 SNPs were replicated in an independent Spanish sample (1140 AD cases and 1209 controls). Genome-wide association analyses were completed in 2007-2008 and the meta-analyses and replication in 2009. Main Outcome Measure: Presence of Alzheimer disease. Results: Two loci were identified to have genome-wide significance for the first time: rs744373 near BIN1 (odds ratio [OR],1.13; 95{\%} confidence interval [CI],1.06-1.21 per copy of the minor allele; P=1.59×10-11) and rs597668 near EXOC3L2/BLOC1S3/ MARK4 (OR, 1.18; 95{\%} CI, 1.07-1.29; P=6.45×10-9). Associations of these 2 loci plus the previously identified loci CLU and PICALM with AD were confirmed in the Spanish sample (P{\textless}.05). However, although CLU and PICALM were confirmed to be associated with AD in this independent sample, they did not improve the ability of a model that included age, sex, and APOE to predict incident AD (improvement in area under the receiver operating characteristic curve from 0.847 to 0.849 in the Rotterdam Study and 0.702 to 0.705 in the Cardiovascular Health Study). Conclusions: Two genetic loci for AD were found for the first time to reach genome-wide statistical significance. These findings were replicated in an independent population. Two recently reported associations were also confirmed. These loci did not improve AD risk prediction. While not clinically useful, they may implicate biological pathways useful for future research. {\textcopyright}2010 American Medical Association. All rights reserved.},
author = {Seshadri, Sudha and Fitzpatrick, Annette L. and Ikram, M. Arfan and DeStefano, Anita L. and Gudnason, Vilmundur and Boada, Merce and Bis, Joshua C. and Smith, Albert V. and Carassquillo, Minerva M. and Lambert, Jean Charles and Harold, Denise and Schrijvers, Elisabeth M.C. and Ramirez-Lorca, Reposo and Debette, Stephanie and Longstreth, W. T. and Janssens, A. Cecile J.W. and Pankratz, V. Shane and Dartigues, Jean Fran{\c{c}}ois and Hollingworth, Paul and Aspelund, Thor and Hernandez, Isabel and Beiser, Alexa and Kuller, Lewis H. and Koudstaal, Peter J. and Dickson, Dennis W. and Tzourio, Christophe and Abraham, Richard and Antunez, Carmen and Du, Yangchun and Rotter, Jerome I. and Aulchenko, Yurii S. and Harris, Tamara B. and Petersen, Ronald C. and Berr, Claudine and Owen, Michael J. and Lopez-Arrieta, Jesus and Varadarajan, Badri N. and Becker, James T. and Rivadeneira, Fernando and Nalls, Michael A. and Graff-Radford, Neill R. and Campion, Dominique and Auerbach, Sanford and Rice, Kenneth and Hofman, Albert and Jonsson, Palmi V. and Schmidt, Helena and Lathrop, Mark and Mosley, Thomas H. and Au, Rhoda and Psaty, Bruce M. and Uitterlinden, Andre G. and Farrer, Lindsay A. and Lumley, Thomas and Ruiz, Agustin and Williams, Julie and Amouyel, Philippe and Younkin, Steve G. and Wolf, Philip A. and Launer, Lenore J. and Lopez, Oscar L. and {Van Duijn}, Cornelia M. and Breteler, Monique M.B.},
doi = {10.1001/jama.2010.574},
file = {::},
issn = {00987484},
journal = {JAMA - Journal of the American Medical Association},
month = {may},
number = {18},
pages = {1832--1840},
pmid = {20460622},
title = {{Genome-wide analysis of genetic loci associated with Alzheimer disease}},
volume = {303},
year = {2010}
}
@article{Seyfried2017,
abstract = {Here, we report proteomic analyses of 129 human cortical tissues to define changes associated with the asymptomatic and symptomatic stages of Alzheimer's disease (AD). Network analysis revealed 16 modules of co-expressed proteins, 10 of which correlated with AD phenotypes. A subset of modules overlapped with RNA co-expression networks, including those associated with neurons and astroglial cell types, showing altered expression in AD, even in the asymptomatic stages. Overlap of RNA and protein networks was otherwise modest, with many modules specific to the proteome, including those linked to microtubule function and inflammation. Proteomic modules were validated in an independent cohort, demonstrating some module expression changes unique to AD and several observed in other neurodegenerative diseases. AD genetic risk loci were concentrated in glial-related modules in the proteome and transcriptome, consistent with their causal role in AD. This multi-network analysis reveals protein- and disease-specific pathways involved in the etiology, initiation, and progression of AD.},
author = {Seyfried, Nicholas T. and Dammer, Eric B. and Swarup, Vivek and Nandakumar, Divya and Duong, Duc M. and Yin, Luming and Deng, Qiudong and Nguyen, Tram and Hales, Chadwick M. and Wingo, Thomas and Glass, Jonathan and Gearing, Marla and Thambisetty, Madhav and Troncoso, Juan C. and Geschwind, Daniel H. and Lah, James J. and Levey, Allan I.},
doi = {10.1016/j.cels.2016.11.006},
file = {::},
issn = {24054720},
journal = {Cell Systems},
month = {jan},
number = {1},
pages = {60--72.e4},
pmid = {27989508},
publisher = {Cell Press},
title = {{A Multi-network Approach Identifies Protein-Specific Co-expression in Asymptomatic and Symptomatic Alzheimer's Disease}},
url = {http://www.cell.com/article/S2405471216303702/fulltext http://www.cell.com/article/S2405471216303702/abstract https://www.cell.com/cell-systems/abstract/S2405-4712(16)30370-2},
volume = {4},
year = {2017}
}
@misc{Simonetti2019,
abstract = {Endosomes constitute major sorting compartments within the cell. There, a myriad of transmembrane proteins (cargoes) are delivered to the lysosome for degradation or retrieved from this fate and recycled through tubulo-vesicular transport carriers to different cellular destinations. Retrieval and recycling are orchestrated by multi-protein assemblies that include retromer and retriever, sorting nexins, and the Arp2/3 activating WASH complex. Fine-tuned control of actin polymerization on endosomes is fundamental for the retrieval and recycling of cargoes. Recent advances in the field have highlighted several roles that actin plays in this process including the binding to cargoes, stabilization of endosomal subdomains, generation of the remodeling forces required for the biogenesis of cargo-enriched transport carriers and short-range motility of the transport carriers.},
annote = {From Duplicate 1 (Actin-dependent endosomal receptor recycling - Simonetti, Boris; Cullen, Peter J.)

Really good current review on actin regulation in endosomes, good section on WASH},
author = {Simonetti, Boris and Cullen, Peter J.},
booktitle = {Current Opinion in Cell Biology},
doi = {10.1016/j.ceb.2018.08.006},
isbn = {09550674},
issn = {18790410},
pmid = {30227382},
title = {{Actin-dependent endosomal receptor recycling}},
year = {2019}
}
@article{Simonetti2017,
abstract = {Endosomal recycling of transmembrane proteins requires sequence-dependent recognition of motifs present within their intracellular cytosolic domains. In this study, we have reexamined the role of retromer in the sequence-dependent endosome- to-trans-Golgi network (TGN) transport of the cation-independent mannose 6-phosphate receptor (CI-MPR). Although the knockdown or knockout of retromer does not perturb CI-MPR transport, the targeting of the retromer-linked sorting nexin (SNX)-Bin, Amphiphysin, and Rvs (BAR) proteins leads to a pronounced defect in CI-MPR endosome-to-TGN transport. The retromer-linked SNX-BAR proteins comprise heterodimeric combinations of SNX1 or SNX2 with SNX5 or SNX6 and serve to regulate the biogenesis of tubular endosomal sorting profiles. We establish that SNX5 and SNX6 associate with the CI-MPR through recognition of a specific WLM endosome-to-TGN sorting motif. From validating the CI-MPR dependency of SNX1/2-SNX5/6 tubular profile formation, we provide a mechanism for coupling sequencedependent cargo recognition with the biogenesis of tubular profiles required for endosome-to-TGN transport. Therefore, the data presented in this study reappraise retromer's role in CI-MPR transport.},
author = {Simonetti, Boris and Danson, Chris M. and Heesom, Kate J. and Cullen, Peter J.},
doi = {10.1083/jcb.201703015},
file = {::},
issn = {15408140},
journal = {Journal of Cell Biology},
month = {nov},
number = {11},
pages = {3695--3712},
pmid = {28935633},
publisher = {Rockefeller University Press},
title = {{Sequence-dependent cargo recognition by SNX-BARs mediates retromer-independent transport of CI-MPR}},
volume = {216},
year = {2017}
}
@article{Singla2019,
abstract = {Protein recycling through the endolysosomal system relies on molecular assemblies that interact with cargo proteins, membranes, and effector molecules. Among them, the COMMD/CCDC22/CCDC93 (CCC) complex plays a critical role in recycling events. While CCC is closely associated with retriever, a cargo recognition complex, its mechanism of action remains unexplained. Herein we show that CCC and retriever are closely linked through sharing a common subunit (VPS35L), yet the integrity of CCC, but not retriever, is required to maintain normal endosomal levels of phosphatidylinositol-3-phosphate (PI(3)P). CCC complex depletion leads to elevated PI(3)P levels, enhanced recruitment and activation of WASH (an actin nucleation promoting factor), excess endosomal F-actin and trapping of internalized receptors. Mechanistically, we find that CCC regulates the phosphorylation and endosomal recruitment of the PI(3)P phosphatase MTMR2. Taken together, we show that the regulation of PI(3)P levels by the CCC complex is critical to protein recycling in the endosomal compartment.},
author = {Singla, Amika and Fedoseienko, Alina and Giridharan, Sai S.P. and Overlee, Brittany L. and Lopez, Adam and Jia, Da and Song, Jie and Huff-Hardy, Kayci and Weisman, Lois and Burstein, Ezra and Billadeau, Daniel D.},
doi = {10.1038/s41467-019-12221-6},
file = {::},
issn = {20411723},
journal = {Nature Communications},
month = {dec},
number = {1},
pmid = {31537807},
publisher = {Nature Publishing Group},
title = {{Endosomal PI(3)P regulation by the COMMD/CCDC22/CCDC93 (CCC) complex controls membrane protein recycling}},
volume = {10},
year = {2019}
}
@article{Slosarek2018,
abstract = {Length-dependent axonopathy of the corticospinal tract causes lower limb spasticity and is characteristic of several neurological disorders, including hereditary spastic paraplegia (HSP) and amyotrophic lateral sclerosis. Mutations in Trk-fused gene (TFG) have been implicated in both diseases, but the pathomechanisms by which these alterations cause neuropathy remain unclear. Here, we biochemically and genetically define the impact of a mutation within the TFG coiled-coil domain, which underlies early-onset forms of HSP. We find that the TFG (p.R106C) mutation alters compaction of TFG ring complexes, which play a critical role in the export of cargoes from the endoplasmic reticulum (ER). Using CRISPR-mediated genome editing, we engineered human stem cells that express the mutant form of TFG at endogenous levels and identified specific defects in secretion from the ER and axon fasciculation following neuronal differentiation. Together, our data highlight a key role for TFG-mediated protein transport in the pathogenesis of HSP. Slosarek et al. demonstrate that pathological mutations in TFG, which underlie various forms of neurodegenerative disease, impair secretory protein transport from the endoplasmic reticulum and compromise the ability of axons to self-associate. These findings highlight a critical function for the early secretory pathway in neuronal maintenance.},
author = {Slosarek, Erin L. and Schuh, Amber L. and Pustova, Iryna and Johnson, Adam and Bird, Jennifer and Johnson, Matthew and Frankel, E. B. and Bhattacharya, Nilakshee and Hanna, Michael G. and Burke, Jordan E. and Ruhl, David A. and Quinney, Kyle and Block, Samuel and Peotter, Jennifer L. and Chapman, Edwin R. and Sheets, Michael D. and Butcher, Samuel E. and Stagg, Scott M. and Audhya, Anjon},
doi = {10.1016/j.celrep.2018.07.081},
file = {::},
issn = {22111247},
journal = {Cell Reports},
keywords = {COPII,L1CAM,Trk-fused gene,axon bundling,early secretory pathway,hereditary spastic paraplegia,neurodegeneration,vesicle trafficking},
month = {aug},
number = {9},
pages = {2248--2260},
pmid = {30157421},
publisher = {Elsevier B.V.},
title = {{Pathogenic TFG Mutations Underlying Hereditary Spastic Paraplegia Impair Secretory Protein Trafficking and Axon Fasciculation}},
url = {http://www.cell.com/article/S2211124718311999/fulltext http://www.cell.com/article/S2211124718311999/abstract https://www.cell.com/cell-reports/abstract/S2211-1247(18)31199-9},
volume = {24},
year = {2018}
}
@article{Sun2019,
abstract = {Hsp70 and Hsp90 chaperones are critical for protein quality control in the cytosol, whereas organelle-specific Hsp70/Hsp90 paralogs provide similar protection for mitochondria and the endoplasmic reticulum (ER). Cytosolic Hsp70/Hsp90 can operate sequentially with Hsp90 selectively associating with Hsp70 after Hsp70 is bound to a client protein. This observation has long suggested that Hsp90 could have a preference for interacting with clients at their later stages of folding. However, recent work has shown that cytosolic Hsp70/Hsp90 can directly interact even in the absence of a client, which opens up an alternative possibility that the ordered interactions of Hsp70/Hsp90 with clients could be a consequence of regulated changes in the direct interactions between Hsp70 and Hsp90. However, it is unknown how such regulation could occur mechanistically. Here, we find that the ER Hsp70/Hsp90 (BiP/Grp94) can form a direct complex in the absence of a client. Importantly, the direct interaction between BiP and Grp94 is nucleotide-specific, with BiP and Grp94 having higher affinity under ADP conditions and lower affinity under ATP conditions. We show that this nucleotidespecific association between BiP and Grp94 is largely due to the conformation of BiP. When BiP is in the ATP conformation its substrate-binding domain blocks Grp94; in contrast, Grp94 can readily associate with the ADP conformation of BiP, which represents the client-bound state of BiP. Our observations provide a mechanism for the sequential involvement of BiP and Grp94 in client folding where the conformation of BiP provides the signal for the subsequent recruitment of Grp94.},
author = {Sun, Ming and Kotler, J. L.M. and Liu, Shanshan and Street, Timothy O.},
doi = {10.1074/jbc.RA118.007050},
issn = {1083351X},
journal = {Journal of Biological Chemistry},
month = {apr},
number = {16},
pages = {6387--6396},
pmid = {30787103},
publisher = {American Society for Biochemistry and Molecular Biology Inc.},
title = {{The endoplasmic reticulum (ER) chaperones BiP and Grp94 selectively associate when BiP is in the ADP conformation}},
volume = {294},
year = {2019}
}
@article{Synofzik2014,
abstract = {Diabetes mellitus and neurodegeneration are common diseases for which shared genetic factors are still only partly known. Here, we show that loss of the BiP (immunoglobulin heavy-chain binding protein) co-chaperone DNAJC3 leads to diabetes mellitus and widespread neurodegeneration. We investigated three siblings with juvenile-onset diabetes and central and peripheral neurodegeneration, including ataxia, upper-motor-neuron damage, peripheral neuropathy, hearing loss, and cerebral atrophy. Exome sequencing identified a homozygous stop mutation in DNAJC3. Screening of a diabetes database with 226,194 individuals yielded eight phenotypically similar individuals and one family carrying a homozygous DNAJC3 deletion. DNAJC3 was absent in fibroblasts from all affected subjects in both families. To delineate the phenotypic and mutational spectrum and the genetic variability of DNAJC3, we analyzed 8,603 exomes, including 506 from families affected by diabetes, ataxia, upper-motor-neuron damage, peripheral neuropathy, or hearing loss. This analysis revealed only one further loss-of-function allele in DNAJC3 and no further associations in subjects with only a subset of the features of the main phenotype. Our findings demonstrate that loss-of-function DNAJC3 mutations lead to a monogenic, recessive form of diabetes mellitus in humans. Moreover, they present a common denominator for diabetes and widespread neurodegeneration. This complements findings from mice in which knockout of Dnajc3 leads to diabetes and modifies disease in a neurodegenerative model of Marinesco-Sj{\"{o}}gren syndrome.},
author = {Synofzik, Matthis and Haack, Tobias B. and Kopajtich, Robert and Gorza, Matteo and Rapaport, Doron and Greiner, Markus and Sch{\"{o}}nfeld, Caroline and Freiberg, Clemens and Schorr, Stefan and Holl, Reinhard W. and Gonzalez, Michael A. and Fritsche, Andreas and Fallier-Becker, Petra and Zimmermann, Richard and Strom, Tim M. and Meitinger, Thomas and Z{\"{u}}chner, Stephan and Sch{\"{u}}le, Rebecca and Sch{\"{o}}ls, Ludger and Prokisch, Holger},
doi = {10.1016/j.ajhg.2014.10.013},
file = {::},
issn = {15376605},
journal = {American Journal of Human Genetics},
month = {dec},
number = {6},
pages = {689--697},
pmid = {25466870},
publisher = {Cell Press},
title = {{Absence of BiP Co-chaperone DNAJC3 causes diabetes mellitus and multisystemic neurodegeneration}},
volume = {95},
year = {2014}
}
@article{Tachibana2019,
abstract = {Carrying the $\epsilon$4 allele of the APOE gene encoding apolipoprotein E (APOE4) markedly increases the risk for late-onset Alzheimer's disease (AD), in which APOE4 exacerbates the brain accumulation and subsequent deposition of amyloid-$\beta$ (A$\beta$) peptides. While the LDL receptor–related protein 1 (LRP1) is a major apoE receptor in the brain, we found that its levels are associated with those of insoluble A$\beta$ depending on APOE genotype status in postmortem AD brains. Thus, to determine the functional interaction of apoE4 and LRP1 in brain A$\beta$ metabolism, we crossed neuronal LRP1-knockout mice with amyloid model APP/PS1 mice and APOE3–targeted replacement (APO3-TR) or APOE4-TR mice. Consistent with previous findings, mice expressing apoE4 had increased A$\beta$ deposition and insoluble amounts of A$\beta$40 and A$\beta$42 in the hippocampus of APP/PS1 mice compared with those expressing apoE3. Intriguingly, such effects were reversed in the absence of neuronal LRP1. Neuronal LRP1 deficiency also increased detergent-soluble apoE4 levels, which may contribute to the inhibition of A$\beta$ deposition. Together, our results suggest that apoE4 exacerbates A$\beta$ pathology through a mechanism that depends on neuronal LRP1. A better understanding of apoE isoform–specific interaction with their metabolic receptor LRP1 on A$\beta$ metabolism is crucial for defining APOE4-related risk for AD.},
author = {Tachibana, Masaya and Holm, Marie Louise and Liu, Chia Chen and Shinohara, Mitsuru and Aikawa, Tomonori and Oue, Hiroshi and Yamazaki, Yu and Martens, Yuka A. and Murray, Melissa E. and Sullivan, Patrick M. and Weyer, Kathrin and Glerup, Simon and Dickson, Dennis W. and Bu, Guojun and Kanekiyo, Takahisa},
doi = {10.1172/JCI124853},
file = {::},
issn = {15588238},
journal = {Journal of Clinical Investigation},
month = {mar},
number = {3},
pages = {1272--1277},
publisher = {American Society for Clinical Investigation},
title = {{APOE4-mediated amyloid-$\beta$ pathology depends on its neuronal receptor LRP1}},
volume = {129},
year = {2019}
}
@article{Takamori2006,
abstract = {Membrane traffic in eukaryotic cells involves transport of vesicles that bud from a donor compartment and fuse with an acceptor compartment. Common principles of budding and fusion have emerged, and many of the proteins involved in these events are now known. However, a detailed picture of an entire trafficking organelle is not yet available. Using synaptic vesicles as a model, we have now determined the protein and lipid composition; measured vesicle size, density, and mass; calculated the average protein and lipid mass per vesicle; and determined the copy number of more than a dozen major constituents. A model has been constructed that integrates all quantitative data and includes structural models of abundant proteins. Synaptic vesicles are dominated by proteins, possess a surprising diversity of trafficking proteins, and, with the exception of the V-ATPase that is present in only one to two copies, contain numerous copies of proteins essential for membrane traffic and neurotransmitter uptake. {\textcopyright} 2006 Elsevier Inc. All rights reserved.},
author = {Takamori, Shigeo and Holt, Matthew and Stenius, Katinka and Lemke, Edward A. and Gr{\o}nborg, Mads and Riedel, Dietmar and Urlaub, Henning and Schenck, Stephan and Br{\"{u}}gger, Britta and Ringler, Philippe and M{\"{u}}ller, Shirley A. and Rammner, Burkhard and Gr{\"{a}}ter, Frauke and Hub, Jochen S. and {De Groot}, Bert L. and Mieskes, Gottfried and Moriyama, Yoshinori and Klingauf, J{\"{u}}rgen and Grubm{\"{u}}ller, Helmut and Heuser, John and Wieland, Felix and Jahn, Reinhard},
doi = {10.1016/j.cell.2006.10.030},
issn = {00928674},
journal = {Cell},
pmid = {17110340},
title = {{Molecular Anatomy of a Trafficking Organelle}},
year = {2006}
}
@article{Tanaka2017,
abstract = {Progranulin (PGRN) haploinsufficiency resulting from loss-of-function mutations in the PGRN gene causes frontotemporal lobar degeneration accompanied by TDP-43 accumulation, and patients with homozygous mutations in the PGRN gene present with neuronal ceroid lipofuscinosis. Although it remains unknown why PGRN deficiency causes neurodegenerative diseases, there is increasing evidence that PGRN is implicated in lysosomal functions. Here, we show PGRN is a secretory lysosomal protein that regulates lysosomal function and biogenesis by controlling the acidification of lysosomes. PGRN gene expression and protein levels increased concomitantly with the increase of lysosomal biogenesis induced by lysosome alkalizers or serum starvation. Down-regulation or insufficiency of PGRN led to the increased lysosomal gene expression and protein levels, while PGRN overexpression led to the decreased lysosomal gene expression and protein levels. In particular, the level of mature cathepsin D (CTSDmat) dramatically changed depending upon PGRN levels. The acidification of lysosomes was facilitated in cells transfected with PGRN. Then, this caused degradation of CTSDmat by cathepsin B. Secreted PGRN is incorporated into cells via sortilin or cation-independent mannose 6-phosphate receptor, and facilitated the acidification of lysosomes and degradation of CTSDmat. Moreover, the change of PGRN levels led to a cell-type-specific increase of insoluble TDP- 43. In the brain tissue of FTLD-TDP patients with PGRN deficiency, CTSD and phosphorylated TDP-43 accumulated in neurons. Our study provides new insights into the physiological function of PGRN and the role of PGRN insufficiency in the pathogenesis of neurodegenerative diseases.},
author = {Tanaka, Yoshinori and Suzuki, Genjiro and Matsuwaki, Takashi and Hosokawa, Masato and Serrano, Geidy and Beach, Thomas G. and Yamanouchi, Keitaro and Hasegawa, Masato and Nishihara, Masugi},
doi = {10.1093/hmg/ddx011},
issn = {14602083},
journal = {Human Molecular Genetics},
number = {5},
pages = {969--988},
title = {{Progranulin regulates lysosomal function and biogenesis through acidification of lysosomes}},
url = {https://www.ncbi.nlm.nih.gov/pubmed/28073925},
volume = {26},
year = {2017}
}
@article{Terman1998,
abstract = {Lipofuscin (age pigment) is a brown-yellow, electron-dense, autofluorescent material that accumulates progressively over time in lysosomes of postmitotic cells, such as neurons and cardiac myocytes. The exact mechanisms behind this accumulation are still unclear. This review outlines the present knowledge of age pigment formation, and considers possible mechanisms responsible for the increase of lipofuscin with age. Numerous studies indicate that the formation of lipofuscin is due to the oxidative alteration of macromolecules by oxygen-derived free radicals generated in reactions catalyzed by redox-active iron of low molecular weight. Two principal explanations for the increase of lipofuscin with age have been suggested. The first one is based on the notion that lipofuscin is not totally eliminated (either by degradation or exocytosis) even at young age, and, thus, accumulates in postmitotic cells as a function of time. Since oxidative reactions are obligatory for life, they would act as age-independent enhancers of lipofuscin accumulation, as well as of many other manifestations of senescence. The second explanation is that the increase of lipofuscin is an effect of aging, caused by an age-related enhancement of autophagocytosis, a decline in intralysosomal degradation, and/or a decrease in exocytosis.},
author = {Terman, Alexei and Brunk, Ulf T.},
doi = {10.1111/j.1699-0463.1998.tb01346.x},
issn = {09034641},
journal = {APMIS},
month = {jan},
number = {1-6},
pages = {265--276},
pmid = {9531959},
publisher = {Wiley},
title = {{Lipofuscin: Mechanisms of formation and increase with age}},
volume = {106},
year = {1998}
}
@article{Thul2017,
abstract = {Resolving the spatial distribution of the human proteome at a subcellular level can greatly increase our understanding of human biology and disease. Here we present a comprehensive image-based map of subcellular protein distribution, the Cell Atlas, built by integrating transcriptomics and antibody-based immunofluorescence microscopy with validation by mass spectrometry. Mapping the in situ localization of 12,003 human proteins at a single-cell level to 30 subcellular structures enabled the definition of the proteomes of 13 major organelles. Exploration of the proteomes revealed single-cell variations in abundance or spatial distribution and localization of about half of the proteins to multiple compartments. This subcellular map can be used to refine existing protein-protein interaction networks and provides an important resource to deconvolute the highly complex architecture of the human cell.},
author = {Thul, Peter J. and {\AA}kesson, Lovisa and Wiking, Mikaela and Mahdessian, Diana and Geladaki, Aikaterini and {Ait Blal}, Hammou and Alm, Tove and Asplund, Anna and Bj{\"{o}}rk, Lars and Breckels, Lisa M. and B{\"{a}}ckstr{\"{o}}m, Anna and Danielsson, Frida and Fagerberg, Linn and Fall, Jenny and Gatto, Laurent and Gnann, Christian and Hober, Sophia and Hjelmare, Martin and Johansson, Fredric and Lee, Sunjae and Lindskog, Cecilia and Mulder, Jan and Mulvey, Claire M. and Nilsson, Peter and Oksvold, Per and Rockberg, Johan and Schutten, Rutger and Schwenk, Jochen M. and Sivertsson, {\AA}sa and Sj{\"{o}}stedt, Evelina and Skogs, Marie and Stadler, Charlotte and Sullivan, Devin P. and Tegel, Hanna and Winsnes, Casper and Zhang, Cheng and Zwahlen, Martin and Mardinoglu, Adil and Pont{\'{e}}n, Fredrik and von Feilitzen, Kalle and Lilley, Kathryn S. and Uhl{\'{e}}n, Mathias and Lundberg, Emma},
doi = {10.1126/science.aal3321},
isbn = {10959203 (Electronic)},
issn = {0036-8075},
journal = {Science},
pmid = {28495876},
title = {{A subcellular map of the human proteome}},
year = {2017}
}
@article{Traag2015,
abstract = {Nodes in real-world networks are repeatedly observed to form dense clusters, often referred to as communities. Methods to detect these groups of nodes usually maximize an objective function, which implicitly contains the definition of a community. We here analyze a recently proposed measure called surprise, which assesses the quality of the partition of a network into communities. In its current form, the formulation of surprise is rather difficult to analyze. We here therefore develop an accurate asymptotic approximation. This allows for the development of an efficient algorithm for optimizing surprise. Incidentally, this leads to a straightforward extension of surprise to weighted graphs. Additionally, the approximation makes it possible to analyze surprise more closely and compare it to other methods, especially modularity. We show that surprise is (nearly) unaffected by the well-known resolution limit, a particular problem for modularity. However, surprise may tend to overestimate the number of communities, whereas they may be underestimated by modularity. In short, surprise works well in the limit of many small communities, whereas modularity works better in the limit of few large communities. In this sense, surprise is more discriminative than modularity and may find communities where modularity fails to discern any structure.},
archivePrefix = {arXiv},
arxivId = {1503.00445},
author = {Traag, V. A. and Aldecoa, R. and Delvenne, J. C.},
doi = {10.1103/PhysRevE.92.022816},
eprint = {1503.00445},
issn = {15502376},
journal = {Physical Review E - Statistical, Nonlinear, and Soft Matter Physics},
month = {aug},
number = {2},
pages = {022816},
publisher = {American Physical Society},
title = {{Detecting communities using asymptotical surprise}},
url = {https://journals.aps.org/pre/abstract/10.1103/PhysRevE.92.022816},
volume = {92},
year = {2015}
}
@article{Traag2019,
abstract = {Community detection is often used to understand the structure of large and complex networks. One of the most popular algorithms for uncovering community structure is the so-called Louvain algorithm. We show that this algorithm has a major defect that largely went unnoticed until now: the Louvain algorithm may yield arbitrarily badly connected communities. In the worst case, communities may even be disconnected, especially when running the algorithm iteratively. In our experimental analysis, we observe that up to 25{\%} of the communities are badly connected and up to 16{\%} are disconnected. To address this problem, we introduce the Leiden algorithm. We prove that the Leiden algorithm yields communities that are guaranteed to be connected. In addition, we prove that, when the Leiden algorithm is applied iteratively, it converges to a partition in which all subsets of all communities are locally optimally assigned. Furthermore, by relying on a fast local move approach, the Leiden algorithm runs faster than the Louvain algorithm. We demonstrate the performance of the Leiden algorithm for several benchmark and real-world networks. We find that the Leiden algorithm is faster than the Louvain algorithm and uncovers better partitions, in addition to providing explicit guarantees.},
archivePrefix = {arXiv},
arxivId = {1810.08473},
author = {Traag, V. A. and Waltman, L. and van Eck, N. J.},
doi = {10.1038/s41598-019-41695-z},
eprint = {1810.08473},
file = {::},
issn = {20452322},
journal = {Scientific Reports},
keywords = {Applied mathematics,Computational science,Computer science},
month = {dec},
number = {1},
pages = {1--12},
pmid = {30914743},
publisher = {Nature Publishing Group},
title = {{From Louvain to Leiden: guaranteeing well-connected communities}},
url = {https://www.nature.com/articles/s41598-019-41695-z},
volume = {9},
year = {2019}
}
@article{Buneman2016,
author = {Uezu, Akiyoshi and Kanak, Daniel J and Bradshaw, Tyler WA and Soderblom, Erin J and Catavero, Christina M and Burette, Alain C and Weinberg, Richard J and Soderling, Scott H},
doi = {10.1126/science.aag0821},
file = {:Users/student/Library/Application Support/Mendeley Desktop/Downloaded/Uezu et al. - 2016 - Identification of an elaborate complex mediating postsynaptic inhibition.pdf:pdf},
issn = {0036-8075},
number = {6304},
pages = {960--962},
pmid = {27609886},
title = {{Identification of an elaborate complex mediating postsynaptic inhibition}},
volume = {353},
year = {2016}
}
@article{Valdez2017,
abstract = {Frontotemporal dementia (FTD) encompasses a group of neurodegenerative disorders characterized by cognitive and behavioral impairments. Heterozygous mutations in progranulin (PGRN) cause familial FTD and result in decreased PGRN expression, while homozygous mutations result in complete loss of PGRN expression and lead to the neurodegenerative lysosomal storage disorder neuronal ceroid lipofuscinosis (NCL). However, how dose-dependent PGRN mutations contribute to these two different diseases is not well understood. Using iPSC-derived human cortical neurons from FTD patients harboring PGRN mutations, we demonstrate that PGRN mutant neurons exhibit decreased nuclear TDP-43 and increased insoluble TDP-43, as well as enlarged electron-dense vesicles, lipofuscin accumulation, fingerprint-like profiles and granular osmiophilic deposits, suggesting that both FTD and NCL-like pathology are present in PGRN patient neurons as compared to isogenic controls. PGRN mutant neurons also show impaired lysosomal proteolysis and decreased activity of the lysosomal enzyme cathepsin D. Furthermore, we find that PGRN interacts with cathepsin D, and that PGRN increases the activity of cathepsin D but not cathepsins B or L. Finally, we show that granulin E, a cleavage product of PGRN, is sufficient to increase cathepsin D activity. This functional relationship between PGRN and cathepsin D provides a possible explanation for overlapping NCL-like pathology observed in patients with mutations in PGRN or CTSD, the gene encoding cathepsin D. Together, our work identifies PGRN as an activator of lysosomal cathepsin D activity, and suggests that decreased cathepsin D activity due to loss of PGRN contributes to both FTD and NCL pathology in a dose-dependent manner.},
author = {Valdez, Clarissa and Wong, Yvette C. and Schwake, Michael and Bu, Guojun and Wszolek, Zbigniew K. and Krainc, Dimitri},
doi = {10.1093/hmg/ddx364},
issn = {14602083},
journal = {Human Molecular Genetics},
number = {24},
pages = {4861--4872},
title = {{Progranulin-mediated deficiency of cathepsin D results in FTD and NCL-like phenotypes in neurons derived from FTD patients}},
url = {https://www.ncbi.nlm.nih.gov/pubmed?myncbishare=dukemlib{\&}dr=abstract{\&}otool=dukemlib{\&}term=+Progranulin-mediated+defciency+of+cathepsin+D+results+in+FTD+and+NCL-like},
volume = {26},
year = {2017}
}
@article{Valdmanis2007,
abstract = {Hereditary spastic paraplegia (HSP) is a progressive upper-motor neurodegenerative disease. The eighth HSP locus, SPG8, is on chromosome 8p24.13. The three families previously linked to the SPG8 locus present with relatively severe, pure spastic paraplegia. We have identified three mutations in the KIAA0196 gene in six families that map to the SPG8 locus. One mutation, V626F, segregated in three large North American families with European ancestry and in one British family. An L619F mutation was found in a Brazilian family. The third mutation, N471D, was identified in a smaller family of European origin and lies in a spectrin domain. None of these mutations were identified in 500 control individuals. Both the L619 and V626 residues are strictly conserved across species and likely have a notable effect on the structure of the protein product strumpellin. Rescue studies with human mRNA injected in zebrafish treated with morpholino oligonucleotides to knock down the endogenous protein showed that mutations at these two residues impaired the normal function of the KIAA0196 gene. However, the function of the 1,159-aa strumpellin protein is relatively unknown. The identification and characterization of the KIAA0196 gene will enable further insight into the pathogenesis of HSP. {\textcopyright} 2006 by The American Society of Human Genetics. All rights reserved.},
author = {Valdmanis, Paul N. and Meijer, Inge A. and Reynolds, Annie and Lei, Adrienne and MacLeod, Patrick and Schlesinger, David and Zatz, Mayana and Reid, Evan and Dion, Patrick A. and Drapeau, Pierre and Rouleau, Guy A.},
doi = {10.1086/510782},
issn = {00029297},
journal = {American Journal of Human Genetics},
title = {{Mutations in the KIAA0196 gene at the SPG8 locus cause hereditary spastic paraplegia}},
year = {2007}
}
@article{Vazdarjanova1998,
abstract = {Evidence that lesions of the basolateral amygdala complex (BLC) impair memory for fear conditioning in rats, measured by lack of 'freezing' behavior in the presence of cues previously paired with footshocks, has suggested that the BLC may he a critical locus for the memory of fear conditioning. However, evidence that BLC lesions may impair unlearned as well as conditioned freezing makes it difficult to interpret the findings of studies assessing conditioned fear with freezing. The present study investigated whether such lesions prevent the expression of several measures of memory for contextual fear conditioning in addition to freezing. On day 1, rats with sham lesions or BLC lesions explored a Y maze. The BLC-lesioned rats (BLC rats) displayed a greater exploratory activity. On day 2, each of the rats was placed in the 'shock' arm of the maze, and all of the sham and half of the BLC rats received footshocks. A 24-hr retention test assessed the freezing, time spent per arm, entries per arm, and initial entry into the shock arm. As previously reported, shocked BLC rats displayed little freezing. However, the other measures indicated that the shocked BLC rats remembered the fear conditioning. They entered less readily and less often and spent less time in the shock arm than did the control nonshocked BLC rats. Compared with the sham rats, the shocked BLC rats entered more quickly and more often and spent more time in the shock arm. These findings indicate that an intact BLC is not essential for the formation and expression of long-term cognitive/explicit memory of contextual fear conditioning.},
author = {Vazdarjanova, Almira and McGaugh, James L.},
doi = {10.1073/pnas.95.25.15003},
file = {::},
issn = {00278424},
journal = {Proceedings of the National Academy of Sciences of the United States of America},
keywords = {Avoidance,Classical conditioning,Excitotoxic lesions,Freezing},
month = {dec},
number = {25},
pages = {15003--15007},
pmid = {9844005},
title = {{Basolateral amygdala is not critical for cognitive memory of contextual fear conditioning}},
url = {http://www.ncbi.nlm.nih.gov/pubmed/9844005 http://www.pubmedcentral.nih.gov/articlerender.fcgi?artid=PMC24565},
volume = {95},
year = {1998}
}
@article{Wan2011,
abstract = {Synaptic transmission mediated by AMPA-type glutamate receptors (AMPARs) is regulated by scaffold proteins in the postsynaptic density. SAP90/PSD-95-associated protein 3 (SAPAP3) is a scaffold protein that is highly expressed in striatal excitatory synapses. While loss of SAPAP3 is known to cause obsessive-compulsive disorder-like behaviors in mice and reduce extracellular field potentials in the striatum, the mechanism by which SAPAP3 regulates excitatory neurotransmission is largely unknown. This study demonstrates that Sapap3 deletion reduces AMPAR-mediated synaptic transmission in striatal medium spiny neurons (MSNs) through postsynaptic endocytosis of AMPARs. Striatal MSNs in Sapap3 KO mice have fewer synapses with AMPAR activity and a higher proportion of silent synapses.Wefurther find that increased metabotropic glutamate receptor 5 (mGluR5) activity in Sapap3 KO mice underlies the decrease in AMPAR synaptic transmission and excessive synapse silencing. These findings suggest a model whereby the normal role of SAPAP3 is to inhibit mGluR5-driven endocytosis of AMPARs. The results of this study provide the first evidence for the mechanism by which the SAPAP family of scaffold proteins regulates AMPAR synaptic activity. {\textcopyright} 2011 the authors.},
author = {Wan, Yehong and Feng, Guoping and Calakos, Nicole},
doi = {10.1523/JNEUROSCI.2533-11.2011},
file = {::},
issn = {02706474},
journal = {Journal of Neuroscience},
month = {nov},
number = {46},
pages = {16685--16691},
pmid = {22090495},
title = {{Sapap3 deletion causes mGluR5-dependent silencing of AMPAR synapses}},
volume = {31},
year = {2011}
}
@article{Wang2018,
abstract = {Networks are ubiquitous in biology where they encode connectivity patterns at all scales of organization, from molecular to the biome. However, biological networks are noisy due to the limitations of measurement technology and inherent natural variation, which can hamper discovery of network patterns and dynamics. We propose Network Enhancement (NE), a method for improving the signal-to-noise ratio of undirected, weighted networks. NE uses a doubly stochastic matrix operator that induces sparsity and provides a closed-form solution that increases spectral eigengap of the input network. As a result, NE removes weak edges, enhances real connections, and leads to better downstream performance. Experiments show that NE improves gene–function prediction by denoising tissue-specific interaction networks, alleviates interpretation of noisy Hi-C contact maps from the human genome, and boosts fine-grained identification accuracy of species. Our results indicate that NE is widely applicable for denoising biological networks.},
archivePrefix = {arXiv},
arxivId = {1805.03327},
author = {Wang, Bo and Pourshafeie, Armin and Zitnik, Marinka and Zhu, Junjie and Bustamante, Carlos D. and Batzoglou, Serafim and Leskovec, Jure},
doi = {10.1038/s41467-018-05469-x},
eprint = {1805.03327},
file = {::},
issn = {20411723},
journal = {Nature Communications},
keywords = {Computational models,Data mining,Machine learning,Network topology},
month = {dec},
number = {1},
pages = {1--8},
publisher = {Nature Publishing Group},
title = {{Network enhancement as a general method to denoise weighted biological networks}},
url = {https://www.nature.com/articles/s41467-018-05469-x},
volume = {9},
year = {2018}
}
@article{Wang2016,
abstract = {Autophagy is protective in cadmium (Cd)-induced oxidative damage. Endoplasmic reticulum (ER) stress has been shown to induce autophagy in a process requiring the unfolded protein response signalling pathways. Cd treatment significantly increased senescence in neuronal cells, which was aggravated by 3-MA or silencing of Atg5 and abolished by rapamycin. Cd increased expression of ER stress regulators Bip, chop, eIf2$\alpha$, and ATF4, and activated autophagy as evidenced by upregulated LC3. Moreover, the ER stress inhibitor mithramycin inhibited the expression of ER stress protein chaperone Bip and blocked autophagic flux. Downregulating Bip significantly blocked the conversion of LC3-I to LC3-II, decreased LC3 puncta formation, and prevented the increase of senescence in PC12 cells. Interestingly, knocking down Bip regulated the expression of p-AMPK, p-AKT and p-s6k induced by Cd. BAPTA, a Bip inhibitor, decreased the expression of p-AMPK and LC3-II, but enhanced neuronal senescence. In addition, we found that siRNA for Bip enhanced GATA4 expression after 6 h Cd exposure in PC12 cells, while rapamycin treatment decreased GATA4 levels induced by 24 h Cd exposure. These results indicate that autophagy degraded GATA4 in a Bip-dependent way. Our findings suggest that autophagy regulated by Bip expression after ER stress suppressed Cd-induced neuronal senescence.},
author = {Wang, Tao and Yuan, Yan and Zou, Hui and Yang, Jinlong and Zhao, Shiwen and Ma, Yonggang and Wang, Yi and Bian, Jianchun and Liu, Xuezhong and Gu, Jianhong and Liu, Zongping and Zhu, Jiaqiao},
doi = {10.1038/srep38091},
file = {::},
issn = {20452322},
journal = {Scientific Reports},
month = {dec},
pmid = {27905509},
publisher = {Nature Publishing Group},
title = {{The ER stress regulator Bip mediates cadmium-induced autophagy and neuronal senescence}},
volume = {6},
year = {2016}
}
@misc{Wang2014,
abstract = {Intracellular protein trafficking plays an important role in neuronal function and survival. Protein misfolding is a common theme found in many neurodegenerative diseases, and intracellular trafficking machinery contributes to the pathological accumulation and clearance of misfolded proteins. Although neurodegenerative diseases exhibit distinct pathological features, abnormal endocytic trafficking is apparent in several neurodegenerative diseases, such as Alzheimer's disease (AD), Down syndrome (DS) and Parkinson's disease (PD). In this review, we will focus on protein sorting defects in three major neurodegenerative diseases, including AD, DS and PD. An important pathological feature of AD is the presence of extracellular senile plaques in the brain. Senile plaques are composed of $\beta$-amyloid (A$\beta$) peptide aggregates. Multiple lines of evidence demonstrate that over-production/aggregation of A$\beta$ in the brain is a primary cause of AD and attenuation of A$\beta$ generation has become a topic of extreme interest in AD research. A$\beta$ is generated from $\beta$-amyloid precursor protein (APP) through sequential cleavage by $\beta$-secretase and the $\gamma$-secretase complex. Alternatively, APP can be cleaved by $\alpha$-secretase within the A$\beta$ domain to release soluble APP$\alpha$ which precludes A$\beta$ generation. DS patients display a strikingly similar pathology to AD patients, including the generation of neuronal amyloid plaques. Moreover, all DS patients develop an AD-like neuropathology by their 40 s. Therefore, understanding the metabolism/processing of APP and how these underlying mechanisms may be pathologically compromised is crucial for future AD and DS therapeutic strategies. Evidence accumulated thus far reveals that synaptic vesicle regulation, endocytic trafficking, and lysosome-mediated autophagy are involved in increased susceptibility to PD. Here we review current knowledge of endosomal trafficking regulation in AD, DS and PD.},
author = {Wang, Xin and Huang, Timothy and Bu, Guojun and Xu, Huaxi},
booktitle = {Molecular neurodegeneration},
doi = {10.1186/1750-1326-9-31},
file = {::},
issn = {17501326},
keywords = {Animals,Extramural,Huaxi Xu,Humans,MEDLINE,N.I.H.,NCBI,NIH,NLM,National Center for Biotechnology Information,National Institutes of Health,National Library of Medicine,Nerve Degeneration / metabolism*,Nerve Degeneration / pathology*,Neurodegenerative Diseases / metabolism*,Neurodegenerative Diseases / pathology*,Non-U.S. Gov't,PMC4237948,Protein Transport / physiology*,PubMed Abstract,Research Support,Review,Timothy Huang,Xin Wang,doi:10.1186/1750-1326-9-31,pmid:25152012},
pages = {31},
publisher = {Mol Neurodegener},
title = {{Dysregulation of protein trafficking in neurodegeneration}},
url = {https://pubmed.ncbi.nlm.nih.gov/25152012/ https://pubmed.ncbi.nlm.nih.gov/25152012/?dopt=Abstract},
volume = {9},
year = {2014}
}
@article{Ward2017,
abstract = {Heterozygous mutations in the GRN gene lead to progranulin (PGRN) haploinsufficiency and cause frontotemporal dementia (FTD), a neurodegenerative syndrome of older adults. Homozygous GRN mutations, on the other hand, lead to complete PGRN loss and cause neuronal ceroid lipofuscinosis (NCL), a lysosomal storage disease usually seen in children. Given that the predominant clinical and pathological features of FTD and NCL are distinct, it is controversial whether the diseasemechanisms associated with complete and partial PGRN loss are similar or distinct.We show that PGRN haploinsufficiency leads to NCL-like features in humans, some occurring before dementia onset. Noninvasive retinal imaging revealed preclinical retinal lipofuscinosis in heterozygous GRN mutation carriers. Increased lipofuscinosis and intracellular NCL-like storage material also occurred in postmortem cortex of heterozygous GRN mutation carriers. Lymphoblasts from heterozygous GRN mutation carriers accumulated prominent NCL-like storagematerial, which could be rescued by normalizing PGRN expression. Fibroblasts from heterozygous GRN mutation carriers showed impaired lysosomal protease activity. Our findings indicate that progranulin haploinsufficiency caused accumulation of NCL-like storagematerial and early retinal abnormalities in humans and implicate lysosomal dysfunction as a central disease process in GRN-Associated FTD and GRN-Associated NCL.},
author = {Ward, Michael E. and Chen, Robert and Huang, Hsin Yi and Ludwig, Connor and Telpoukhovskaia, Maria and Taubes, Ali and Boudin, Helene and Minami, Sakura S. and Reichert, Meredith and Albrecht, Philipp and Gelfand, Jeffrey M. and Cruz-Herranz, Andres and Cordano, Christian and Alavi, Marcel V. and Leslie, Shannon and Seeley, William W. and Miller, Bruce L. and Bigio, Eileen and Mesulam, Marek Marsel and Bogyo, Matthew S. and Mackenzie, Ian R. and Staropoli, John F. and Cotman, Susan L. and Huang, Eric J. and Gan, Li and Green, Ari J.},
doi = {10.1126/scitranslmed.aah5642},
file = {::},
issn = {19466242},
journal = {Science Translational Medicine},
month = {apr},
number = {385},
pmid = {28404863},
publisher = {American Association for the Advancement of Science},
title = {{Individuals with progranulin haploinsufficiency exhibit features of neuronal ceroid lipofuscinosis}},
volume = {9},
year = {2017}
}
@article{Ye2020,
abstract = {Retromer, including Vps35, Vps26, and Vps29, is a protein complex responsible for recycling proteins within the endolysosomal pathway. Although implicated in both Parkinson's and Alzheimer's disease, our understanding of retromer function in the adult brain remains limited, in part because Vps35 and Vps26 are essential for development. In Drosophila, we find that Vps29 is dispensable for embryogenesis but required for retromer function in aging adults, including for synaptic transmission, survival, and locomotion. Unexpectedly, in Vps29 mutants, Vps35 and Vps26 proteins are normally expressed and associated, but retromer is mislocalized from neuropil to soma with the Rab7 GTPase. Further, Vps29 phenotypes are suppressed by reducing Rab7 or overexpressing the GTPase activating protein, TBC1D5. With aging, retromer insufficiency triggers progressive endolysosomal dysfunction, with ultrastructural evidence of impaired substrate clearance and lysosomal stress. Our results reveal the role of Vps29 in retromer localization and function, highlighting requirements for brain homeostasis in aging.},
author = {Ye, Hui and Ojelade, Shamsideen A and Li-Kroeger, David and Zuo, Zhongyuan and Wang, Liping and Li, Yarong and Gu, Jessica Y.J. and Tepass, Ulrich and Rodal, Avital Adah and Bellen, Hugo J and Shulman, Joshua M},
doi = {10.7554/eLife.51977},
file = {::},
issn = {2050084X},
journal = {eLife},
month = {apr},
pmid = {32286230},
publisher = {eLife Sciences Publications, Ltd},
title = {{Retromer subunit, vps29, regulates synaptic transmission and is required for endolysosomal function in the aging brain}},
volume = {9},
year = {2020}
}
@article{Yoshikawa2002,
abstract = {Background: CLC-3 is a member of the CLC chloride channel family and is widely expressed in mammalian tissues. To determine the physiological role of CLC-3, we generated CLC-3-deficient mice (Clcn3-/-) by targeted gene disruption. Results: Together with developmental retardation and higher mortality, the Clcn3-/- mice showed neurological manifestations such as blindness, motor coordination deficit, and spontaneous hyperlocomotion. In histological analysis, the Clcn3-/- mice showed a pattern of progressive degeneration of the retina, hippocampus and ileal mucosa, which resembled the phenotype observed in cathepsin D knockout mice. The defect of cathepsin D results in a lysosomal accumulation of ceroid lipofuscin containing the mitochondrial F1F0 ATPase subunit c. In immunohistochemistry and Western blot analysis, we found that the subunit c was heavily accumulated in the lysosome of Clcn3-/- mice. Furthermore, we detected an elevation in the endosomal pH of the Clcn3-/- mice. Conclusions: These results indicated that the neurodegeneration observed in the Clcn3-/- mice was caused by an abnormality in the machinery which degrades the cellular protein and was associated with the phenotype of neuronal ceroid lipofuscinosis (NCL). The elevated endosomal pH could be an important factor in the pathogenesis of NCL.},
author = {Yoshikawa, Momono and Uchida, Shinichi and Ezaki, Junji and Rai, Tatemitsu and Hayama, Atsushi and Kobayashi, Katsuki and Kida, Yujiro and Noda, Masaki and Koike, Masato and Uchiyama, Yasuo and Marumo, Fumiaki and Kominami, Eiki and Sasaki, Sei},
doi = {10.1046/j.1365-2443.2002.00539.x},
file = {::},
issn = {13569597},
journal = {Genes to Cells},
number = {6},
pages = {597--605},
title = {{CLC-3 deficiency leads to phenotypes similar to human neuronal ceroid lipofuscinosis}},
volume = {7},
year = {2002}
}
@article{Zech2011,
abstract = {The actin cytoskeleton provides scaffolding and physical force to effect fundamental processes such as motility, cytokinesis and vesicle trafficking. The Arp2/3 complex nucleates actin structures and contributes to endocytic vesicle invagination and trafficking away from the plasma membrane. Internalisation and directed recycling of integrins are major driving forces for invasive cell motility and potentially for cancer metastasis. Here, we describe a direct requirement for WASH and Arp2/3-mediated actin polymerisation on the endosomal membrane system for $\alpha$5$\beta$1 integrin recycling. WASH regulates the trafficking of endosomal $\alpha$5$\beta$1 integrin to the plasma membrane and is fundamental for integrin-driven cell morphology changes and integrin-mediated cancer cell invasion. Thus, we implicate WASH and Arp2/3-driven actin nucleation in receptor recycling leading to invasive motility. {\textcopyright} 2011. Published by The Company of Biologists Ltd.},
author = {Zech, Tobias and Calaminus, Simon D.J. and Caswell, Patrick and Spence, Heather J. and Carnell, Michael and Insall, Robert H. and Norman, Jim and Machesky, Laura M.},
doi = {10.1242/jcs.080986},
file = {::},
issn = {00219533},
journal = {Journal of Cell Science},
keywords = {Actin assembly,Arp2/3 complex,Integrin recycling,Invasion,Membrane trafficking},
month = {nov},
number = {22},
pages = {3753--3759},
pmid = {22114305},
publisher = {The Company of Biologists Ltd},
title = {{The Arp2/3 activator WASH regulates $\alpha$5$\beta$1-integrin-mediated invasive migration}},
volume = {124},
year = {2011}
}
@misc{Zhang2019,
abstract = {Stroke is one of the leading causes of death worldwide, and the majority of cerebral stroke is caused by occlusion of cerebral circulation, which eventually leads to brain infarction. Although stroke occurs mainly in the aged population, most animal models for experimental stroke in vivo almost universally rely on young-adult rodents for the evaluation of neuropathological, neurological, or behavioral outcomes after stroke due to their greater availability, lower cost, and fewer health problems. However, it is well established that aged animals differ from young animals in terms of physiology, neurochemistry, and behavior. Stroke-induced changes are more pronounced with advancing age. Therefore, the overlooked role of age in animal models of stroke could have an impact on data quality and hinder the translation of rodent models to humans. In addition to aging, other factors also influence functional performance after ischemic stroke. In this article, we summarize the differences between young and aged animals, the impact of age, sex and animal strains on performance and outcome in animal models of stroke and emphasize age as a key factor in preclinical stroke studies.},
author = {Zhang, Hongxia and Lin, Siyang and Chen, Xudong and Gu, Lei and Zhu, Xiaohong and Zhang, Yinuo and Reyes, Kassandra and Wang, Brian and Jin, Kunlin},
booktitle = {Neurochemistry International},
doi = {10.1016/j.neuint.2018.10.005},
file = {::},
issn = {18729754},
keywords = {Aging,Ischemia stroke,Lifespan,Preclinical study,Risk factors},
month = {jul},
pages = {2--11},
publisher = {Elsevier Ltd},
title = {{The effect of age, sex and strains on the performance and outcome in animal models of stroke}},
url = {https://pubmed.ncbi.nlm.nih.gov/30291954/ https://pubmed.ncbi.nlm.nih.gov/30291954/?from{\_}single{\_}result=The+effect+of+age{\%}2C+sex+and+strains+on+the+performance+and+outcome+in+animal+models+of+stroke},
volume = {127},
year = {2019}
}
@incollection{Zhou2018,
abstract = {Accumulating evidence suggests that progranulin is essential for proper lysosomal function. Progranulin is a lysosomal resident protein and sortilin has been demonstrated to be the lysosomal trafficking receptor for progranulin. Here we describe the methods used to study the interaction between progranulin and sortilin, as well as the critical role of sortilin in mediating the lysosomal delivery of progranulin.},
author = {Zhou, Xiaolai and Sullivan, Peter M. and Paushter, Daniel H. and Hu, Fenghua},
booktitle = {Methods in Molecular Biology},
doi = {10.1007/978-1-4939-8559-3_18},
issn = {10643745},
keywords = {Frontotemporal lobar degeneration,Lysosome,Neuronal ceroid lipofuscinosis,Progranulin,Sortilin,Trafficking},
pages = {269--288},
publisher = {Humana Press Inc.},
title = {{The interaction between progranulin with sortilin and the lysosome}},
volume = {1806},
year = {2018}
}
@article{Zhou2019,
abstract = {A critical component in the interpretation of systems-level studies is the inference of enriched biological pathways and protein complexes contained within OMICs datasets. Successful analysis requires the integration of a broad set of current biological databases and the application of a robust analytical pipeline to produce readily interpretable results. Metascape is a web-based portal designed to provide a comprehensive gene list annotation and analysis resource for experimental biologists. In terms of design features, Metascape combines functional enrichment, interactome analysis, gene annotation, and membership search to leverage over 40 independent knowledgebases within one integrated portal. Additionally, it facilitates comparative analyses of datasets across multiple independent and orthogonal experiments. Metascape provides a significantly simplified user experience through a one-click Express Analysis interface to generate interpretable outputs. Taken together, Metascape is an effective and efficient tool for experimental biologists to comprehensively analyze and interpret OMICs-based studies in the big data era.},
author = {Zhou, Yingyao and Zhou, Bin and Pache, Lars and Chang, Max and Khodabakhshi, Alireza Hadj and Tanaseichuk, Olga and Benner, Christopher and Chanda, Sumit K.},
doi = {10.1038/s41467-019-09234-6},
file = {::},
issn = {20411723},
journal = {Nature Communications},
month = {dec},
number = {1},
pmid = {30944313},
publisher = {Nature Publishing Group},
title = {{Metascape provides a biologist-oriented resource for the analysis of systems-level datasets}},
volume = {10},
year = {2019}
}
@article{Zimprich2011,
abstract = {To identify rare causal variants in late-onset Parkinson disease (PD), we investigated an Austrian family with 16 affected individuals by exome sequencing. We found a missense mutation, c.1858G{\textgreater}A (p.Asp620Asn), in the VPS35 gene in all seven affected family members who are alive. By screening additional PD cases, we saw the same variant cosegregating with the disease in an autosomal-dominant mode with high but incomplete penetrance in two further families with five and ten affected members, respectively. The mean age of onset in the affected individuals was 53 years. Genotyping showed that the shared haplotype extends across 65 kilobases around VPS35. Screening the entire VPS35 coding sequence in an additional 860 cases and 1014 controls revealed six further nonsynonymous missense variants. Three were only present in cases, two were only present in controls, and one was present in cases and controls. The familial mutation p.Asp620Asn and a further variant, c.1570C{\textgreater}T (p.Arg524Trp), detected in a sporadic PD case were predicted to be damaging by sequence-based and molecular-dynamics analyses. VPS35 is a component of the retromer complex and mediates retrograde transport between endosomes and the trans-Golgi network, and it has recently been found to be involved in Alzheimer disease. {\textcopyright} 2011 by The American Society of Human Genetics. All rights reserved.},
author = {Zimprich, Alexander and Benet-Pag{\`{e}}s, Anna and Struhal, Walter and Graf, Elisabeth and Eck, Sebastian H. and Offman, Marc N. and Haubenberger, Dietrich and Spielberger, Sabine and Schulte, Eva C. and Lichtner, Peter and Rossle, Shaila C. and Klopp, Norman and Wolf, Elisabeth and Seppi, Klaus and Pirker, Walter and Presslauer, Stefan and Mollenhauer, Brit and Katzenschlager, Regina and Foki, Thomas and Hotzy, Christoph and Reinthaler, Eva and Harutyunyan, Ashot and Kralovics, Robert and Peters, Annette and Zimprich, Fritz and Br{\"{u}}cke, Thomas and Poewe, Werner and Auff, Eduard and Trenkwalder, Claudia and Rost, Burkhard and Ransmayr, Gerhard and Winkelmann, Juliane and Meitinger, Thomas and Strom, Tim M.},
doi = {10.1016/j.ajhg.2011.06.008},
file = {::},
issn = {15376605},
journal = {American Journal of Human Genetics},
month = {jul},
number = {1},
pages = {168--175},
publisher = {Cell Press},
title = {{A mutation in VPS35, encoding a subunit of the retromer complex, causes late-onset parkinson disease}},
volume = {89},
year = {2011}
}
